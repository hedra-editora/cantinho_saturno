Asas de Saturno

Maria João Cantinho

Ao Meu pai (\emph{In Memoriam), }que me legou a paixão pela música.

Em agradecimento a António Cabrita, que reviu pacientemente o livro.

\emph{Words move, music moves\\
Only in time; but that which is only living\\
Can only die. Words, after speech, reach\\
Into the silence. Only by the form, the pattern,\\
Can words or music reach\\
The stillness, as a Chinese jar still\\
Moves perpetually in its stillness.}

\textbf{T.S.Eliot, }\emph{Four Quartets}

\section{\textbf{A CASA}}

No ar ondula a cortina das janelas. Ao longe, uma fileira de prédios,
nem novos nem velhos, com as suas marquises fechadas e as fachadas
escuras, sujas de humidade e de poluição. Ao menos não são tão altos que
tapem o sol e impeçam a circulação do vento, sobretudo ao final da
tarde.

A secretária reflecte o brilho da luz que entra. Há a desordem habitual
sobre a secretária: lápis, canetas, folhas de papel rabiscadas com
minúcia, uma caligrafia minúscula e metódica. Uma pilha de livros,
aparentemente desordenada, mas da qual ele conhece a sua secreta ordem.

O rapaz assenta o queixo sobre a mão direita e tenta concentrar-se numa
frase de Spinoza que transcrevera, na margem da folha, mas o seu
espírito está bem longe dali. A música invade o quarto. À sua volta, a
realidade sacode-o: um cão que ladra na rua, o barulho de tacões de
saltos altos, correndo apressadamente, arranca-o ao devaneio. Uma luz
crepuscular fustiga-lhe os lábios, deixando neles o seu ténue calor.

Observa o sol a declinar. Uma luz rubra desaba sobre os móveis e tudo
ganha uma vida própria, infiltrando-se e acendendo as ripas do soalho;
grãos de poeira rodopiando sobre si e trepando secretamente os acordes
da sinfonia. Daqui a pouco será noite. Sobrevirá esse silêncio, cobrindo
as árvores do jardim, deixando nuas as ramadas das árvores e o mundo
recolher-se-á na escuridão.

Florimundo fecha os olhos e ouve o murmúrio da folhagem crescendo, num
som que lhe chega da memória, ao longe, e esse chilreio é-lhe familiar,
antigo. Os seus olhos batem na iluminação dos candeeiros, emergindo
fantasmaticamente. E o som do piano escoa-se para dentro de si,
transformando-o, acolhendo-o.

O odor das flores do terraço perturba-o, entrando pela janela,
desperta-o desse limbo para onde a música sempre o conduz, esquecido de
si.

O instante fende-se e Florimundo acede a um qualquer ponto obscuro de si
mesmo. Por momentos é arrastado até ao passado. Ao tempo em que o vento
era um murmúrio, soprando nas dunas da sua infância, ao longe
avistando-se a casa. Esse tempo escava-lhe na pele uma dor vaga, a
cicatriz reabrindo-se. Descontínua, a torrente de imagens flui
livremente, indo e vindo numa lentidão aquática. O som do violino
perde-se na dobra do tempo, da carne, faz-se voz na noite que desaba. É
este saber que o move e o transporta para um lugar que nada alcança,
neste desvão de ser e de escombros.

O casarão vermelho sobre os penhascos, sobranceiro ao mar, assalta-lhe a
memória. E, como se estivesse a olhar para um filme antigo, tudo se
fragmenta, reacendendo o mundo petrificado. A casa rasga o horizonte, na
sua solidão imponente sobre o mar. Surge-lhe a imagem da criança
debruçando-se curiosa sobre um carreiro de formigas caminhando em fila,
imperturbáveis. Era preciso descobrir para onde caminhavam essas
obreiras, absorvidas na sua interminável tarefa de sobrevivência.

Florimundo levanta-se. Doem-lhe as pernas pela insistência da posição.
Está há horas sentado, concentrado na leitura de um livro de estética
musical. Espreguiça-se e encosta a cabeça à vidraça fria e brilhante. Os
olhos são arrastados pelo pardal pousado nas grades da varanda. São as
pequenas coisas que o emocionam, arrancando-o a um mundo frio, de noções
abstractas, vazias. De súbito, o cansaço abate-se sobre si, deixando-o
suspenso entre a luz que deseja e a sua própria escuridão. Essa luz do
passado.

O passo arrastado de Clara fá-lo lembrar-se que é tarde. Nada avançou
significativamente, às voltas com os seus pensamentos. Há muito que lhe
falta algo para o guiar. Sorri, com ar triste e ensimesmado.

- Comeste alguma coisa? - O ar dela é o de quem já sabe a resposta,
enfiando as mãos dentro dos bolsos. - Está cá uma nortada! O vento
rasga-me a pele.

Ele olhou-a como se procurasse o rasgão da pele, a ferida, entre as
muitas que ela tinha. Como de costume, não encontrava nada para dizer.
Tal como a solidão, o silêncio transformara-se numa segunda veste, à
força dessa mudez, uma melancolia cúmplice entre ambos.

A casa é a sua própria infância, afundando-se no mar e numa escuridão
líquida.

Daquilo que ele se lembra melhor é a sua solidão. O aspecto inóspito que
rasga o horizonte, dando à paisagem o seu ar selvagem e bravio, do longo
areal batido pelo vento durante todo o ano. Das ervas solitárias e
abandonadas no areal selvagem, onde se amontoava a areia, formando
dunas. E do cheiro a maresia, às vezes violento porque o vento fustigava
o mar e trazia um cheiro de água viva, das ondas a rebentar, no meio do
silêncio. Reconstrói à sua volta uma imagem, não sabe bem se memória ou
sonho, e quase consegue ouvir o uivo do vento, ver o salitre roendo as
grossas paredes da casa e trespassando-a de humidade.

Lembra-se delas, das aves, das suas asas a rasgar a linha do céu,
cortando a penumbra. De quando acordava em criança, flutuando entre o
seu voo lento e o azul da madrugada. Do canto das aves junto de si, do
lado de fora da janela, perdurando na noite.

Quando o sol irrompia, o corpo enchia-se de um formigueiro e de vontade
de ir até à praia. O rapaz descia pela falésia, através de uma escadaria
interminável em madeira. A madeira rangia, inchada pela humidade. O
cheiro da maresia impregnava-a. Ao fundo havia o mar, essa paisagem
interminável, sempre presente e viva.

A mãe cantava e a sua voz rasgava o frio da manhã. Florimundo caminhava
para ela, que o esperava de braços abertos, o corpo magro e seco,
envolto num vestido já desbotado, que lhe reflectia a cor dos olhos.
Lembra-se dos seus lábios, das sardas no rosto e dos anéis alourados do
cabelo. É difícil esquecer essa leveza que havia nos seus gestos rápidos
e nervosos. Difícil era também esquecer-lhe a voz. Não raro, via-a
avançar ao longe, no passo ligeiro que a tornava deslizante, pelo
quintal. Espreitava-a silenciosamente pela janela. Era essa a medida
exacta do seu mundo.

Depois, sabia que ela estava ali. Chamava-a pelo seu nome: Clara. Às
vezes, Gabriel brincava, chamando-lhe ``claridade'' e a ideia
divertia-o. O nome da mãe era mais do que um nome, uma concha que trazia
em si um segredo, com o dom de rasgar a escuridão com um simples relance
dos seus olhos azuis.

E vinha-lhe às narinas o odor do café, acabado de fazer, do pão quente à
sua espera, ritual de cada manhã. Clara esperava-o, à entrada da cozinha
grande e antiga, iluminada pela luz que entrava pela janela. E, na maior
parte, das vezes, a porta que dava para o terraço estava aberta, se
fazia calor.

- Vai calçar-te. - Os olhos dela fingiam mal a rispidez e assomava um
sorriso que não resistia a aparecer-lhe ao canto dos lábios.

O silêncio dos olhos do gato espreitava-o, do fundo do seu olhar
amarelado.

- Tira-me esse gato daí. Ele não pode estar sentado aí. - Pedia-lhe a
mãe. - Que mania a do gato, de se sentar sempre na mesma cadeira, como
se fosse uma pessoa.

O rapaz agarrava no gato e acariciava-o. Colocava-o com suavidade no
chão.

De cada vez que regressa, ainda vê o casarão vermelho enclausurado entre
as árvores. Envolto nessa cápsula que o tempo lhe trouxe. A erva cresce
em redor das rosas, em desleixo emaranhado. Mas subsiste o desejo
nostálgico do regresso, sobretudo quando os dias mais longos de
Primavera retornam e o pólen flutua, quando o ar é invadido pelos
esporos que se agarram à roupa, à pele.

Florimundo regressa incansavelmente àquela imagem que lhe traz a
presença da infância concentrada, lembrando-se de como descia a
escadaria a correr. Agora, descer a escadaria, que apodrecera com os
anos, tornara-se perigoso. Com a humidade e a degradação dos anos, a
madeira tornara-se frágil, estremecendo perigosamente, prestes a ceder.
Era o mar a puxá-lo para si, a trazê-lo até à sua orla, onde gostava de
brincar e de andar na areia molhada e grossa, de enterrar os pés na
areia, seguindo trilhos que só ele e o pai conheciam. Até meados da
Primavera, a praia estava praticamente deserta. Apenas as gaivotas a
invadiam, era o seu território e ocupavam-na às centenas, sem haver
perturbação da presença humana. Chegavam ali com o seu voo lento, quase
solene, e escolhiam o lugar para fazerem os seus ninhos, abrigando-se
nas pequenas cavidades naturais, entre as escarpas que as protegiam das
intempéries e do vento, sobretudo no Inverno, quando este se tornava
gélido, na costa voltada a Norte.

Florimundo perdia-se horas a fio, de olhos presos ao mar. Começava por
ser um ponto de fuga onde cravava os olhos, deixava-se arrastar pelas
fantasias que nasciam dos livros que lia, navios imaginários sulcando o
mar, como nas histórias de piratas e exploradores que o pai lhe passava
para as mãos. Olhava para esses barcos que partiam, largando linhas de
espuma invisíveis e que só ele via. Hoje era o capitão Ahab percorrendo
a senda da sua branca baleia, amanhã, talvez um pirata sanguinário, um
herói dos mares, cheio de marcas pelas lutas que travara.

Fosse o que fosse esse mar, esse sonho ou desejo, havia nele a
necessidade de se deixar ir. Outras vezes ouvia, apenas. Pela alba, mal
o sol havia nascido, já ele percorria a praia até à zona afastada das
dunas. Atirava-se para a areia e rolava por ali abaixo, ouvindo o vento,
a valsa lenta do mar, o piar das gaivotas, deitava-se de barriga para o
ar e ali ficava preguiçosamente, deitado de costas, os braços atrás da
cabeça, apoiando a nuca, atento à música das ondas, ao som dos jactos de
água salgada infiltrando-se nas fissuras dos rochedos e retornando à sua
origem. Tão incrível era a profusão de sons que o ouvido desatento
perdia, as vozes que o vento trazia, ora em sussurro, ora mais
violentamente. E, no entanto, esses sons tinham um padrão essencial que
absorvia e integrava essa diversidade. E, de novo, a música se renovava
até à sua vertigem, sempre, sempre, sem tempo nem espaço, incessante.

Fora Gabriel, o pai, que lhe havia ensinado esse gesto de concentração.
Dizia-lhe que na natureza nada era desperdício, mas a mais perfeita
economia do canto e da criação musical. Da vida que se confundia com o
ritmo da natureza. Um padrão regular e repetitivo, que se aplicava a
tudo o que vivia. Um ouvido atento, dizia-lhe Gabriel, poderia ir até à
perfeição da audição de um animal. Os homens, explicara-lhe, quando
ainda era muito pequeno, pareciam ter perdido essa capacidade de
perceber o ritmo, a respiração das coisas, a regularidade das suas leis.
Haviam-se evadido no labirinto das ``suas pequenas vidas'', achavam que
tudo se reduzia ao que viam e apenas a isso. A atenção era coisa de
velhos, de gente que não tinha nada para fazer. Ou, então, de loucos que
viam coisas que os outros não viam.

Às vezes, Florimundo avistava ao longe um ou outro vulto de solitário
pescador, mas havia quase sempre aquele deserto, um universo intocado e
selvagem, entre o amarelo esmaecido da areia e o azul do mar, uma névoa
de espuma e salitre. Homens que se integravam na paisagem, como figuras
de sempre, confundindo as suas roupas com o tom acastanhado dos
penhascos, as mãos grossas e firmes, os dedos deformados pelas artrites
e pelos calos.

Ainda hoje Florimundo sabe que pertence a este mundo fragmentado e que
se apresenta em estilhaços na sua memória. Mesmo que o presente se tenha
sobreposto a tudo, com o seu véu de tristeza, a sua pertença está
entranhada na música que ouve. Nada poderia roubar-lhe essa pertença. Ao
mesmo tempo vem a dor, a sombra que esconde a luz de cada criatura.
Conhece o silêncio que tudo envolve em si, a repetição do fragor das
ondas, atirando-se contra os penhascos. Cada lembrança evoca um mundo
que escapa às palavras. Gabriel ensinara-o a ouvir essa linguagem íntima
da matéria e chamava-lhe carinhosamente ``meu anjo do futuro''.

Em torno da casa havia sempre as aves. Às centenas, iam e vinham, de
todo o lado, misturando-se e desaparecendo, em formação, desordenadas,
rasantes, por vezes rápidas, outras vezes tão lentamente que pareciam
flutuar na espuma branca que se elevava das ondas, batendo contra as
rochas. Aninhavam-se nas fissuras das falésias, escuras e altas, onde
procriavam e procuravam alimento. Esse foi o primeiro som que aprendeu e
conheceu. Ao mesmo tempo que aprendeu a falar. Cedo começou a imitá-las
e rapidamente aprendeu que era simples comunicar com elas. Percebeu,
mesmo antes de falar, que já conhecia uma outra fala, uma outra
linguagem, uma outra música.

As gaivotas eram capazes de passar horas seguidas, imóveis e
silenciosas, esperando o peixe. Para elas, não havia tempo. Alimentavam
as suas crias e estas partiam quando se tornavam aptas. Depois vinham
outras aves, substituíam as que morriam e tudo recomeçava. Sem
sobressaltos ou angústias. O mar estava ali sempre, imemorial, como as
aves, a areia e os sonhos.

\section{O CANTO}

Um dia, Florimundo encontrou um cadáver de uma gaivota e debruçou-se
sobre esse corpo inerte, fascinado. Uma frieza de mármore, ainda que o
vento fizesse oscilar as penas, dando a impressão de que algo ainda se
movia. Pela primeira vez viu uma criatura morta. A princípio não
compreendeu, a ausência de movimento. Já tinha visto insectos mortos,
mas não era a mesma coisa. As aves eram mais próximas de si, como o
gato, a mãe ou o pai. Teve medo, a princípio, de tocar naquele corpo,
qualquer coisa o repelia e não sabia o quê. Não sabia explicar, não
compreendia, limitou-se a olhar para aquele corpo a tentar compreender o
que era já não estar vivo. Depois deu-se conta de que isso podia
acontecer com tudo, consigo próprio, deixar de respirar, deixar de ter o
corpo quente.

Abraçou o corpo e aqueceu-o. Começou a cantar baixinho. O vento
percorria a praia, levantando uma fina camada de areia que rodopiava
sobre si. Tudo era atravessado pela desolação. Uma necessidade interior
impelia-o a cantar e era como se o canto suavizasse essa estranheza que
sentia. A opacidade daquele corpo, daquele ser que não reagia. Cantava
sem compreender, seguindo a voz do vento, numa língua que ele
desconhecia. E os sons saíam-lhe como se sempre tivesse conhecido essa
língua que não compreendia.

Florimundo percebeu que o seu canto tinha um estranho poder sobre o
corpo da ave. Sentiu que ela retomava o seu calor e se movera
ligeiramente. Horrorizado, deixou-a cair e desatou a correr pela praia,
completamente desnorteado. Soube-o, nesse instante. Como os animais ou
os homens no seu estado mais puro e instintivo, a criança sabia que não
devia usar a força cujos limites desconhecia.

Gabriel encontrava-se a poucos metros. A princípio observara a cena
entre o rapaz e o pássaro. Ficou atónito. Quando ele fugiu, aproximou-se
da ave e também ele se assustou. Soube-o também Gabriel. Olhou para os
caracóis loiros, a oscilar ao vento, o rosto branco e assustado, à
medida que se afastava. O pai foi encontrá-lo a uma larga centena de
metros dali. Escondera-se por detrás de uma duna. Abraçou-o, fingindo
que nada se tinha passado, e trouxe-o para o pé do mar.

O rapaz lembra o pai descalço, sentado na areia, o rosto mergulhado na
sombra da tarde, ao final do dia. Estavam sós, quando Gabriel lhe falou
pela primeira vez desse estranho e indomável deus, a primeira imagem, na
origem da vida, da arte e da música. E de poderes inexplicáveis que
alguns seres humanos possuíam.

Do lugar onde se encontrava via o rosto do pai no seu nítido contorno,
em contraluz, com o cabelo ligeiramente comprido e solto, caindo-lhe
sobre o rosto. Estava de frente para ele. E sentiu-se amedrontado.
Descobrira o horror de um rosto na escuridão, mesmo sendo o do seu pai.
Nesse momento, o sobressalto do coração deu-lhe a primeira das muitas
indicações que teria do mal. Uma sombra devorando o rosto.

Lembrou-se de que poderia ficar horas a ouvir Gabriel falar das suas
viagens, com os seus gestos largos e lentos, das lendas e histórias
mitológicas, contendo significados estranhos. Esse mundo em que os
deuses, os duendes e os seres mágicos convivem com os homens e a noite
corre densa, profunda. Não a noite do esquecimento e do sono, mas a
grande mãe dos sonhos e da poesia, da metamorfose.

De Gabriel guarda ainda uma imagem vaga, vinda de uma memória longínqua,
com os seus traços fortes e mediterrânicos meio esbatidos, o seu rosto
de maxilares proeminentes e angular, como se apenas fosse possível
lembrá-lo com a neblina do mar, avançando pela praia, ou no alpendre da
casa, sob a lâmpada fosca, em redor da qual esvoaçavam insectos atraídos
pela luz. Às vezes eram besouros que se atiravam cegamente contra o
candeeiro, fazendo-o oscilar. Depois estatelavam-se pesadamente no chão.

Havia noites em que a luz da lua atravessava o jardim e deixava uma
cintilação prateada, iluminando-os aos três. O pai encostava-se ao muro
do alpendre e lia alto. Possuía uma voz lenta e grave, de timbre claro.
Jamais ouviria alguém ler poesia como Gabriel o fazia, erguendo as suas
mãos compridas e de dedos longos e esguios que pareciam levantar voo, os
versos a demorar-se na sua voz, até caírem no silêncio. Mergulhava num
estado meio hipnótico que o rapaz se habituara a ver-lhe.

A mãe semicerrava os olhos. Mas não escondia uma certa apreensão diante
do seu alheamento. Esse transe, à medida que o tempo passava,
acentuava-se cada vez mais, como se estivesse fora do mundo, e Clara
sentia-o. Como se visse para além do visível e o fixasse.

Ouvira vezes sem conta o poema ``Burnt Norton'' de Eliot, um dos
favoritos de Gabriel. Conhecia-o de cor e salteado. Ainda não
compreendia o significado das palavras e já se sentia arrastado,
esmagado pelo poema, pelo canto, que se insinuava no cérebro e se fundia
com imagens, fragmentos.

\emph{``Mas para quê/ Perturbar a poeira numa taça de folhas de rosa/
Não sei./ Outros ecos/ Habitam o jardim. Vamos segui-los?/ Depressa,
disse a ave, procura-os, procura-os/ Na volta do caminho. Através do
primeiro portão,/ No nosso primeiro mundo./ Ali estavam eles, dignos,
invisíveis (...)''.}

Sentada na cadeira, a mãe ouvia os versos, empurrando as costas da
cadeira contra a parede. Tinha o cabelo solto, movendo-se com a brisa.
Sonolenta. Reflexos da lua acariciavam-lhe o pescoço, subiam-lhe a nuca,
afogando-se nos anéis do cabelo. Florimundo deitava a cabeça no seu
colo, enquanto ela lhe afagava o cabelo.

O pai sentava-se no alpendre e ensinava-lhe o nome das constelações,
enquanto olhavam para o céu e procuravam pequenos pontos luminosos na
noite. Tudo, no seu universo, tinha uma correspondência precisa. A
palavra, o canto, o som, o número. Um alfabeto, dizia-lhe o pai,
sorrindo na escuridão. Uma escrita indecifrável. Florimundo não
compreendia porque o mundo não podia ser simples e acessível e brincava
às escondidas. Era então que o sono vinha e ouvia a voz de Gabriel a
flutuar ao fundo, como um peixe luminoso numa camada indistinta de sons.

Gabriel voltava-se, detendo-se no rosto adormecido do filho. Pegava nele
e abraçava-o, deitava-o na cama a ouvir-lhe o manso ressonar que se
confundia com o vento na folhagem.

Alheia ao sonho de Gabriel, que durante anos perfilhara, Clara queria
voltar à cidade, trabalhar, conviver, sair do ambiente opressivo, porém,
ele não lho consentia. Dizia-se demasiado feliz para abdicar da sua
solidão perfeita. "E Florimundo?", perguntava ela, insistindo que o
rapaz deveria ir à escola. A mãe receava que o miúdo se tornasse
selvagem, incapaz de acatar leis, de as respeitar. Discutia isso com
Gabriel. E ele respondia-lhe, com alguma tristeza na voz e no olhar:
"Porque desejas tanto que ele seja igual aos outros? De que tens medo?"

Clara calava-se. Qualquer coisa dentro dela fazia-a pensar que, depois
daquela infância, o rapaz se revelasse incapaz de aceitar a dureza e a
crueldade da realidade, do mundo. Era de tal modo protegido que ela
receava o pior, como o alheamento que via crescer de dia para dia em
Gabriel.

No recanto mais afastado da casa, no primeiro andar, iluminado por uma
pequena janela que dava directamente sobre o mar e para o jardim, ficava
a biblioteca. No princípio, Clara gostara de brincar, dizendo que ali
era a ``torre do alquimista''. Depois cansou-se e começou a achar que
era uma ideia estafada. As paredes forradas de livros, de cima a baixo,
o espaço que já não chegava e os livros amontoavam-se e subiam como
trepadeiras, proliferavam por toda a sala em desordem incalculável.
Cobriam-se de poeira, o que dava a entender que alguns haviam sido
esquecidos e deixados ao acaso.

Às vezes, Florimundo penetrava nesse espaço caótico, onde os livros se
acumulavam uns em cima dos outros, pois Gabriel não gostava que os
tirassem da sua ordem. Enfurecia-o não os encontrar à primeira. O miúdo
sentava-se no chão silenciosamente e folheava os livros sobre pássaros,
flores, pedras, mitos e lendas, que o pai lhe ia passando para a mão.
Outros era ele que os escolhia. Ficavam assim durante muito tempo, ambos
concentrados, até à chegada da noite. Enquanto isso, Gabriel escrevia ou
lia, muitas vezes desenhava.

No entanto, se a biblioteca lhe parecia um espaço fascinante quando o
pai ali trabalhava, assim que o sol declinava, ele ganhava um certo
receio de entrar nela. Ao olhar para as paredes, parecia-lhe ver
estranhas sombras, movendo-se com lentidão, provavelmente reflexos das
ramadas das árvores. Ao longo da parede, pequenos corpos silenciosos e
imateriais. Escuros. Talvez os livros ganhassem vida própria e se
deslocassem e as histórias e personagens ganhassem vida.

E depois havia também as perturbadoras gravuras de Piranesi e as imagens
de bestiários antigos, que o aterrorizavam. O pai bem tentava
explicar-lhe que tudo aquilo eram projecções da sua mente, que nada
tinham de ameaçador, mas nada feito. O rapazinho corria a refugiar-se
nos braços da mãe. E pedia-lhe que ela cantasse, para que afastasse o
medo. Murmurava o nome da mãe, baixinho, como um talismã.

Clara queria protegê-los. Tinha sérias dúvidas de que Gabriel chegasse
algum dia a publicar aquilo a que ele chamava a sua obra. Ele tinha-se
proposto escrever um tratado sobre seres imaginários, animais, anjos,
feiticeiros, fadas, monstros, seres de passagem, procurando concentrar
nessas imagens as suas obsessões, desde muito jovem. Tinha dedicado uma
parte da sua vida ao estudo da vasta galeria de seres imaginários que
existiam na literatura. Mantinha caixotes guardados e empilhados,
resultantes das suas investigações e que mantinha intactos, muitos
desenhos, feitos por ele próprio.

Os anos haviam passado e Gabriel não parava de ler tratados antigos,
poetas arcaicos e autores medievais, cabalistas, herméticos que lhe
despertavam uma infinita curiosidade, mas que o desviavam do seu
objectivo. Quando se lhe referia, falava na \emph{Obra}, com letra
maiúscula, e punha-se sério como se estivesse a desvendar segredos.
Quando falavam nisso, ele próprio dizia-lhe que receava fazê-lo. Havia
naquele dédalo infindável e construído como uma sobreposição de
referências e histórias vividas uma profusão de personagens que se
metamorfoseavam e recusavam a fixar-se na sua escrita. Vida e ficção
fundiam-se e entrelaçavam-se. Por brincadeira, Gabriel dizia-lhe muitas
vezes que queria escrever a vida ou que a sua própria vida se escrevia
através dele.

O pensamento de Gabriel, Clara sabia-o, era ágil, pulava de clássico em
clássico, dos pensadores gregos aos gnósticos e aos poetas orientais,
indianos e chineses, onde se deixava enredar infinitamente. A tragédia
grega comovia-o intensamente. Propusera a si mesmo que haveria de
escrever uma tragédia moderna, procurando recuperar o sentimento que a
alimentasse.

E atirava-se à escrita, incansavelmente, numa concentração que excluía o
mundo. Não dava pelo passar das horas, numa suspensão que fazia pairar a
sua própria vida e integrá-la nas páginas dos seus livros. As palavras,
esses seres com vida própria, pequenos animais autónomos e fechados na
sua música, resistiam-lhe. Procurava-as como um caçador, organizava-as e
voltava a organizá-las, numa nova sequência, recombinava-as, mas
parecia-lhe ainda que não atingira a consistência da obra que procurava.
Aspirava a um ideal capaz de o resgatar ao real e ao desgaste das
palavras, à inutilidade dos gestos.

Esta forma de viver radicalmente a escrita e a criação, e que o rapaz
viria a herdar, afastara-o definitivamente do mundo e dos pequenos
gestos, das conversas quotidianas, dos outros, dos amigos que o faziam
perder tempo. E sempre que fazia um esforço para romper um pouco a sua
barreira, procurando agradar a Clara, sofria horrores que ela jamais
poderia adivinhar. Só o simples facto de dar aulas, provocava-lhe imenso
sofrimento.

Depois do trabalho, saía exausto, para apanhar ar, mesmo quando chovia,
deambulando entre os caminhos da falésia. Via na escrita o seu espelho,
onde se descobria cada vez mais perto de um fundo qualquer que não
ousava nomear. Essa dilaceração insinuava-se como um golpe afiado de
vento na pele e ia alastrando surdamente por toda a sua epiderme e pelo
seu sangue, sufocando-o.

As mãos prendiam-no à realidade. Eram ainda as suas mãos que ele
reconhecia, enquanto procurava aquela frase justa, certeira, com os
dedos que pareciam navegar na escuridão e procurava a claridade do mar,
quando tudo se aquietava. Quando as palavras-animais ou as
palavras-concha já não o maceravam nem o obrigavam a perseguir-se a si
próprio. Eram as mesmas mãos que mergulhava na areia, à procura de uma
luz qualquer, de uma matéria limpa. Não havia beleza que não lhe doesse,
ao sentir que o mundo lhe fugia mesmo debaixo dos pés. E essa maldição
que lhe parecia sempre tão próxima, diante de si, era uma fulguração
vislumbrada na escrita. Obsessiva. Gabriel dava-se conta do imenso
terror que é escrever sobre o mais obscuro dos desígnios, o mundo. Estar
perto de tocar o sonho ou de pensar tocá-lo e não o atingir
endoidecia-o.

Parecia regressar com um aspecto mais sereno. Clara adivinhava-lhe a
angústia. Uma parte dele queria libertar-se, esse lado selvagem e
irredutível, que assomava inesperadamente por detrás do seu rosto. Via
bem que ele vivia asfixiado.

Silenciosa, trazia-lhe algo para que ele comesse e também ele o fazia em
silêncio. E depois agradecia-lhe, nesse único gesto que parecia ainda
restar-lhe.

Colava-se àquela expressão de gratidão simples a ameaça de um olhar
arruinado. A muralha que se interpunha entre ambos. De nada adiantava
explicar-lhe o vazio e a nudez, o seu fechamento. Uma coisa viva a
petrificar-se. Ela própria haveria de lá chegar e compreender, pobre
Clara, mas queria protegê-la desse conhecimento.

O que estava a acontecer-lhe? Perdido, Gabriel interrogava-se. Queria
compreender o que sentia, mas sobrevinha-lhe a impotência. Sentia-se
abandonado. Seria a escrita uma forma de desafio dos deuses? Ecoava-lhe
no cérebro o riso, momentaneamente entregue ao cão da loucura, que o
despedaçava e se alimentava da sua alegria, deixando-o exangue.

Tinha momentos de desespero e, depois, voltava à serenidade. O halo
branco do vazio descia sobre ele. Todos os fantasmas e terrores de que
fora acometido se afastavam. Inexplicavelmente, tal como tinham vindo,
sem que isso dependesse de si. E repetia interiormente Eclesiastes como
uma oração monocórdica, num esforço sobre-humano, sonhada e recomeçada
em cada dia. Como uma forma de reinventar a alegria.

Florimundo espiava-lhe o regresso, a fome. O silêncio. Quase sempre o
pai reparava no olhar da criança que o esperava. Sentia aquele halo de
vazio à sua volta. À medida que crescera, o rapaz apercebera-se da
crescente distância do pai em relação ao mundo, engolido por ele. E, à
medida que a melancolia se acentuava, compreendia claramente que só o
seu canto o acalmava. Nos olhos negros e sombrios do pai assomavam as
lágrimas, o cansaço.

Florimundo aprendia rapidamente as línguas que ele lhe ensinava,
aprendia a contar pelos elementos naturais: pedras, conchas, estrelas,
pássaros, flores.

Quando passeava com o pai, percorriam o longo caminho que se estendia ao
longo da falésia, coleccionando objectos fora do vulgar, pequenos
fósseis, pedras de formatos estranhos, observando os pássaros e
escondendo-se nos arbustos. Distinguia as flores, pois o pai ensinara-o
a reconhecê-las. E tudo tinha um nome, um desejo, uma fonte. Gabriel
procurava ensiná-lo a reconhecer cada coisa pelo seu nome. O nome de
cada flor, de cada animal. Achava Gabriel que só assim seria possível o
filho conhecer o mundo, tornar-se apto a comunicá-lo aos outros. Como
uma partilha íntima. Isso, sim, era conhecer. Conhecer pelos nomes, pelo
som, pela essência.

Nesse tempo, já o pai o sabia, ele transformava os sons que ouvia em
música. Desde o episódio da gaivota que Gabriel sabia existir no filho
esse poder estranho, de pequeno xamã. Com algum receio, contou a Clara
apenas uma parte da verdade, dizendo-lhe que a criança possuía uma
audição fora do vulgar:

- Ele ouve música, Clara...

- Como? - Perguntara-lhe ela, sem compreender o que ele queria dizer.

- Enquanto nós ouvimos palavras, sons, ele ouve música. É difícil
explicá-lo...

- Como é que o faz? - Boquiaberta, ela olhava-o, sem acreditar. Também
ouvia a criança trautear um canto que não reconhecia e que parecia fazer
mexer as coisas à sua volta, mas pensou que isso não passava de
imaginação sua. Agora aquilo?!

- Meu Deus? Que fazemos? Isso não é normal...as crianças não andam a
ouvir coisas. É uma alucinação, Gabriel...

Gabriel pensou que nunca se atreveria a contar a história daquela tarde
a Clara. Mais tarde ela descobri-lo-ia, certamente. Queria poupá-la.

Alarmada, propôs-lhe levá-lo a um psicólogo ou a um médico, mas Gabriel
opôs-se.

- Que disparate! Como vais fazer? Chegas lá e pagas-lhe, enquanto ele
explica à criança que o que ela ouve intuitivamente não passa de uma
alucinação?

- Mas não achas que pode ser um sintoma de esquizofrenia? Ou de
histeria? Ouvir música? É assustador...

- Sempre que tens de lidar com a diferença, assumes essa incapacidade.
Porquê? Passas o tempo todo a querer que ele seja igual às outras
crianças. Bolas, encara o facto! Tens um filho que não é igual. Porque
não aceitas o facto? Já reparaste que nada, nele, é comparável, nem a
capacidade de expressão, nem o modo como sente ou representa o mundo? É
um sonhador, talvez seja o termo mais adequado.

- Queres transformá-lo num pequeno génio, é isso? E se o tornas incapaz
de viver? Não tens o direito de o moldar às tuas imagens\ldots{}

- Não estou a moldá-lo, estou a deixá-lo ser o que ele é. Deixa-o ser
ele próprio. Ajuda-o a descobrir-se. Sem lhe alterares a vida ou
anulares o prazer que ele retira da sua forma de ser. E acredita que
aquilo o leva para regiões inimagináveis. Basta que o observes
atentamente, quando está sozinho, e perceberás que ele nunca está
triste. Há uma alegria nele que nos escapa. Olha para ele, raios! -
Disse-o num tom furibundo, mas baixo.

Era esse o tom de Gabriel, quando se irritava, fazendo o possível para
não perder o controlo.

- Clara, ele fala com os pássaros, sabe coisas que nem imaginas, um
saber que nenhum de nós possui. Florimundo pertence a um mundo antigo,
arcaico, que desconhecemos.

- Eu não tenciono fazer nada que altere a ordem das coisas. -
Acrescentou em voz baixa, mais para o acalmar do que por acreditar no
que dizia. - Mas fico aterrorizada. Sabes como tenho medo...vive
completamente isolado, provavelmente nem sabe brincar com nenhuma
criança da sua idade, mas isso não te preocupa. Devias ver a questão do
ponto de vista dele. Como é que vai viver com os outros quando chegar a
altura?

- Ele é diferente dos outros, não tenho dúvidas. Tu recusas-te a
aceitá-lo.

- Mas é diferente porquê? Porque queres que ele seja diferente? Mais
criativo que as outras crianças? Tens o direito de o isolar para manter
essa diferença?

As palavras de Clara iam-lhe direitas ao coração. Pela primeira vez, ela
ousava dizer-lhe o que pensava e a educação do filho cavava entre eles
um abismo de que jamais suspeitara. Toda a sua concepção de vida, o
ideal que julgara partilhar com Clara desfazia-se em pó.

- Evoluir talvez seja aceitar isso sem o questionar. Há tanta coisa que
nunca conseguirás compreender com essa tua clareza racional. Esse tem
sido o meu longo combate.

- Estás louco, Gabriel...Cada vez... - Não chegou a concluir a frase,
pois ele já lhe virara costas. A frase proferida como um soco
perturbara-a. Que espécie de combate era esse com o que não compreendia?

Para Clara, o mundo era transparente e reduzia-se às suas leis. Era o
mundo das coisas naturais, visíveis, com as suas regras bem
estabelecidas. Optara por afastar do seu entendimento as questões que
não compreendia. E irritava-a que se pudesse perder tempo a questionar
as obscuras razões. O sentido pragmático da vida empurrava-a para uma
compreensão sem zonas de sombra. A única frivolidade que se permitia, de
vez em quando, era interrogar-se sobre a existência de Deus. Esse
pragmatismo que, na verdade, roçando a falta de imaginação ou a recusa
do que se teme, empurrava-a para um combate cada vez maior e mais tenso
contra Gabriel, em nome de Florimundo.

Taciturno, Gabriel fechara-se na biblioteca. Mantinha firme a ideia de
ensinar música ao filho e educar-lhe o ouvido. Parecia-lhe ser essa a
única forma de racionalizar o obscuro sem lhe anular a potência do dom.
Antecipava para ele um destino de músico, tal a facilidade e intuição
com que ele era capaz de apreender e produzir a linguagem musical.

Contra o seu próprio desejo, o rapaz foi estudar música com uma
professora que vivia na povoação mais próxima. Essa seria a forma como
iria familiarizar-se com a aprendizagem mais técnica da música, a
leitura do solfejo e de um instrumento. Era demasiado jovem e habituado
à liberdade para perceber a utilidade de tal aprendizagem. Respondeu a
Gabriel que não precisava de aprender o que já sabia, provocando-lhe um
sorriso. O garoto detestava atravessar o umbral daquela casa que
cheirava a mofo, onde a poeira se estendia como um finíssimo tapete
sobre os móveis antigos da sala meio submersa na penumbra, demasiado
arrumada, muito diferente do mundo que conhecia.

Duas vezes por semana, Clara descia à pequena povoação, perto do mar. As
mulheres espreitavam-nos pelas janelas, nas ruas, procurando perceber
por que razão Clara, de corpo ágil e rosto bonito, nunca saía do seu
exílio. Levava-o à professora e entretinha-se nas proximidades, ora
sentada no café à sua espera, a ler um livro, ou então dava um passeio
pela povoação, entrando ao acaso numa ou noutra loja e detendo-se com o
seu olhar vago e habitado por uma tristeza indefinível que o tempo
cavava.

Para Florimundo era um absurdo aprender o que já sabia. Era como encher
um recipiente que já estava repleto.

O início da tempestade ameaçava o ar. O silêncio carregava-se de
electricidade; um pássaro negro deslocava-se velozmente no céu.

Florimundo desenhava, reproduzindo inteiramente a casa. Um mundo entre
dois lugares ou entre dois tempos. Gabriel olhou-o com aprovação. Os
olhos detiveram-se nos pequenos traços brancos, de texturas cheias, as
asas das gaivotas que pontilhavam o obsessivo vermelho. Por cima do
desenho observou a legenda escrita em traços indecisos, onde estava
escrito "A Minha Infância". O desenho emocionara Gabriel. Um universo
opressivo. Uma marca de irrealidade e de expressionismo distinguia-o de
um simples desenho infantil.

Era verdade que o rapaz tinha apenas seis anos, passados num isolamento
quase total, sem convívio, mas já escrevia coisas bem complexas.
Escrevia, contava e desenhava bem.

Durante muito tempo, Gabriel mergulhou num silêncio sem razão de ser.
Sentado no alpendre até bastante tarde. As palavras de Clara, opondo-se
a tudo o que ele defendia, haviam-no perturbado. Mas ele amava-a e
compreendia que ela tinha razão.

Uma neblina persistia em torno da casa. Gabriel compreendia agora como o
seu ideal implicava a perda de liberdade dos que o rodeavam. De Clara,
que continuava a amar, apesar de achar nela uma estranha oposição a tudo
o que ele sempre defendera para Florimundo, negando-lhe a possibilidade
de uma vida «normal» a que qualquer criança tem direito. Ele não sabia
sequer o que significava isso, já estava tão afastado do mundo que tinha
de fazer um esforço para se lembrar de como era ou para imaginar como
seria.

Viu o olhar triste de Clara numa apatia crescente, pois aquela casa
isolada no penhasco deixava-a exaurida pela tristeza. O rosto e a
expressão abatidos constituíam uma acusação silenciosa que ele não podia
ignorar. E a tensão dela avolumava-se, culpando-o pela solidão que a
deixava quase sem ar, durante o dia, a olhar pela janela, a procurar um
sinal de vida humana. Jamais se habituaria a esta clausura, que só se
atenuava pelo facto de Florimundo ser muito pequeno e se absorver
inteiramente na sua educação. Por muito que se ocupasse e preenchesse os
seus dias, precisava do bulício urbano, do barulho dos carros e das
pessoas, incomodando-a aquele silêncio que parecia tudo engolir, como se
a vida tivesse deixado de acontecer. Os dias permaneciam iguais e sem
qualquer alteração, a rotina ia-se tornando cada vez mais impossível de
suportar.

Nessa noite Gabriel não falou. Foi passear à praia. Algures entre a
noite e as dunas tomou uma decisão. O medo perlava-lhe a testa de suor.
Então, Gabriel viu-o. Imóvel, ele esperava-o. Repetia-se em todos os
lugares para onde ele dirigia o olhar, aterrorizado. Jamais a alucinação
se fizera tão insistente. Enigmática, resistente a qualquer compreensão.
Havia nessa figura uma mudez ameaçadora, um rosto ausente. De frente,
era tão opaco como se estivesse de costas. Mas o que doía e fazia
avançar as asas da loucura era essa vertigem de imaginar um olhar que
tanto podia ser o de um deus como o de um demónio. A mansidão da noite
não era suficiente para lhe aplacar o terror, o gelo que lhe contaminava
o sangue e o fazia estremecer. Era inútil esperar.

Regressou tarde, a lua ia alta e foi até à biblioteca. Desenhou e
redesenhou o homem sem rosto, em todas as suas variantes, em todas as
posições em que o vira, na esperança de, pelo desenho, matar o que
estava a dar com ele em doido, numa alucinação que não ousava contar a
ninguém. Tinha os seus momentos de lucidez, pois tinha uma clara noção
de que as alucinações e as imagens eram produzidas por ele. Sentia-se
encurralado entre imagens, sonhos, palavras. Era como se tivesse
penetrado um lugar interdito, do qual não sabia sair.

A sua alma não conhecia sossego. À medida que o representava, procurando
destruí-lo, erradicá-lo da sua imaginação e memória, parecia abrir-se,
então, uma nova clareira, convocando uma infinidade de visões que se
confundiam com as representações. Finalmente, entorpecido pelo cansaço e
pelo delírio, Gabriel escreveu numa folha de papel: \emph{``De espelho
em espelho''.}

Não conseguiu adormecer e sentia-se exausto e horrorizado. Ele
continuava ali sentado, à sua espera. Lá fora, o sol atravessava a névoa
da madrugada. O homem sem rosto jamais dormia. Virado na direcção do
mar, era como se procurasse algo, como se apontasse para alguma coisa
que ele não conseguia ver.

O romance de Gabriel não tinha fim e ele já não sabia quanto lhe faltava
para terminar essa montagem de fragmentos estilhaçados, após uma vida
que sacrificara em nome de uma ideia. Desde que saíra de Lisboa que
começara a escrevê-lo. Tecia o seu manuscrito com uma minúcia digna de
Penélope, numa complexa tapeçaria em que ele representava um herói
moderno e seduzido pelo canto da sereia, hesitante em retornar. Fosse
ela onde fosse. Na verdade, possuía em si essa fome de exílio, que o
arrastava para fora do mundo. Clara jamais compreenderia isso. Essa
contradição que nele existia, entre a vontade de um regresso a si
próprio, inscrita na linguagem pela escrita, e o desejo de um lugar
físico que lhe permitisse sustentar indefinidamente a sensação de
exílio. Substituíra a geografia do seu corpo e dos lugares por uma
outra, indiferente à passagem do tempo e às contingências humanas.

Não sabia que o seu sonho de descoberta haveria de revelar-se um
fracasso. E que a floresta, mais do que uma promessa, era uma ilusão
onde cada atalho ou vereda o conduzia a locais de onde jamais poderia
sair. E soube que não se encontrava mais perto das coisas e do mundo,
não se aproximara da essência, como havia pensado. Justamente, a
linguagem afastara-o irremediavelmente dos seres. Ao ponto de não se
reconhecer. A obsessiva procura dos limites da alma revelara-lhe um
abismo, onde não conseguiria jamais tocar o solo.

Florimundo sempre conhecera apenas aquele universo. Tinha nascido quase
imediatamente à mudança de Lisboa, pois a decisão de Gabriel era
encontrar um sítio onde a natureza fosse a escola do filho. A Clara, a
decisão parecera-lhe encantadora, ficando deslumbrada pelo aspecto
extraordinário da casa, descoberta durante um passeio pela costa. Na
altura em que a tinham visto, ela parecera-lhes semi-abandonada.

Gabriel ficara tão entusiasmado com a ideia que parara o carro junto ao
portão e olhara-a durante muito tempo. O céu estava cinzento e o mar
brilhava como prata. Um belo dia, sem dúvida. Enquanto Clara sonhava,
dentro do carro, a olhar para a vasta paisagem, ele foi tocar à
campainha. Ninguém veio atender. Não desistiu. Na casa mais próxima
perguntou quem era o dono da casa, acrescentando que desejava comprá-la.

O vizinho encolhera os ombros. Não gostava da casa nem nunca gostara. Os
últimos donos que nela tinham vivido já tinham morrido, sem deixar
filhos. Eram pessoas sombrias e que nunca tinham estabelecido contactos
com ninguém. Não gostavam de aproximações e o modo como tinham fechado a
casa revelava o seu modo de viver.

Os sobrinhos lutavam entre si, tanto mais que ela teria de ser
recuperada. Gabriel passou a ocupar-se do assunto com uma obstinação que
raiava a obsessão. Ao fim de dois meses de uma negociação complicada,
acabou por conseguir comprar a casa que estava em ruínas. Depois, teria
de ir para obras. Nessa altura, Clara ainda nem grávida estava. O
processo demorara tanto tempo que só lá conseguiram instalar-se na
altura do nascimento da criança.

A casa e o seu universo fechado formavam uma unidade. O manuscrito,
quando pensava nele, parecia-lhe todo organizado a partir desse
universo. Nunca desligado. Sentia agora que Clara era incapaz de
acompanhá-lo na sua procura. Não imaginava concluir o romance em nenhum
outro lugar que não ali. Como se aquele lugar fosse o único capaz de
conduzi-lo à reconciliação consigo próprio.

Gabriel pensava frequentemente nos arquitectos das catedrais, que
passavam o seu testemunho às gerações vindouras. Também via a
impossibilidade de o terminar em vida. Vislumbrou em Florimundo o
prolongamento da sua obra. E essa percepção, tida pela primeira vez,
causou-lhe repulsa. Tinha usado Clara e o filho para servir o seu
propósito, tornando-os escravos da sua solidão, para que nada
interferisse com a sua obra. Compreendeu a sua loucura, aparentemente
tão normal. O seu egoísmo sobre os outros, privando os que o rodeavam de
toda a necessidade de relações sociais, do seu caminho. Na verdade,
tinha criado à sua volta um ambiente cuja estabilidade lhe permitia
escrever. Um círculo fechado e sem interferências. A única presença
tolerada eram as esporádicas visitas da sua mãe e dos sogros.

Tudo lhe parecia agora uma aberração monstruosa. Queria encontrar as
palavras certas, o juízo que lhe permitisse avaliar o seu comportamento.
A vida pareceu-lhe impossível de ser resgatada. Nesse dia entrou a pique
nos seus próprios labirintos. Descobriu-se nos seres erráticos que
atravessavam as múltiplas entradas que havia criado. Seres que se
sonhavam e se fechavam nos anéis que os envolviam. O homem sem rosto
esperava-o ao canto da biblioteca.

Queria libertar-se desse mundo artificial, rasgado por inúmeras
passagens e reflexos. De uma profusão alucinante de imagens e rostos que
se confundiam entre a sua imaginação e a sua memória. Em que ele já não
reconhecia o seu próprio passado, transformado numa ficção. Um mundo em
que as leis de Hermes vigoravam selvaticamente, deixando os seus
habitantes à mercê das manhas do deus, vítimas do encantamento das
palavras e da música.

O homem sem rosto permanecia sentado ao canto da biblioteca, rodeado de
livros, supondo que a olhá-lo sem ver, pois não possuía olhos nem boca.
Nesse rosto, o vestígio da humanidade tinha-se dissipado. Gabriel deixou
de suportar a sua presença que o perseguia agora por todo o lado.
Sentava-se silencioso entre ele e Clara. Entre ele e Florimundo. E,
quando o rapaz sentia a sua presença, a música interior irrompia nele,
afastando o mal.

Tomou uma decisão sob o paciente olhar do homem sem rosto.


\section{A PARTIDA}

Nessa madrugada de setembro, um homem alto e moreno desceu à cave onde
guardava o pequeno barco de borracha. Transportou-o às costas, pela
escadaria abaixo, que rangia sob o peso excessivo. O homem sem rosto
esperava-o ao fundo da descida. Esperava-o para o conduzir nessa viagem.
Ele seria os olhos desse homem sem rosto, a sua boca. Estava um dia frio
e o mar estava calmo. Gabriel voltou a subir, foi ao quarto de
Florimundo e beijou-o na testa, com suavidade. Demorou-se à porta a
olhá-lo. Os seus olhos sombrios tinham uma expressão resignada. Não
voltaria atrás na sua decisão. De uma vez por todas libertar-se-ia do
homem que passara a acompanhá-lo sempre, a cada minuto e instante da sua
vida.

O rapaz entreabriu os olhos. Viu o pai a olhá-lo, à porta. Sorriu e
voltou a adormecer tranquilamente.

Então Gabriel, esse anjo maldito, saiu porta fora, sem voltar a olhar
para trás, o lugar onde ficara tudo o que tinha amado. Não sentia
nenhuma emoção forte nem nenhuma espécie de arrependimento. Tinha uma
missão que era a de libertar-se dessa personagem que o enlouquecia. No
meio da sua confusão ocorria-lhe pensar que aquele homem não era senão
ele próprio, esse pesadelo em que acabara por se transformar: alguém
incapaz de ver, ouvir e sentir o mundo. Preferia ter pensado em si como
um animal, consolando-o a possibilidade de sentir, apesar de não
conseguir interpretar o que sentia.

Mas era ainda menos que um animal. Perdera o rosto. A capacidade de
olhar, de escutar, de falar e de ver, para além de si próprio. Imergia
na escuridão como Jonas fora engolido pela baleia, à procura da redenção
e do sonho. Nem a memória o salvava.

Meteu-se no pequeno barco de borracha e fez-se ao largo. O vento
arrepiava a água e a escuridão da madrugada dissipava-se lentamente,
enquanto o sol, que se reflectia em tons de cobre, começava a romper. O
pequeno barco avançava na direcção dos penhascos, escuros. A caminho das
altas falésias, onde as ondas rebentavam e uma espuma branca se
projectava no ar, a toda a volta. Ele pensou, por momentos, em como lhe
apetecia mergulhar naquele abismo. Nadar e renascer.

Porém, sabia que não tinha tempo a perder. Tinha uma única ideia na sua
cabeça, que o arrastava, sem que percebesse por que razão ela o devolvia
à sua liberdade. Destruir a imagem, obrigá-la a recuar e a desaparecer
diante de si. Depois, poderia dormir descansado. Talvez aí pudesse
mergulhar no mar e sentir o corpo limpo, o espírito vazio e renovado. A
ideia, a princípio, agradável e repousante, começou a assustá-lo, mas
depressa sobreveio uma estranha calma. Sem desejo e sem nenhuma espécie
de sentimento. Um puro vazio. Sem dor, nostalgia ou medo. Era esse o
único estado que lhe permitiria encontrar a paz, agora que já não
conseguia deter-se sobre nada sem que a insidiosa imagem da ausência lhe
aparecesse. A mão gelada de Saturno, a mais escura das noites.

A aceitação de que teria de suprimir todas as imagens e representações
que o destruíam era tão serena e natural que lhe parecia não vir de lado
nenhum e era como se o seu corpo se reconciliasse ou que sempre soubesse
que deveria ter sido sempre assim.

Então, Gabriel cruzou os braços sobre o peito e deixou-se conduzir pela
corrente do mar. Não sabe se sonha ou se vive noutro qualquer lugar. Há
uma ilha para o silêncio?

\section{O LIVRO}

Florimundo deu por si próprio ao sentir o frio arrepiar-lhe a pele.
Estivera embrenhado nos livros durante toda a tarde. Vestiu um casaco de
lã, começava a ter frio.

Ouve os passos da mãe, ecoando no soalho de madeira.

Adormece de cansaço e é assaltado pelo sonho. Adormece por pouco tempo e
acorda. Relembra o sonho. O velho homem que se afasta na planície
gelada, caminhando. Incansavelmente. Os pés avançando, encharcados de
neve e humidade gelada. Um fito obscuro atrai-o, empurrando-lhe as
pernas, o corpo e as mãos para a frente. À sua volta, o único hálito é o
da brancura.

Florimundo jamais lhe viu o rosto e tem uma suspeita vaga de que nem
sequer é humano. Mas também não sabe o que é. Nunca compreendeu porque
anda sempre na mesma direcção, sem hesitação. Porque caminha,
simplesmente. Num local desolado, onde a vida se tornou improvável.
Dirigindo-se para a montanha gelada, ele não leva nada consigo. Apenas o
corpo. Cansado. Obstinado, persegue o sonho, repetindo-se, dia após dia.

Um dia ele verá o rosto. Ele talvez tenha um olhar terrível, como o dos
profetas alucinados. Mas também pode ter a candura de um velho, indo a
caminho da montanha, como nas lendas japonesas. Imagina algumas
hipóteses, todas elas igualmente possíveis. Talvez queira apenas chegar
ao monte, para aí adormecer e esperar pela manhã. Uma viagem para a
encontrar, entre os trevos da montanha. A probabilidade diverte-o. O
sonho persegue-o de forma constante, arrasta-o.

Um ruído seco e inesperado interrompe-lhe o devaneio, fazendo-o
estremecer. Volta de um mundo inexplicável e que o projecta para fora de
si. Onde as coisas são voláteis. Leves folhas, corpos embriagados e
dançantes, volteando na brisa da manhã, e que varrem a calçada, deixando
um rasto de nostalgia. Está-se no início do Inverno. As folhas dos
plátanos resistem ao vento, balançam perigosamente nos ramos. Um brilho
metálico prenuncia a escuridão.

Por detrás de Florimundo há estantes cobertas de livros, um cheiro a
mofo, que Clara procura disfarçar constantemente. Um quarto imerso em
livros, como lhe fazia notar Clara. Nas paredes do quarto havia alguns
desenhos seus. Num deles, que se intitulava ``A Minha Infância'', estava
representada a casa, rodeada de aves. Os outros eram também desenhos de
infância, mas não tão marcantes. Excepto, talvez, o do velho que
caminhava na neve.

Procurara libertar-se do sonho, mas jamais o conseguira. Ao lado do
desenho da casa estava pendurado e emoldurado esse desenho. Um era fruto
do seu passado, o outro da sua imaginação. Como se poderiam
interseccionar dois mundos tão distintos? Ainda que a memória fosse uma
reconstituição, ela jamais poderia confundir-se com um sonho.

Desde a morte de Gabriel que Clara sentia os livros como uma ameaça. O
rapaz impedira-a de se desfazer de tudo aquilo que o pai tinha guardado
durante toda a sua vida. Para proteger a memória da mãe e os livros do
pai, trouxera todos os que conseguira para o seu quarto. Nesse dia,
ficara tacitamente acordado que aquele território era interdito a Clara.
Durante muitos anos guardados em caixotes, no sótão da nova casa. As
folhas haviam amarelecido ao longo dos anos. O cheiro da humidade
marítima persistia ainda nessas páginas, ao fim de tantos anos, como se
elas tivessem querido guardar em si a sua vida secreta.

Florimundo abria-o e folheava-o, mas era-lhe difícil compreender o
universo do pai, uma escrita carregada de símbolos. Tudo em Gabriel era
obscuro, arcaico, impenetrável.

Uma névoa desce, após um dia inteiro de leitura atenta e minuciosa.
Começa a sentir-se cansado, sente os olhos a arder. Estuda para um
exame, respira fundo, esfrega os olhos e volta à leitura. Aponta a lápis
alguns fragmentos retirados de uma obra de estética musical. Anota as
margens. Pára. Lento, o espírito divaga no fio das frases, procura
compreender a razão por que se assinala com tanta frequência, a tensão
entre a palavra e a música.

Vêm-lhe à memória algumas ideias que o haviam marcado no discurso de um
dos seus professores:

- Pode-se dizer que a música procura libertar-se das palavras e, muitas
vezes, sobretudo na música instrumental, da materialidade da voz...
procurando uma formalidade absoluta.

Deu alguns exemplos, sobretudo da música contemporânea, indo buscar ao
minimalismo a justificação.

O rapaz procura compreender o alcance daquelas palavras. Pensa no que
significa essa materialidade da voz, de que falara o professor. Tem
dificuldade em acreditar na ``pureza'' da música, de que o professor
fala, desse formalismo, que lhe parece vazio. A música também tem essa
contaminação. Pelo menos pela subjectividade do compositor e do
universo, da vivência e das suas emoções. Tudo isso, na verdade, lhe
parece tão desinteressante, que desconfia dos conceitos e das suas
aplicações. Parece um pouco fútil para quem, como ele, só interessa
compô-la e criá-la. Pensa que as regras são importantes para dar as
bases, mas depois só servem para atrapalhar.

- Mas que libertação, que pureza é essa? Não é isso um conceito vazio?
-- Respondeu timidamente.

Estavam sentados, à volta da mesa do bar, com o professor. Este tinha o
cabelo grisalho, os olhos claros. Animado e conversador, ele gostava de
discutir com os alunos, valorizava-lhes o raciocínio vivaz, a erudição.
Gostava sobretudo de os espicaçar e provocar, com as suas tiradas, que,
por vezes, eram um pouco conservadoras.

Nessa tarde, a conversa rodava em torno da origem da música, na
continuação de um tema da aula que ficara em suspenso. Os mitos sobre a
origem da música, dizia ele aos alunos, eram todos, ou quase todos,
oriundos da mitologia grega e tinham nascido com ela. Não esquecendo,
como ele lhes chamara a atenção, que os poemas gregos eram musicados,
como letras de canções. E havia os mitos, como os de Orfeu e Eurídice,
do deus Pã, de Apolo e de Mársias, a sua competição que levou ao
sacrifício de Mársias. Nunca, dizia-lhes o professor, a música aparecera
desligada da mitologia e da poesia gregas.

- Mas não sejamos redutores. Se pensamos em Apolo e Mársias, por
exemplo, não devemos esquecer-nos de outras origens\ldots{} - aqui
hesitava, porque sabia entrar por caminhos menos académicos e olhados de
soslaio - como o xamanismo e as religiões ligadas à música. A música
sagrada e que permite a viagem até aos espíritos, por exemplo, uma
equivalência para a libertação das forças operada por Mársias com a sua
flauta. São muitos os mitos da possessão, por esse mundo fora...

- mmmm\ldots{}Professor\ldots{}- objectava um dos rapazes -- quer mesmo
falar de xamanismo como uma das origens para a música? É que aqui
entramos por caminhos pouco claros e nada objectivos\ldots{}

- Sim, porque não? Porque só considerar a tradição ocidental, ignorando
esse lado mágico? Uma das principais razões para a falência dos nossos
modelos racionais está precisamente no facto de ignorarmos essa origem,
quando os deuses estavam em todo o lado. Como o vento ou a luz. Não se
esqueçam do que diz Platão no \emph{Íon }sobre essa inspiração divina na
música.

- Claro! -- Respondeu um outro -- Não são poucos os compositores
contemporâneos que aí bebem. Não estamos só a falar dos antigos e dos
clássicos. Estou a lembrar-me, por exemplo, do Keith Jarrett\ldots{} ou
da influência da música sufi. Das experiências de Gurdjieff e de
Hartmann. Uma componente mística da música muito marcante em alguma
música contemporânea, por exemplo, que aparece nalgumas correntes e
escolas.

- Claro! Há um elemento obscuro na música. -- Retorquiu um outro rapaz -
Mas a música sufi, por exemplo, afasta-se desse xamanismo, creio que vem
de um outro filão, mais visceral e arcaico, mais ligado aos espíritos
animais, por essa razão ela é ainda tão importante nas comunidades
indígenas, onde o homem possui uma relação íntima com os espíritos...

Fez-se silêncio e alguém se despediu. A conversa foi interrompida.

- Voltando ao início e retomando a origem dos mitos gregos sobre a
música, essa ambivalência entre Mársias e Apolo\ldots{} Este simboliza a
luz e a lira é o instrumento da ordem, repetindo na terra a ordem
cósmica, mas é também o deus que mata. - Continuou ele.

- Ah, mata? Mas ainda não consegui compreender a razão de ser da morte
de Mársias...a razão simbólica, quero dizer. - Respondeu um aluno não
muito escorreito, mas esforçado e com o cenho franzido, revelando o
esforço para acompanhar a discussão.

Florimundo corou até à raiz dos cabelos.

- A música resulta dessa imposição de Apolo sobre Mársias e do domínio
da ordem sobre a soberba de Mársias, que desafiava os poderes da
Natureza contra a racionalidade e a ordem... - respondeu Florimundo, com
a coragem que conseguira arranjar para vencer a sua timidez natural.

O olhar do professor pousou sobre ele. Gostava do seu ar calmo e
inteligente, silencioso. Uma atenção rara que denunciava uma
concentração do ouvido em relação à música. Era um rapaz educado, com um
ar um pouco triste. Já tinha ouvido e lido algumas composições suas,
embora não fosse da especialidade. No entanto, o rapaz revelava-se um
óptimo aluno de teoria da música e de estética.

O outro segredou-lhe que as aulas daquele professor mais pareciam aulas
de filosofia do que de música. Ele respondeu-lhe que, não se
compreendendo os aspectos filosóficos da questão, jamais seria possível
compreender a natureza da música. Incapaz de se compreender que ela era
sempre mais do que o puro tecnicismo ou virtuosismo. Uma expressão, uma
linguagem de uma outra ordem e que ultrapassava todas as técnicas e
instrumentos, universal.

Imaginava se iria sentir sempre essa contradição de que o professor lhe
falava. Por um lado, ser impulsionado pela música que o habitava e que
não dominava, espécie de caudal subterrâneo, por outro, o ter de lhe
impor a ordem e a racionalidade, domesticando a sua própria natureza. E
seria ainda a música uma inscrição da ordem cósmica no corpo humano? A
ideia não lhe pareceu destituída de sentido. Mas não saberia formulá-la
objectivamente e, por isso, calou-se. Tudo isso constituíam questões que
ficariam sempre sem resposta, pois não havia como chegar ao ponto
originário.

Que música pura, ideal, seria essa que era capaz de sobreviver sem a
nota de humanidade?

Procurava, na leitura dos filósofos e dos teóricos que estudava, uma
resposta para as suas inquietações. Achava que a única ponte para a
compreensão passava por essa procura ávida. Sabia que o seu talento
exigia uma compreensão das possibilidades que se abriam a partir de si.
Uma espécie de porta por abrir que poderia levá-lo a algum lado ou não.
Mas era uma passagem necessária.

A literatura suavizava-lhe a existência, pela força das imagens e a
claridade da linguagem que descobria nos grandes romancistas. O pai
tinha-lhe deixado bem entranhada na carne essa dor de procurar perceber
o mundo através dos livros. Nada lhe dava prazer como um longo romance,
que demorava muito tempo a ler, esse tempo lento que o fazia mergulhar
num universo tão diferente do seu. Onde se sentia amigo próximo de
algumas personagens, outras a quem odiava, consoante o livro, a
história, mas era preciso imergir para viver o prazer da leitura. Uma
outra forma de realidade. Um repouso que se estendia muitas vezes entre
as páginas. Às vezes ficava profundamente perturbado, inquieto. Mas se a
vida se podia aprender e a natureza humana poderia ser compreendida,
então os grandes romances eram um caminho largo.

Porém, a filosofia afasta-o da vida, tornando-a dolorosa. Com a
filosofia não podia distrair-se, ainda que houvesse pensadores
extraordinários de que ele gostava particularmente. Pouco ou nada, na
vida dos homens, parece relacionar-se com as grandes aspirações ideais.
A metafísica ou os ideais propostos, como belas asas de anjo, cujo canto
deveria seduzir os homens, à partida, e arrancá-los à pequenez das suas
vidas, além de inacessível, parecia-lhe impraticável. Os caminhos dessa
floresta pareciam-lhe árduos, quase impraticáveis. Não que fosse difícil
acreditar nessas ideias, mas impossível era levá-las à prática. Só um
ingénuo acreditava nessa na possibilidade e entediavam-no as grandes
propostas filosóficas e ideológicas. Parecia-lhe ingénuo, patético
acreditar, como o defendia Platão, num mundo de ideias puras. Platão
esquecia-se do humano, desse lado sujo, material, da vida e da
realidade. Para ele não havia sujidade nem fealdade, os homens não
comiam, não fornicavam. Para que serviam, então, essas «belas ideias»?

Contra o tempo humano e a convulsão de história e de catástrofes, que
pode o homem erguer, pensava ele, a não ser a arte. Mas uma arte que
integre em si a morte, a contingência, a fragilidade humana, não essa
arte despojada, limpa de emoção. Todavia, deslumbrava-o o lirismo da
escrita de Platão e do pensamento, tão palpitante nas suas personagens.
Ao ler \emph{O Banquete}, sentira-se particularmente tocado, pois julgou
encontrar nele algumas das concepções que o poderiam guiar ao longo da
vida, como o discurso de Diotima, que relera tantas vezes que já não
saberia dizer quantas. Para perceber essa dor de ser ``entre''. A dor e
o júbilo de não poder sair dessa clareira de desejo, entre o humano e o
divino. Por outro lado, uma teimosia impelia-o a acreditar nessas
ideias, menos por convicção do que por uma espécie de nostalgia e de
esperança.

Todos os dias via à sua volta o que não desejava, a natureza humana na
sua violência, que lhe causava repulsa. Esse era o mundo que o rodeava,
o das guerras, do massacre e do sacrifício das vítimas, em nome de leis
que permaneciam ocultas, mas que beneficiavam os senhores. Não havia
nada de libertador nessa violência nem reconhecia qualquer ímpeto
nietzschiano nos atentados, nos ataques diários, na impunidade e na
prepotência do poder que os mais fortes impunham aos mais fracos. Se
essa era uma das mais atraentes ideias filosóficas que encontrara, na
prática ela não passava de um desvio da ética e da presença do mal,
gratuito e sem lei.

Por isso, muitas vezes, receava dissolver-se nesse canibalismo da
filosofia em relação à vida e cair num mundo de ilusões, incapaz de
perceber a crueza das emoções humanas e dos sentimentos, a vida no seu
estado bruto, a barbárie enquanto segunda pele. Só a arte, acreditava,
lhe permitia a superação dessa banalidade da violência, não que isso se
chamasse amor ou outro nome assim, mas haveria, ainda, em nome da arte a
possibilidade de uma redenção? Sempre tinha sido assim, na história do
homem.

E conseguiria alguma vez atingir um estado de fusão entre uma
consciência racional e uma intuição imediata? Então a música seria um
médium perfeito, estabelecendo todas as passagens numa ansiada fusão,
que ligasse todos os lados do humano num Eu único, à maneira romântica.
Nessa procura inquieta, talvez Nietzsche fosse, entre todos os
filósofos, o que mais lhe agradava. Sentia-se mais próximo dele, como de
Schopenhauer. De Nietzsche aproximara-se mais, atraído pela sua
concepção da arte e da música, pela aceitação da manifestação do
dionisíaco e do trágico da existência. Estavam mais próximos da vida do
que qualquer outro. Sim, talvez parecesse ingénuo aos olhos dos outros,
mas acreditava que apenas pela criação e pela arte fosse possível essa
ponte desejada entre o animal e o divino.

O último andamento da sinfonia de Arvo Pärt desagua no silêncio. Um fim
que nunca é um fim, mas um começo de qualquer outra coisa. É essa a
imagem que perdura, ainda. Mergulhara nesse estado quase febril de
ensimesmamento, como se um nevoeiro fino se interpusesse entre si e o
mundo. Sente-se vagamente perdido, o frio a arrepiar-lhe o pescoço.
Faz-se noite, que mansamente cobre os livros de sombra e o protege,
fazendo deslizar sombras pelas paredes do quarto. Amava esse interregno
entre o final do dia e a noite. O momento de ternura e de reencontro, de
uma descida lenta em si mesmo.

Uma imagem apresenta-se diante dele, atravessando-lhe o olhar como uma
lâmina. Sabe que deve segui-la, que está dentro do seu sonho. É o tempo,
esse fulgor que emerge lentamente. O tempo da música que o engole e se
apodera de si, chamando-o. O tempo de um outro rumor, onde o mundo se
apaga. Um homem observa-o, na vidraça. Surpreende-o o olhar concentrado,
as mãos encostadas ao rosto, onde se desvanece o contorno da face. Ali
está ele, envolto pela desordem dos livros, entregue ao devaneio, à voz
que vem de lado nenhum.

A mãe surge à porta do quarto. Encosta-se à parede. Olha-o, taciturna.
Assim, projecta-se a sua sombra contra a parede. Tem as pálpebras
escuras, maceradas pela falta de sono e pelo cansaço.

Por momentos, revê Clara a imagem de Gabriel, sentado a escrever, diante
da janela, ainda que Florimundo não se parecesse fisicamente com o pai.
Era parecido com ela, nos seus traços mais claros e suaves, onde havia
uma certa falta de vivacidade, precisamente como ela. Uma beleza que
esmaecia rapidamente com a idade, apesar da regularidade e da harmonia
dos traços, da perfeição do rosto.

Em Gabriel, tudo era paixão, os olhos escuros e melancólicos, a tez
tisnada, um lume que atravessava o olhar. A loucura que ela fora incapaz
de deter. Nada nele deixava lugar à indiferença. No filho, havia uma
suavidade que a acalmava.

- Entra. - Pediu-lhe o rapaz. - Voltou-se, tendo-se apercebido da
presença da mãe, menos pelo barulho do que pelo reflexo na janela.

- Não te isoles tanto. Faz-me medo. - Clara olhava para o filho, mas a
imagem de Gabriel demorava-se no filho, algo do seu cansaço, do seu
abatimento, não sabia dizer o quê.

- O que fazes?

- Acho que vou sentar-me e fingir que vejo televisão. É só desgraças,
atentados em todo o lado. Mortos à bomba, atropelados, esfaqueados. Na
Síria bombardearam novamente, já não sei o que há para bombardear.

Parou de repente e achou que a conversa estava a ficar mórbida, não
tinha o direito de lhe levar a sua angústia.

- Não vais descansar? Trabalhaste durante todo o dia.

O rapaz viu-lhe as olheiras fundas, a palidez do rosto, que ela
procurava disfarçar durante o dia, pintando discretamente o rosto.
Ficava com um ar mais saudável. Porém, ao cair da noite, a luz e o
desaparecimento da pintura artificial, acrescido de cansaço,
roubavam-lhe a frescura e ficava com o habitual tom anémico e pálido.
Florimundo preocupava-se, mas de nada lhe servia admoestar a mãe. Ela
comia mal, tinha falta de apetite e trabalhava excessivamente.

- Preciso de entregar este trabalho amanhã. -- Disse o rapaz - Mas
trabalho lentamente, não avanço nada. Perco-me a ouvir música, divago...
As ideias atropelam-se-me umas nas outras... julgo que só agora e que
vou começar a redigir... Também não quero sair daqui para não me
distrair, vou agarrar-me ao computador. Não é que não saiba o que quero
dizer, compreendes, mas as ideias fogem, desafiam-me. Espero que elas se
cansem e sosseguem.

Ela sorriu. A alegria tinha desaparecido havia muito tempo. Aquela de
que ele se lembrava e que lhe trazia sonhos antigos. O tempo em que ele
deitava a cabeça no seu colo.

- Queres que te traga alguma coisa? - Clara procurava quebrar-lhe o
isolamento. - Deves ter fome. E está frio, não achas?

Abraçou-o por detrás, como gostava de o fazer.

-- Precisas de te agasalhar ou de aquecer o quarto.

Quando estavam assim, agarrados, pareciam irmãos. Fisicamente e na
tristeza, na mudez antiga. Ele sentiu-a tremer, contra o corpo dele. Um
corpo ossudo e desprotegido.

- Está frio. Vai deitar-te. Eu fico bem, não te preocupes.

O rapaz sabia a que ela se referia. O fantasma do pai era um grito,
ecoando por todo o lado, deflagrando na memória de ambos. O fantasma que
havia conduzido Gabriel a um mundo onde todas as relações lhe apareciam
distorcidas, os objectos desfeitos e amalgamados, em que os efeitos da
linguagem o tinham conduzido à mais absurda descrença. O eterno
espectro.

Clara teria de estar sempre atenta para que não se repetisse o peso
dessa mão que lhe roubara Gabriel. Para que ele soubesse sempre que o
lugar no mundo era onde os pés pisavam e não num lado qualquer. Um lugar
onde houvesse raízes, em lugar de vagos sonhos ou ideias desgarradas. O
mundo tinha piorado muito.

Florimundo lembrava-se de quando a mãe aparecia nas reuniões de pais e
ele se sentia aliviado se ela não estivesse presente. Pensa hoje que
talvez quisesse protegê-la dos olhares indiscretos. Conhecia as mães dos
colegas e dos amigos. Pintadas, de cabelos cuidados, bem vestidas, tudo
o que ela não era. Esforçavam-se para parecerem mais jovens, falavam de
uma maneira afectada, entre si, e procuravam impressionar-se mutuamente.

Estivesse onde estivesse, Clara fazia o possível para que não se
percebesse que existia. Demasiado reservada para que as pessoas pudessem
ter uma imagem favorável dela. Quando eram vistos juntos, nas ocasiões
de festas da escola, homenagens e espectáculos, o rapaz sentia-se
envergonhado e simultaneamente culpado por se sentir assim. Lembra-se de
ter desejado, em criança, que Clara fosse mais exuberante e alegre.

Costumava pensar na mãe como uma ave silenciosa. Não abria a boca e
escutava com atenção tudo o que se passava à sua volta. Achava inútil
perder tempo a olhar para as lojas, não tinha dinheiro para comprar
nada. Mantinha um rigor espartano, tanto no que se referia a si própria
como ao filho. Quando, em miúdo, ele pensara em mudar de instrumento ela
apenas lhe disse, o desamparo instalado no rosto:

- Florimundo, não posso comprá-lo. Não tenho dinheiro.

Não lhe dissera que não. Não fizera qualquer objecção. Apenas o
confrontara com um facto incontornável. A sua pobreza não o permitia,
acrescentara-o.

Com o tempo, a criança percebeu porque era diferente dos outras. Porque
teria de ser sempre diferente.

Durante muito tempo odiara ser pobre. A mãe revelava sempre a sua
indecisão, em relação a todas as decisões que ele tomava. Perguntava-lhe
imediatamente quanto é que tal iria custar.

A mãe olhava-o. Parecia esperar por ele.

A semana depressa acabara. De manhã cedo, a mãe entrara-lhe pelo quarto
adentro, já vestida.

- Vou até ao jardim? Queres vir? Apetece-me apanhar sol, andar um
bocadinho...

Florimundo decidiu acompanhá-la, apesar de lhe apetecer dormir mais.
Despachou-se, tomou o pequeno-almoço, enfiou um casaco e saiu de braço
dado com ela. Eram tão parecidos que ela poderia ser a sua irmã mais
velha.

Queria andar, nessa manhã bonita e de céu claro, sem nuvens. Já havia
muitas folhas de árvores espalhadas no passeio e pelo chão.
Apetecia-lhe, como quando era miúdo, pisar as folhas e ouvi-las estalar.
Era domingo e metade da cidade dormia, provavelmente. Na rua, um ou
outro café abria as suas portas aos clientes. Agora havia enxames de
turistas por todo o lado e apareciam logo de manhã, ruidosos,
barulhentos. Uma coisa recente que vinha descaracterizar a cidade,
aumentar os preços, desalojar os habitantes mais velhos. Ao fundo da rua
da Misericórdia, avistava-se uma nesga do rio. Mãe e filho desceram até
ao cais do Sodré. Um vento frio surpreendeu-os. Florimundo gostava de
sair assim com a mãe, era quase um ritual.

Eram capazes de deambular ao longo da Av. 24 de Julho até deixarem de
sentir as pernas. A mãe embrenhada nos seus pensamentos. O filho
escutava tudo, nele tudo era um imenso ouvido que se transformava em
ideias musicais. Quando voltaram da longa caminhada, sentaram-se na
esplanada dum jardim. A mãe estava serena, bem-disposta. Ela sempre
desejara voltar a Lisboa, era aqui que tudo vibrava e parecia sentir-se
à-vontade com tudo. Semicerrava os olhos e encostava-se na cadeira,
deixando que o calor lhe afagasse o rosto, as faces.

O olhar de Florimundo segue o curso do voo dos pássaros, hesitantes na
escolha da árvore, afastando-se das pessoas. As árvores, apesar de tudo,
estão cheias de ninhos. Apesar da poluição e do barulho, que não
afastaram a sua alegria e o seu bulício coexiste com o ruído e a
confusão. Aos domingos há menos trânsito e menos confusão e tudo parece
mais sereno.

Todos os dias, mais ou menos àquela hora, chegava o jardineiro, já
bastante velho. Tem as mãos sardentas e deformadas, com a pele tão
grossa que parece insensível ao toque das pétalas, mas há nos seus
gestos uma delicadeza devota. O homem observa meticulosamente o estado
das flores, rega-as e corta as ervas daninhas que vão aparecendo.
Pragueja, de cada vez que encontra um pequeno verme, um caracol ou um
parasita, o que provoca o riso de quem se encontra à volta. Esmaga os
caracóis e as lesmas com a pá, o que provoca um som de gordura a
estatelar-se no chão. Ficam as moscas a esvoaçar, à volta dos vermes de
corpo rasgado e esventrado. Feridas de gordura, ao sol, a aquecer, e
naturalmente os seus líquidos servirão para alimentar outras vidas, mais
invisíveis.

Ao longe, as crianças invadem o parque. Trazem os pais pela mão, é o que
mais parece, querem correr, mas são travadas por eles. Ainda é manhã,
demasiado cedo para quem passou a semana a levantar-se cedo e queria
dormir mais. Para quem provavelmente acordou durante a noite ou se
deitou tarde. Descidas velozes dos escorregas, um salto no ar do
baloiço. As mães sentadas ou a ajudá-los a subir para os escorregas,
para cima dos baloiços. Largam aos gritos de nervoso e de contentamento.
Florimundo reviu-se nesse gesto de subir até ao cimo do escorrega.

- Agora, vá! - Gabriel esticava os braços, para o apanhar,
encorajando-o.

- Tenho medo. É muito alto! - Queixava-se a criança, que não conseguia
sequer pensar no salto.

O pai ria e disse:

- Estou aqui. Não tenhas medo.

Caiu e magoou-se no queixo. Um golpe pequeno, esfolou-se na dobra do
escorrega.

Às vezes surpreendia os pais, correndo que nem um louco, de braços
abertos, fechava os olhos e sonhava que subia nos ares.

Junto de Florimundo, sentada no jardim, a mãe mantinha os olhos
semicerrados. Nesses momentos, era quase feliz. Solitária. A mãe era
ainda jovem, apercebia-se Florimundo. De algum modo, doía-lhe que não
tivesse nenhum homem. Devia ser necessária coragem a qualquer homem,
para se aproximar de uma mulher com um ar tão triste. Clara era meiga,
falava baixo e era tão silenciosa que jamais incomodaria alguém.
Recordava-se dela, ao lado de Gabriel. Ele muito alto, ela um bocadinho
mais baixa, muito alta, bonita, nos seus vestidos de Verão, os lábios
pintados de vermelho-escuro, as unhas da mesma cor. O vermelho
sobressaía no rosto pálido e leitoso, pontilhado por pequenas sardas, de
um tom acastanhado. O cabelo loiro e encaracolado, sempre solto.

O sítio favorito de Florimundo, no jardim, era junto da velha faia, com
mais de uma centena de anos, grossa e sólida. Insólita, naquele local da
cidade. Sob a árvore, quer fosse Inverno ou Verão, um grupo de homens
havia envelhecido à sua sombra, jogando dominó, conversando ou
simplesmente deixando correr os dias. O dominó era sempre igual, os
vencedores os mesmos, os diálogos escassos, invariáveis. Aquele estranho
círculo iria certamente romper-se, um dia, mas, enquanto permanecesse,
haveria de assemelhar-se às rodas que as crianças fazem, círculos
mágicos, onde as vivências são preenchidas de memórias comuns,
acumuladas nos anos. Era engraçado ouvir as conversas, o repto rápido, a
malícia comum dos diálogos, entre homens que haviam partilhado as mesmas
vivências e as mesmas amizades. Talvez as mesmas mulheres, pensava ele,
já que muitas vezes a conversa maliciosa incidia sobre elas.

Era fácil amar os homens, quando eles apareciam assim, na sua
simplicidade. Quando apareciam ao olhar tão despojados e nus, tocados
pelo esplendor de um tempo que já não era o nosso.

Florimundo sabia que tinha trocado a vida por um ideal. Talvez a mãe
tivesse algo a ver com isso, mas ele não seria capaz de a culpar pelo
seu próprio isolamento, pela incapacidade de se relacionar com as
raparigas, pela timidez que o afastava dos outros. Fora habituado desde
criança a ter objectivos, deixando tudo o mais de lado. O dinheiro que
havia era para as coisas essenciais e sobretudo para a aprendizagem da
música.

Ela própria tinha de aproveitar o tempo, para fazer coisas que não
conseguia, durante o horário de trabalho, pequenas traduções, biscates.
Normalmente, a lida da casa ocupava-lhe os fins-de-semana por inteiro.

O rapaz entretinha-se a ler ou a estudar, a fazer os trabalhos de casa.
Poucos prazeres lhe eram concedidos. Habituara-se de tal forma a estar
sempre ocupado que lhe parecia estranho quando não tinha de fazer. Até
que ponto isso iria revelar-se como um absurdo, não o sabia. Deixara a
infância e o tempo escorrer para um qualquer lugar desconhecido. Um
quarto, nessa casa cheia de ausência, que era a sua vida. Na verdade, se
fosse feliz, pensava, não seria obrigado a procurar nada.

Quer fosse Inverno, Primavera ou Verão, fechava-se no seu quarto,
indiferente às pequenas mudanças do dia, às solicitações do ócio,
concentrado a dedilhar o piano, estudando composição, lendo e
trabalhando. Obstinadamente, lutando contra o cansaço, desafiando as
leis do seu próprio corpo, violando a sua resistência.

Possuía mãos longas, com alguma falta de mobilidade. Além disso, não se
sentia motivado para a interpretação. A interpretação revelava-se-lhe
como a melhor forma de se aperceber do que ia compondo. Uma coisa é
escrever, outra é ouvir o que se escreve. Tal como escrever poesia e
ouvi-la. Duas experiências que frequentemente se traem uma à outra.

Os professores disseram-lhe, a medo, pois conheciam-lhe e
adivinhavam-lhe o empenho, que nunca seria grande intérprete de piano.
Não havia nele a agilidade, a leveza e a destreza dos grandes
intérpretes. Seria melhor dedicar-se à composição, já que demonstrava
notáveis capacidades e uma originalidade surpreendente. Ele recebeu a
notícia sem pestanejar. Com o olhar conformado, diante dos seus juízos
doutorais. Sem mágoa. Sabia desde sempre, apesar da voz interior, que as
suas mãos não eram suficientemente ágeis e velozes para acompanhar a
música que havia em si. Tocar distraía-o da percepção da música.
Percebeu que havia um lado muito racional nesse acto de escuta.

A ideia não era nova e tinha-a desenvolvido ao longo dos anos. A
primeira vez que ouvira falar dos antigos pitagóricos ficara deslumbrado
com a possibilidade de reproduzir o segredo da harmonia cósmica na
música. Este poderia reflectir, tal como um caleidoscópio gigante, a
complexidade colorida e diversa do universo, em todos os seus modos de
ser. Mais tarde, estudara a evolução da música, deslumbrado pelo
estatuto mítico que ela atingia em diversas épocas históricas.

Aprendera sobre os seus estranhos poderes. Como naquele episódio em que
David cura a loucura de Saul, tocando harpa, ou o soar das trombetas e a
vozearia que derrubaram as muralhas de Jericó. Os efeitos da música,
sabia-o, podiam ser terríveis ou benéficos, de acordo com a utilização
que lhe era dada. Irrompendo em toda a sua força e plenitude, a música
podia desencadear os mais terríveis ou, pelo contrário, os mais
benéficos efeitos sobre o homem. Conhecia os efeitos misteriosos do
xamanismo ou da possessão da música, os quais, aparentemente próximos,
eram, no entanto, bem diferentes.

Na procura de auto-conhecimento, investigara acerca do seu próprio
poder, não sabendo como haveria de classificá-lo. Reconhecia-se entre o
xamane e o possuído. Lembrava-se de, em criança, possuir uma mediunidade
que acabara por racionalizar, limitando-a a uma simples capacidade
auditiva, puramente receptiva. Não queria utilizá-la porque lhe temia o
seu alcance. Mas queria compreendê-la, torná-la inteligível para si
próprio. Havia estudado com afinco os tratados medievais de música,
debruçando-se sobre os livros herméticos, com a secreta esperança de
encontrar neles as leis que que regulavam secretamente a música.
Obcecava-o a componente oculta da composição musical, operando sobre a
realidade, à semelhança da alquimia e do seu trabalho de transmutação.

Em miúdo, a evidência da música bastara-lhe, aceitando-a no seu
mistério. Tal como aceitava a evidência das flores ou da chuva, da fome
e dos gestos, dos caminhos e dos nomes das coisas. Nisso reconhecia a
inocência e o espanto das crianças, demasiado ocupadas a descobrir o
novo sem o questionarem. Mais tarde, o que lhe tinha sido evidente
tornou-se obscuro.

Durante muito tempo questionara-se como lhe chegavam as sensações, as
imagens e as ideias. Tal como os sonhos que o assaltavam pela madrugada,
mas jamais ouvira falar do que sentia desde sempre. Desde que se
lembrava. E dessa origem obscura, que permanecia para ele inacessível.
Mais tarde, lera algures que os anjos não falavam, mas que faziam do
canto a sua linguagem, um gesto de que se lembra de menino. Os anjos não
existem, repetia amiúde, para se convencer a si próprio de que não fazia
parte de nenhuma estirpe de seres inatingíveis. Porém, o episódio da
gaivota, na sua infância, atestava o mistério e obrigava-o a pensar-se
nessa charneira entre ser homem e anjo, entre cantar e ter linguagem.

Começara a chover. No início mansamente, molhando as pedras da calçada,
as ruas, os troncos e os bancos do jardim, as folhas das árvores. As
gotas de chuva assemelhavam-se a notas suaves, na sua sucessão
monocórdica. O cheiro da terra encharcada subia no ar. Com a idade
perdera a capacidade de distinguir a diversidade dos cheiros que a chuva
trazia consigo, consoante a estação do ano, a terra inundada, os
objectos inchados de água e de humidade, exalando odores diferentes.

Florimundo sentia aquele odor que despertava nele sentimentos intensos
que se ligavam à infância. O regresso às aulas, nos finais de setembro,
quando chegavam as primeiras chuvas e um vento macio na pele do rosto,
anunciava o frio. Quando as roupas de Verão já não bastavam para lhe
aquecer o corpo.

Obsessiva, a memória voltava-lhe ao passado. Queria travá-la, impor-lhe
disciplina, mas a ausência do presente, esse vazio actual levava-o de
volta ao tempo cheio. Lutava contra o peso e a memória operava essa
transformação do não-tempo em tempo puro. Sem que isso dependesse da sua
razão. As imagens reconstituindo-se sozinhas, forçando as leis da
associação, rasgando a sucessão. Ficava, então, preso nas imagens como
uma criança. Perdido nos escombros da infância.

Sabia de antemão que a casa continuaria abandonada ao fragor das ondas,
o único ruído consentido, mas não podia deixar de desejar esse tempo
primevo. Só as gaivotas permaneciam, nos seus longos voos demorados e
circulares. Deixavam as suas pegadas, escritas num vestígio, que se
perdia de cada vez que uma onda lambia a areia. E, à medida que o olhar
se aproximava dessa imagem da casa, revia-a cada vez mais longínqua.

Por vezes, era como se estivesse a ver um filme, podia compor toda a
matéria da sua experiência, metamorfoseá-la, fazer deslocar as
personagens, variar infinitamente os pormenores. Nada nessa imagem era
dado, pacífico, mas uma fonte permanente de signos que exigiam a sua
paciência e a sua decifração.

Lembrava-se da mão da mãe, agarrada à sua, vestida de negro, logo após a
morte do pai. Fora precisamente no final do Verão. A mãe vencida e
enraivecida, como um guerreiro prostrado. O desamparo da vida e do amor.
Havia no olhar da mãe uma dor alucinada e uma revolta que não a deixavam
chorar. Que a enlouquecia de raiva e impotência. Gabriel tinha apenas
deixado na mesa da cozinha uma simples folha de papel, a despedir-se
dela.

Quando ela se levantara, bem cedo, estranhara não o encontrar a seu
lado. Ouvira-o, durante a noite, na biblioteca, e deu pela luz acesa.
Mais uma vez pensou na tremenda fixação que ela já não conseguia
combater e afundara-se nos lençóis, fechando os olhos, entre lágrimas
silenciosas. O mundo desfazia-se à sua volta.

Gabriel não queria que ela trabalhasse. Dizia que não havia necessidade,
que tinha o filho para criar e, se fosse trabalhar, o menino teria de
ser abandonado num infantário ou numa ama qualquer das redondezas.
Quando o miúdo entrasse finalmente na escola, e Deus sabia o quanto ela
se esforçava por arrancá-lo à influência do clima opressivo que Gabriel
havia criado à sua volta, daria finalmente o passo decisivo e
libertar-se-ia da tristeza.

Tinha decidido, contra ele, uma vez que Florimundo iria fazer sete anos,
que não passaria desse ano. Descansava-a o facto de a criança, apesar de
tudo, saber ler e escrever. A cegueira de Gabriel havia chegado a um
ponto insuportável. Compreendia que o sustentava a ideia da criação de
um lugar onde pudesse educar o filho de acordo com a ideia do bom
selvagem, sem as más-influências que via na sociedade.

Discutia com ele o que lhe parecia estranho e assustava-a o facto de a
criança mal falar e não utilizar as palavras, apesar de as conhecer bem.
Porém, havia na sua vivência um mistério que o tornava quase
incomunicável. O miúdo não falava, não pedia, sentava-se e procurava
compreender o que passava à sua volta. E, sobretudo, ouvia tudo com
interesse. E mergulhava nesse alheamento próprio de um animal a quem lhe
bastava a experiência.

Sentia que ele possuía estranhos poderes. Cantava e aquietava tudo em
seu redor. Deslocava-se com a lentidão de um deus que conhecia tudo à
sua volta, parecendo compreender as coisas como se elas lhe falassem.
Permanecia manhãs inteiras, deitado ao sol, como se fizesse parte da
paisagem. Jamais conhecera uma criança assim e não se sentia preparada
para enfrentar os seus gestos lentos.

Não havia nele qualquer necessidade de procurar outras crianças, de
brincar. A solidão bastava-lhe e procurava-a como um danado, onde podia
esquivar-se à palavra.

Observava-o, meio entorpecida. E aos desenhos dele. De onde poderia ele
extrair os temas dos seus desenhos se não tinha contacto com nada nem
ninguém? E, todavia, as figuras repetiam-se, o velho que caminhava na
paisagem de neve, a imagem obsessiva da casa, das aves à volta da
biblioteca, imagens de duendes e de fadas.

- Donde vem isto, Florimundo? - Perguntava-lhe ela, na sua ânsia por
compreendê-lo. -- Onde os viste?

A criança apontara para a sua cabeça com o indicador.

Atónita, ela olhou-o. Era um mistério, a cabeça dele.

Nessa manhã, o miúdo aparecera diante dela, com o seu ar muito sério.
Preocupada com a demora, sentara-se diante da janela, os cotovelos
fincados no parapeito. Hesitava se haveria de ligar à polícia.

Florimundo beijou-a e disse-lhe:

- O papá não volta, mamã. Foi viajar e disse-me adeus.

- Como sabes que não volta? Que disparate! Ele disse-te algo?

- Deu-me um beijo e foi-se embora.

Havia nas palavras do rapaz um tom que a afligiu, percebeu que ele
falava a sério. Agarrou no telefone e ligou para a polícia e para os
hospitais. Não havia sinal de Gabriel. Ligou também para a polícia
marítima, ao ver que o pequeno barco de borracha havia desaparecido.

Procurava, por todo o lado, o menor indício que pudesse dar-lhe a
informação do seu desaparecimento. Na biblioteca encontrou o livro
aberto e a folha onde Gabriel desenhara tantas vezes a imagem que o
matara. O homem sem rosto. E depois aquela frase, deixada entre rabiscos
quase ininteligíveis e que agora pareciam fazer todo o sentido:
\emph{``De espelho em espelho''.}

Esteve sentada, durante muito tempo, na biblioteca. Finalmente,
penetrava no mistério de Gabriel.

Sabia que ele tinha passado ali as suas últimas horas e podia ter
impedido a sua última loucura, ainda. Talvez bastasse, como sempre o
fazia, sem muitas palavras, oferecer-lhe uma fatia de bolo e um chá
quente, a lembrar-lhe que a vida existia fora das suas imagens.

Escondeu a frase, pensando que Florimundo poderia encontrá-la. Escondeu
também os desenhos, que tanto a perturbavam. Adivinhava a tortura que
essa imagem de um rosto sem conteúdo, confinando apenas consigo próprio,
devia ter exercido sobre o espírito frágil de Gabriel.

Florimundo ouviu um grito na biblioteca. Subiu as escadas a correr e foi
dar com a mãe hirta e de olhar enlouquecido. Pela primeira vez na sua
vida sentiu-se só.

O Verão acabara. As primeiras chuvas se setembro tombavam sobre a terra.
As memórias doíam. No primeiro Natal depois da morte do pai não houve
festa, a mãe fechou-se no quarto, por dentro.

Nesse Natal soube que o Pai Natal era um velho ingrato, pois não chegara
à sua casa, exilada da alegria. Os avós tinham vindo buscá-lo e a mãe
continuara fechada por dentro. Trouxeram-lhe canções de Natal, porém o
olhar era pesado. Chovia, sempre.

Depois não voltara mais à casa, vendida pelos avós. Por todos os cantos
da casa, a mãe via o rosto do pai a chamá-la, perseguindo-a no alpendre,
saindo do mar, com um sorriso a dançar-lhe nos olhos. O pai muito alto e
a sua figura a sair do mar, com a água a escorrer-lhe pelo corpo. O pai
ria-se, gelado até aos ossos. E Clara esperava-o à saída, com uma
toalha. Muitos anos mais tarde, Clara guardava ainda a recordação do
riso solto de Gabriel, ecoando na manhã, com a alegria das coisas
acabadas de nascer. Gabriel retornava à sua própria infância.

Lembra-se ainda dessa fatídica madrugada de setembro, em que o pai se
detivera a olhá-lo, à porta do quarto, com uma tristeza infinita.
Revia-lhe o olhar e lamentava não ter sabido retê-lo. Porém, esse olhar
derradeiro conhecia uma paz e uma serenidade que o filho não lhe via há
muito tempo.

Não conseguia sentir raiva ou culpá-lo, pois tinha a certeza de que ele
o amava, onde quer que estivesse. Com o tempo, aprendera a reunir todas
as peças desse misterioso puzzle e compreendera que certos homens não
são feitos para a vida, mas para habitar espaços e tempos entre sonhos,
não sabendo como resistir à realidade.

A mãe designava essa incapacidade por loucura e lembrava-lho
frequentemente, frisando uma certa amargura que, pouco a pouco, se fora
esmaecendo. Ele não sabia como dizer-lhe que não acreditava na loucura e
que jamais aceitaria essa ideia, sobretudo quando se aplicava àquele ser
que amara tão intensamente.

Como o seu corpo jamais fora encontrado, guardava uma memória límpida da
sua morte, longínqua, perdida na neblina da madrugada, entre o canto das
ondas e o cheiro, a luz do mar. Como se tivesse sido um sonho.

Ele tinha-lhe dito, uma vez, que procurava uma ilha, mas que sabia,
também, que essa ilha se encontrava em si. Uma ilha onde não havia medo
nem desejo ou ambição. Não havia ``entre'' mundos nem lutas interiores.
Nada, a não ser a luz, a areia, o mar e o silêncio, a toda a volta.

- Papá...- interrompera-o - as ilhas não são lugares?

Gabriel olhara-o em silêncio, durante algum tempo. Não saberia explicar
a uma criança tão pequena que existem ilhas que não são lugares, mas
pontos de reencontro, utopias de regresso.

- E onde existe essa ilha? Também posso ir?

O silêncio permaneceu e Florimundo compreendeu que era inútil perguntar.

O ruído do eléctrico ecoou na noite. Um lamento agudo que anunciava a
paragem. Depois partiu novamente, rangendo na subida. O rapaz podia
reproduzir todos aqueles sons até à exaustão. A sua concentração
desenvolvera nele uma memória auditiva invulgar. E ela estendia-se muito
para além dos limites do normal. Era capaz de reproduzir o som de algo,
mesmo sem ouvir. O ruído da água a cair do jarro para um copo, algo que
representamos imperfeitamente, ou o de uma folha que, esvoaçando e
dançando no ar, cai finalmente por terra.

Esse jogo divertia-o imenso, tentava imaginar a correspondência musical
com uma determinada nota, um compasso musical que reflectisse o tempo
demorado em cada gesto. Aprendera a perscrutar com a máxima concentração
os mais ínfimos sons. Dessa forma, havia entre ele e o mundo animal uma
proximidade arcaica, estabelecida por essa intuição auditiva.

Pouco a pouco, a intensidade das bátegas crescera, enquanto a luz dos
candeeiros atravessava a noite, fazendo brilhar os troncos húmidos.
Incansável, a chuva caía, em fios que se desenhavam contra a luz. Um
ruído fustigava as vidraças da janela. Além da chuva, o vento fazia
vibrar os vidros. Schubert chegara ao fim da sua \emph{Viagem de
Inverno}, compreendeu, ao ouvir os últimos acordes. Trágicos. Esse final
lembrava-lhe sempre a pintura de Bram del Velde, as bandas de cor
desamparadas, como se errassem a procura do sublime, de algo que as
resgatasse ao inacabamento. Descobrira a sua pintura na biblioteca do
pai, que possuía inúmeros catálogos desse pintor.

Possuía agora um ouvido tão treinado que já não precisava de realizar
nenhum esforço para operar essa transformação. Ela ocorria naturalmente,
sem que ele precisasse de esforçar-se. Por vezes, até tinha medo de não
vir a controlá-lo. Gostava do que ouvia, do que lhe chegava. Se um dia
viesse a ouvir outra música que não desejasse? Ele limitava-se ao papel
de um simples actor ou de um guardião. Cabia-lhe, apenas, a
manifestação, a apresentação do modo como ela lhe chegava.

O jovem músico persistia na ideia da construção de um vocabulário
musical que lhe possibilitasse uma combinatória infinita de sons. Para
poder exprimir o mundo enquanto totalidade. Delirava, certamente,
pensando nisso como tarefa quase impossível, mas achava que as notas e
os sons produzidos eram insuficientes para exprimir a pan-musicalidade a
que a sua audição tinha acesso. Ocorria-lhe a frase de Leibniz, em que
ele afirmava que \emph{"a música é a álgebra de Deus."} Ou o dito de
John Cage, defendendo que \emph{``tudo era música''.} Ambas as
afirmações eram inteiramente verdadeiras, pelo menos para ele, que
sentia trazer esse conhecimento consigo.

Gostava de imaginar que os sons que ele conhecia correspondiam a uma
ínfima parte das possibilidades de escuta. O que ouviria o insecto ou a
larva, no seu casulo? Seriam esses sons, ainda, audíveis? O que ouviria
a ave quando voava? Ou a águia no cimo dos desfiladeiros? O som do mundo
para um recém-nascido? Tentar imaginar essas possibilidades levavam-no,
cada vez mais, a dilatar os limites da sua audição, convertendo o ouvido
numa faculdade cósmica. O que ouviria o golfinho ou a anémona, ao
escutar o bater das suas membranas na água? Ou uma baleia nas
profundezas do oceano? Seria necessário, no seu caso, não apenas usar o
ouvido e a imaginação, mas partir da ideia para a sua apresentação
musical. Fazer um movimento inverso e artificial. Mas poderia também lá
chegar por uma espécie de analogia, o que, sem dúvida, lhe tornaria tudo
mais fácil. Como poderia estabelecer uma passagem entre a visão e a
audição?

Pensou que o mar e a água representam para o homem o silêncio. O fundo
do mar uma espécie de ausência do som. A ideia, construída a partir da
aparência e do modo como vemos as coisas, era totalmente inadequada. Um
dia, certamente, produziria os sons que o fascinavam, mas essa ideia
musical teria de ser extraída a partir de si próprio, num movimento de
toupeira incansável. Mais do que ver e ouvir, ansiava por um perceber
originário, que lhe permitisse ouvir o universo, na sua inocência
primeva.

\section{\textbf{AS ASAS DE
SATURNO}}

Um som baixo, como uma espécie de silvo rente ao silêncio, despertou-o
dos seus pensamentos. Uma borboleta nocturna, escura e de asas irisadas,
caíra no chão. Cega, investira pesadamente contra a janela, procurando o
ar nocturno. Tentava furar o vidro. As suas asas, grandes demais para o
corpo, batiam numa fibrilação pesada, deslocando camadas de ar
invisíveis. Refazia o trajecto vezes sem conta. Começava a subir a
vidraça, caía e voltava ao início do seu suplício. Uma cartografia
traçada de forma regular, sulcando incansavelmente os mesmos caminhos,
voltando atrás, lutando de forma absurda contra o seu destino.
Finalmente o cansaço obrigou-a a sucumbir e o ruído da sua queda
cristalizava o som do inelutável. Anunciado num baque.

O jovem arrepiou-se, horrorizado. Por mais que se tentasse adiar o
final, o baque seria trágico.

No cadeirão junto à janela, descortinou uma figura.

Envolto na sombra. Metade do rosto era visível, envolto pela luz que
vinha da rua, num tom sobrenatural. Do outro lado do rosto, era a
escuridão total. Apenas se podia ver o branco do olho. Tudo nessa figura
de homem parecia artificial, composta à pressa para parecer verosímil.
Lembrou-se de alguns quadros barrocos, em que os rostos ostentavam
aquela cor cadavérica, meio esverdeada, num rosto demasiado pálido. O
estranho era ver-se apenas um lado desse rosto. Ou um ou o outro, nunca
os dois em simultâneo.

De repente, Florimundo ouviu a sua voz. Grave, cavernosa, arrepiava.
Disse alguns versos que ele não conseguiu situar.

-...\emph{Ver-me e ouvir-me, contemplar-me o rosto;/ Acedo ao forte
apelo, e é com gosto/ Que aqui me tens! - Que mesquinhos tremores/ Te
assaltam, super-homem? E esse grito}\textsuperscript{\emph{\footnote{Goethe,
  Fausto, 485, editora Relógio de Água, tradução de João Barrento, p.
  56. }}}\emph{...}

Atemorizado pelo tom da voz e pela inesperada presença, que ali havia
entrado, sem ele fazer a mínima ideia como, tartamudeou:

- Quem és tu? Como é que entraste no meu quarto?

- Perguntas?! - Embora ele não lhe visse o rosto, sentia o tom irónico
da sua voz - Deixo-te algum tempo para adivinhares. Foste tu que me
chamaste...

Uma lâmina de ar frio percorreu o quarto. O rapaz tremeu. A janela
permanecia fechada. A borboleta jazia no chão.

- Acendo a luz... - propôs o jovem. - Sempre via o teu rosto.

- Talvez prefiras ouvir-me a olhar-me...

No espelho do quarto parecia ter-se acendido uma labareda.

Florimundo estava perfeitamente abismado. Alguém sonhara aquele encontro
e ele entrara no sonho dessa pessoa?

- Que te assusta? Acalma-te...

- Chamas-te?... - Interrompeu-o bruscamente o rapaz.

- Dá-me os nomes que te apetecer. São múltiplos os nomes da danação,
múltiplos os rostos...- sorriu, antes de lhe fazer uma pergunta
desconcertante:

- Tens tremoços? - Florimundo sentia-se atónito.

- O Santo Agostinho costumava dar-me tremoços... refresca-me... -- Antes
que Florimundo reagisse, retomou - Todas as categorias se revelaram
ineficazes para me classificar. Poderíamos, até, dissertar longamente
sobre isso...embora creia que o louco de Sils Maria já tenha dito tudo
sobre o assunto! Pelo menos, tentou ser racional. Detesto misticismos
baratos...

- Se te referes a quem eu penso...- Odiou-se a si próprio pela
ingenuidade. - Não me parece que seja assim tão fácil classificá-lo de
louco. -- Porquê tremoços? A esta hora?

- Não compliques! - Retorquiu-lhe o outro. - Alguém incapaz de se reger
pelos parâmetros sociais...que escreve e assina com aquele nome, depois
de escrever um livro com aquele título... Ora, o \emph{Anti-Cristo...}

- Ah! Um inconformista. Se ele não tivesse escrito assim, não estaríamos
aqui a discuti-lo.

- Toda a concepção do teu filósofo era louca, inaplicável. Oh...esse
livro, o do Zaratustra, será que é possível pensar a liberdade e a
superioridade humana daquela forma? Toda a filosofia dele não passa de
um louvor a Dioniso! A alegria de Zaratustra, a sua leveza... é a das
bacantes, não achas? Porque as putas que ele conheceu eram mais graves!
Só a estroinice das bacantes é que podem desfazer Orfeu. A trivialidade
das Bacantes é que o destrói. Enquanto os sátiros e as ninfas se
distraíam, naquela explosão de luxúria... rabos, coxas, mamilos. Sempre
foi o meu alimento preferido, o sexo. Sabes, porquê? Porque nele tudo se
equivale. A seriedade platónica é que me aborrece.

Florimundo viu-se a rebater, sem o desejar, como se não comandasse os
seus labios:

- Não sejas tão simplista! Tudo aquilo é uma sucessão de metáforas.
Muito conseguidas, por sinal, se reflectires sobre o destino da arte. É
possível pensar a criação sem a libertação do peso? Era preciso devolver
a vitalidade à arte. Ela estava completamente submetida aos valores,
anestesiada... comodamente rendida a um maniqueísmo estéril, de uma
ética que em nada...

- Tsst...tsst! Falas contra a ética burguesa, mas és incapaz de
superá-la, no verdadeiro sentido. Ao pé desse filósofo és um menino de
coro e ele... coitado... Os filósofos são uma cambada de
impotentes...toda a arte nasce da violência e da possessão, da paixão
desenfreada, da incapacidade de raciocinar filosoficamente ou de reduzir
o mundo à racionalidade. Mesmo Nietszche não terá dito essa frase que
faz o meu deleite?

- Que frase?

- Que tínhamos a arte para suportar a verdade? Não era isso que ele
queria dizer? Que toda a arte nasce da violência e é uma canalização dos
instintos?

- Que horror! Repara no que dizes. Imagina uma comunidade artística toda
ela dominada pela violência... bem vês que os artistas são seres de
urbanidade.

- Sim, excepto quando se auto-destroem a si próprios e se encharcam em
drogas e álcool, para fugirem à possessão... e não são poucos os casos!
Ou os que deixam de o ser, porque não aguentam mergulhar nas suas
entranhas, suicidando-se...

Florimundo percebeu que ele se referia a Gabriel. O rapaz não reagiu.
Guardava aquele segredo consigo, mas agora compreendia, em sua
amplitude, a dor do pai.

- Bem...mudemos de assunto! És tão meditativo e ao mesmo tempo
pareces-me tão naïve, mas vejo que estás bem instruído. Quem foi o
músico que te ensinou filosofia? - O tom de voz era irónico, mas
perturbador. Florimundo percebeu que ele sabia demasiado acerca de si. -
Na verdade, um assunto tão inextricável é cansativo... atalhemos pela
simplicidade: é preciso aceitar o que não se entende... - acrescentou,
ultimando o assunto.

Florimundo tinha frio como nunca tivera, em toda a sua vida. Mesmo nas
montanhas, perto da casa dos avós. Mas um frio que se tornava
irrespirável, queimando tudo à sua volta. Um frio seco.

- Bom, posso chamar-te "senhor Metamorfose"? -- Batia os dentes de frio,
mas apetecia-lhe provocar o seu ilustre visitante. - Ou o "senhor dos
Círculos"? Sempre achei alguma graça à designação...

- Sim... Ou, por que não, Mefistófeles, o mais literário de todos os
meus nomes? - O riso casquinado irrompeu da sombra. - Escreveu-se tanto
sobre mim! O meu poder é tentador...

- Alucino? Isto é um sonho? - um certo tom desesperado indiciava a
confusão do rapaz. - Este frio!

- Há alturas na vida em que as coisas se fundem. São partes de um todo.
Sonho ou realidade... será que isso importa verdadeiramente?
Actualmente, parece que as distinções e os rótulos se tornaram mais
importantes que os próprios conteúdos... a tirania do progresso. Eu sou
demasiado velho para me incomodar com isso. Preocupa-me a arte, apenas.
Se existe ou não a danação, os limites, tanto me faz... Interessa-me o
que fazem deles\ldots{}

- O mistério da transfiguração...- respondeu timidamente o rapaz, sem
saber o que dizer.

- Ora, lá vens tu com conceitos caducos... Sempre achei esse conceito
parvo. Quem faz transfiguração? O artista? Ou são meia-dúzia de
parasitas idiotas que afirmam a pés juntos que o artista o faz? Ah, é
preferível falarmos de metamorfoses, de labirintos e de sonhos, desse
vaivém que alimenta a criação humana. Transfiguração é o quê? Uma arte
culinária? Pegas em meia-dúzia de elementos, mistura-los e já está!
Soa-me tudo tão artificial.

Respirou fundo, interrompeu-se e retomou a palavra. A voz cavernosa e
sibilante continuou:

- Olha e se mudássemos de assunto? Não tenho muita paciência para os
argumentos! Fartei-me de ouvi-los durante a Escolástica... fiquei
enjoado para o resto da vida...

- Fica-te mal! Devias respeitar o adversário... - respondeu rapidamente
o rapaz, antecipando-se ao que ele ia dizer.

- Ora... como se eu não saísse sempre vencedor! \emph{Ele} revela um
total desconhecimento da natureza humana, uma inexistência da
compreensão psicológica, biológica, social, desse animal entorpecido
pelas paixões. Tretas! Já eu, pelo menos passei a divertir-me mais.

- Pois! Não nego que tu estejas mais próximo da natureza humana. Da
vaidade, da ostentação, da inveja. - Respondeu o rapaz, com sarcasmo.

- Diz claramente o que pensas. Da tendência para o mal, ias acrescentar.
Quer se queira ou não, é o que predomina! Basta olhar em redor. O homem
é um ser tão básico! Se avaliares com atenção à tua volta, verás o que
atrai o homem, os seus interesses, a sua vontade de poder. Sobretudo
quando eles arvoram a bandeira da dignidade. Dá-me vontade de rir.
Chegam a acreditar em si próprios. O inferno está cheio de poetas, de
artistas, de homens que passaram a vida inteira a dizer que eram capazes
de melhorar o mundo através da arte! Não queiras misturar arte e vida.
São duas coisas que não podem misturar-se. A arte redime-lhes a
existência terrena, maldosa, imperfeita...

- Pareces pesaroso em relação a isso. Deverias mostrar-te triunfante.

- Rejubilo! Irrita-me é que não reconheças a evidência do facto. Os
factos, as pequenas e ininterruptas guerras em nome de nada, por um
punhado de dólares, a sustentar a indústria do armamento, as economias
dos países assentes sobre a lavagem de droga, o aval dos bancos e metade
do mundo a aplaudir carnificinas e actos terroristas... coitados! Só
existe o Homem, esse ser que é capaz de tudo. Capaz de matar enquanto
ama. Capaz de compor a mais bela sinfonia, enquanto odeia, e criar na
mais árdua adversidade... essa é até a sua grandeza! Irritam-me as boas
intenções, o querer fazer com que todas as manifestações da arte se
colem a objectivos respeitáveis e que se continue a acreditar nisso. É
tão risível!

- O teu mundo é escuro. Denso e escuro como breu. Mas há justos no
mundo, quase me atrevo a chamar-lhes anjos.

- Ah ah! - A gargalhada irrompeu sinistra, ocupando o quarto, empurrando
o frio em vagas até ao pobre rapaz, que já se havia embrulhado em tudo o
que havia à sua volta. - Anjos! Esqueces-te que sou um anjo...

- Caído...- atalhou Florimundo.

- Um anjo nunca deixa de o ser. Ainda que seja um ser estropiado, com as
asas calcinadas. Compreende em simultâneo a dor e a impotência humana e
sabe que pertence a um outro mundo. Os anjos conhecem toda a pauta do
mal e já não existem asas para evitar o frio\ldots{} Em boa verdade te
digo, é o mesmo, o que mata, o que ama, o que trabalha, o que cria.
Todos são a face do mesmo. E isso é que é infernal, todos serem o mesmo,
a repetição do mesmo...

Aliás, creio que o teu pai o sabia bem...lembras-te? - Interrompeu-se,
como se desejasse ouvir a confirmação.

Florimundo compreendia agora uma série de paradoxos que, durante todos
aqueles anos lhe haviam resistido à sua compreensão.

- Um homem admirável, um mago das palavras, um tecelão de artifícios,
porém tão trágico... Esse saber infernal, ele possuía-o e foi isso que o
matou. Nenhum homem consegue viver com tal saber. Só poderia tornar-se
um cínico, algo que ele recusou sempre e que iria inteiramente contra os
seus princípios.

- Princípios? E a individualidade, o rosto? Essa concepção do mal
destrói a dignidade do rosto...a crença na possibilidade do humano.

- Dignidade humana, já viste ideia mais balofa, batida e estéril? Falas
de dignidade, assim, de ânimo leve, enquanto milhões de crianças morrem
ao frio, de fome, prostituindo-se? Enquanto se bombardeiam civis em nome
da paz? Já pensaste que a maior parte dos seres humanos nem sequer sabe
o que significa essa palavra?

A ideia de que o pai também teria tido semelhantes encontros incomodou-o
de forma obsessiva. Tinha do pai uma ideia que em nada combinava com
esta nova imagem, a de um conhecimento infernal. Sabendo ainda que esse
conhecimento e a impossibilidade de recriar um mundo o tinha destruído.
Tiritava de frio, enquanto pensava nisso.

- Talvez esta alegoria te ajude a pensar o homem. O Homem. Imagina um
ser perfeito, em que tudo se concentra, o espelho de Deus, à sua imagem
criado.

- Sim, cliché, adiante...um Golem! Criado a partir do barro...

- Imagina que esse espelho se fragmenta e se dispersa pelo mundo, numa
repetição infinita... É isso...um pedaço dessa imagem de Deus,
estilhaçado até ao mais pequeno fragmento. Cada fragmento, mostrando o
lado separado do todo, cada pedaço reflectindo a angústia, a nostalgia
dessa perda da imagem originária. Cada homem é um pedaço desse imenso
caleidoscópio que é a realidade estilhaçada. Poderíamos fazer um esforço
para pensar essa visão.

Florimundo tapou os ouvidos. Não queria ouvir mais nada. A alegoria era
horrível. Uma analogia da própria dispersão das línguas, do tempo, do
paraíso. Mas ampliada por um espelho gigantesco. Fragmentado, partido. A
morte era o reconhecimento final dessa imagem. Esse era o mal, o
conhecimento de que a morte se apresenta em cada pedaço ínfimo da
natureza, mesmo no mais pequeno átomo, na mais bela mulher e na mais
inocente criança.

Florimundo desejava agora não mais pensar nisso. Reconheceu que isso era
o que o pai passara a vida a fazer. Uma vez mergulhando nas vísceras do
mal, era impossível olhar de frente a claridade do mundo.

Uma sensação de desmaio e de vertigem deixou-o prostrado.

O outro continuou, apesar do silêncio do rapaz.

- O que sabes tu da vida, da natureza humana? Só pensas \emph{em}
música. Sim, disse bem, não faças esses olhos arregalados. Pensas
através dessa linguagem que escorre em ti. Que achas tu que sabes da
natureza e da linguagem dos homens?... Bem te esforças, pequeno, mas não
adianta. Sempre foste mais que tu, um ser que continha em si essa
representação de um ideal. Aliás, como Gabriel, como todos os meus
filhos. Há quem lhes chame os filhos de Saturno. O seu olhar é interior
e quando olham à sua volta vêm apenas...

- Filhos de Saturno? O que significa isso? Que espécie de horror é esse?
--

O rapaz entrara num solilóquio delirante, parecendo não se dar conta da
presença do outro:

- Comecei por observar uma borboleta que morria contra a noite,
esmagando-se no vidro...agora descubro que estiveste sempre a ver-me. E
que sabes tudo sobre mim. Porquê, pode saber-se? Além disso, devo
confessar que tudo isto é assustador\ldots{}o que sou eu, afinal? Talvez
me possas responder, já que não sou igual aos outros. Um anjo, um
demónio? Gostava verdadeiramente de sabê-lo.

- Essas perguntas são inúteis. Descobrirás por ti ... Ao menos tu
deixar-nos-ás a tua bela música. A beleza é o outro lado do horror. -
Continuou o outro - E nunca perguntes acerca de ti próprio. Há uma
tremenda vanidade nesse gesto. Não sei bem o que é pior, se a vaidade ou
a inutilidade!

O rapaz estremeceu. Sentiu que o estranho sabia exactamente o que lhe
havia passado na cabeça, nesse momento.

- Foste tu quem produziu aquele fenómeno da borboleta, para me testares?

- Limito-me a ser um reflexo do teu pensamento. Se quiseres, uma
alucinação. Aquela é a tua condição, se não fores capaz de atravessar o
vidro... apenas sugeri um pequeno fenómeno óptico...olha, repara! -
Enquanto isto, estalava os dedos e a borboleta desaparecia.

Ele ria. Abertamente. Um bufão que transformava tudo numa ilusão, cuja
gargalhada sonora fazia estremecer os vidros, o ar, os espelhos do
roupeiro.

E de repente pôs-se a citar, pensava Florimundo que seria de memória.
Com uma memória antiga e perfeita, como a de um sacerdote:

- \emph{"Todo o homem de elite aspira instintivamente à sua torre de
marfim e reclusão, em que se libertou da massa, dos muitos, da maioria,
em que pode esquecer a regra "homem", sendo ele próprio a sua
excepção..."}\textsuperscript{\emph{\footnote{\emph{\textbf{Nietzsche,
  }}"Para Além do Bem e do Mal", 26.}}}\emph{.}

- Extraordinário, não é? Só depois poderás compreender o que te está
destinado...

Quando o rapaz quis responder, sentiu a sua ausência. Ele desaparecera.
Imediatamente, o frio desapareceu, levado por ele. Tudo se aquietara,
subitamente. O espelho não conhecia nenhuma espécie de luz. Era a noite,
apenas. Lá fora, a existência da neblina e a chuva atestavam a
realidade. O odor de pétalas molhadas, que tinham um cheiro menos pesado
e atenuado pelo peso da água. A ternura da escuridão, com o seu halo de
suavidade.

Correu a acender as luzes, procurando o mínimo vestígio do insólito
visitante. Nada se ouvia, a borboleta desaparecera, a chuva continuava a
cair, interminável. Tinha fome, mas uma vontade imensa de compor
arrastava-o, uma doença imperiosa. Necessidade de rasgar o azul opaco da
noite, de inscrever nela a incandescência dos sons. O seu pensamento
trabalhava a um ritmo impressionante. Não se sentia nada cansado, apesar
de ter estado a trabalhar, durante todo o dia. Era como se tivesse
acabado de acordar.

Ao passar pelo armário, olhou-se. Assustou-se. Viu um homem que
permanecia de costas. Não conseguiu ver o seu próprio rosto.

Durante dias não saiu de casa. Compunha obsessivamente, sem horas nem
regras, possuído por uma euforia estranha e inexplicável. Uma nova
consciência emergira nele. Um conhecimento do obscuro, que muitas vezes
o deixava aterrorizado. O seu humor tornara-se irregular.

Durante vários dias, não saiu nem sequer para ir às aulas ou ao jardim.
Achava tudo o que o distraía uma pura inutilidade. Arrastado por uma
clareza e uma lucidez tais, que produzia de forma compulsiva.

Saíram dessa lavra duas sonatas para piano e violino, numa homenagem aos
seus compositores favoritos. Escreveu algumas canções, ainda muito
influenciadas pela sua paixão pelo romantismo tardio, muito pouco
consonantes com o espírito do seu tempo. Iria deitá-las fora,
certamente\ldots{}

Com efeito, havia estudado as correntes contemporâneas, desde o
dodecafonismo de Schönberg às escolas minimalistas, revelando um
conhecimento e um domínio da composição verdadeiramente precoces.
Recusara sempre, no entanto, a dominância do atonalismo, pois sentia-se
indiferente perante as vanguardas musicais. O autor que mais o
impressionava era Webern, pela extrema condensação. Esse mesmo silêncio
que o pai lhe havia ensinado, quando lia poesia no alpendre da casa da
praia. Quando \emph{Burnt Norton }ecoava pelo roseiral e despertava os
seus visitantes invisíveis.

Nunca tivera uma atitude elitista perante a música, o que lhe permitia
estar disponível para a aprendizagem de ritmos mais populares, mais
ligados ao jazz, por exemplo, e ao soul. Havia uma componente emocional
que o atraía nesses ritmos e que não sentia na música serialista e de
vanguarda, o encontro com a emoção e com a melodia, livre de
preconceitos. Tinha-se interessado igualmente pela música indiana,
talvez a mais próxima da música pitagórica, porém, as atitudes
subjacentes, o aspecto religioso que sempre se lhe encontrava associado,
desmotivavam-no.

Céptico, ele rejeitava as crenças religiosas, as grandes ideologias em
geral, as teorias que anulavam a singularidade e a liberdade
individuais. Ser-lhe-ia intolerável aceitar qualquer imposição à sua
liberdade. Intimamente passou a desprezar manifestações culturais de
massas, fenómenos de popularidade e de sucesso imediato. Contra essas
manifestações opunha a ideia de um esforço árduo, de uma conquista feita
a pulso, selada com suor. A única verdade que existia no mundo, pensava.

Recordava-se nitidamente da sua infância, inteiramente consagrada ao
estudo do piano, à aprendizagem do solfejo. A primeira professora de
piano que tivera, a dona Laura Pimentel, antiga professora do
Conservatório e que já se encontrava reformada, habituara-o ao trabalho
regular. Começara a aprender piano com ela aos cinco anos. Ao fim de
três, a ríspida senhora demitia-se das suas funções, confessando a Clara
que já não tinha nada para ensinar ao filho. Aconselhou-a a pô-lo a
estudar no Conservatório, pois a criança revelava dotes fora da média e
uma invulgar capacidade de concentração e de aplicação ao trabalho.
Clara seguiu-lhe o conselho. A passagem para o Conservatório aliviou-lhe
um pouco a rigidez da educação, mas habituara-se a olhar para as outras
crianças como seres estranhos. Passavam o tempo todo a brincar, não
tinham qualquer interesse pela escola e a maioria andava no
Conservatório apenas para agradar aos pais, não por vocação.

O único luxo que se permitia eram os passeios pelo jardim, ao final da
tarde. Por vezes, a atenção do rapaz era distraída por uma colega sua,
de traços reservados, silenciosa. Margarida sorria-lhe quando o
encontrava, mas ficavam estúpidos a olhar um para o outro. Observava-a
nas suas brincadeiras de menina, integrada num grupo de miúdas da mesma
idade. Todas elas mais ousadas e capazes de o pôr rapidamente à
distância, assustado pela sua desenvoltura.

À medida que os anos passavam, habituara-se à presença de Margarida,
ligando-os o silêncio manso, carregado de uma pertença. Passavam um pelo
outro, sorriam e continuavam o seu caminho. Ela caminhava com lentidão,
consciente da sua presença. Tinha um rosto oval, pálido, olhos
castanho-claros. Usava o cabelo atado num rabo-de-cavalo, como as
bailarinas. O penteado fazia ressaltar a curva do seu nariz.

O sorriso da rapariga tinha um enorme poder sobre ele. Sem que o rapaz
percebesse a razão. Ela estudava canto. Às vezes, passava junto à sala
em que Margarida se encontrava a cantar. No meio de outras, ele
conhecia-lhe a voz, inconfundível, que o mergulhava num silêncio feliz.

A memória da voz e o sorriso uniam-se, para desenhar-lhe o retrato da
mulher ideal. Os seus olhos rasgados, cor-de-mel, entravam-lhe a direito
no coração. Transformara-a num sonho, num talismã que transportava
consigo, poderoso e erguendo-se contra a solidão e a banalidade dos
dias. Sempre que revia o seu belo rosto, era como se estivesse a ver um
quadro de Rossetti, onde as raparigas se assemelhavam a delicadas flores
de fogo, vestais modernas, rodeadas de lírios e rosas, olhos e silhuetas
de ninfas, divinas, etéreas, fechadas no seu mistério. Intensamente
pálidas e vibrantes de sensualidade. Sim, pensava o rapaz, ela tinha
pequenos seios e ele evitava olhá-los para não se perder neles. O corpo
estremecia-lhe.

\section{MARGARIDA}

Nessa manhã, Margarida levantara-se tarde. Tirou a camisa de noite leve
que trazia e olhou-se ao espelho. Viu os seios, firmes, pequenos e
redondos, um ventre liso e as pernas longas e magras. Preparava-se para
ir para o banho. Pôs um CD do Nick Cave a tocar.

O seu corpo começou a dançar, era mais forte do que ela. O sol assim a
entrar pela janela, o corpo reflectindo-se no espelho, as pernas a
ondularem. Era bom começar o dia a dançar, enquanto o mundo lá fora
continuava o seu curso. O corpo nu, apenas coberto pelo cabelo, que lhe
caía pelos ombros. Era ela, a transformar-se em energia e leveza puras.
Ela, sem tristeza, ela menina, como no tempo em que o pai lhe chamava
``minha pequena Margot''.

A voz do pai chamou-a. Perguntou-lhe se estava acordada.

Correu a vestir o robe. Parou de dançar. O pai batia à porta.

- O que tencionas fazer? Passar o dia trancada no quarto a dançar?

- Como sabes? Agora espreitas-me pelo buraco da fechadura?

- Ora...nunca te libertaste dessa obsessão...e é bom que nunca o faças!
- Respondeu-lhe o pai, pregando-lhe um beijo repenicado na face. - Anda
comigo. Tens aulas?

- Bem, não. Ia ficar por aqui a ler...ou a preguiçar.

- Tenho um pintor que me quer mostrar os quadros. Sabes que não dispenso
a minha menina. Dá-me sorte. Dás-me a tua opinião.

- Papi, mas eu não percebo nada de pintura....

- Não interessa. Tens essa sensibilidade. E vai lá estar o Aires, o
crítico de arte.

- É novo? É bonito?

- Quem, o Aires?

- Não, claro que não, o pintor...

- Sim, deves ter sido uma dessas raparigas do Rossetti, na outra
reencarnação... - respondeu-lhe a rir. Não conseguia zangar-se com os
atrevimentos da rapariga. Aliás, sabia que ela lhe dizia aquilo para lhe
provocar ciúmes.

- Bem...ou uma bacante\ldots{}ou a Isadora Duncan... Nunca se sabe. Sim,
vou tomar um banho e pôr-me bonita para o teu pintor. Assim, ele baixa o
preço e tu dás-me uma comissão.

- E depois pagas o que me deves com a tua comissão. Pode ser?

- És um sovina. Faz-me umas torradas e um chazinho. Vá lá...não custa
nada.

Ela desceu as escadas a correr. Trazia umas sapatilhas cor-de-laranja,
que não combinavam com a roupa que trazia vestida. Quando reparou que a
mãe, sentada numa cadeira, na mesa da cozinha, olhava fixamente para os
seus pés, avançou lentamente.

- Filha...

- Não vale a pena, mamã. Já sei que não gostas. O pintor vai gostar, não
vai, papi? Dá com artistas. Vais ver que ele aparece como os artistas
gostam de andar. Cheio de piercings no nariz, de rabo-de-cavalo. E
depois vamos ao cinema, não vamos? Tenho saudades de ir ao cinema
contigo. De te ver a chorar às escondidas, nas partes gagas,
românticas...E depois sais a mostrar que aguentas a vida, estóico, com
ar de mecenas de fato assertoado, que afinal é o que tu és.

- Estás a ficar uma cínica, querida. -- Riu-se, não conseguia zangar-se
com ela.

A mãe mergulhava o olhar no livro que estava a ler.

- O que é isso? Uau! Ainda impõem esse chato na faculdade? E
percebes?...Eu....- ia começar a dizer, quando foi surpreendida pelo
olhar reprovador da mãe, por cima dos óculos.

- Na tua idade, também não me dedicava a estas coisas. - Respondeu a
mãe, com o seu ar sério. - É preciso uma certa maturidade. O domínio dos
conceitos...

- E percebê-los...- acrescentou ela, enquanto o pai ria, pela
provocação.

- E tu não devias estar a rir, pois no outro dia bem te vi a falar de
Heidegger como se o tivesses bebido de berço...

- Eu esforço-me. - Acrescentou a rapariga. - E a mamã lê mesmo. Mas é
tão discreta, nunca a vi...

O pai interrompeu-a, adivinhando o sentido das suas palavras:

- Um dia vais perceber, minha querida, que nada na vida é o que parece.
A maior parte das pessoas faz parte desse universo de faz-de-conta. E se
não alinhas, és criticada... imagina eu, a dizer num dos meus serões:
``Heidegger, Levinas? Não, nunca os li... O quê? Pois, nunca li''. Vamos
mas é trabalhar.

- Sim, à tarde tenho aulas de canto, quase me esquecia, e afinal não
podemos ir ao cinema.

Quando o pai foi buscar o sobretudo, a rapariga estava à janela.
Observava Florimundo, sentado num banco do jardim. A casa de Margarida
ficava mesmo em frente. Toda a sua vida se havia organizado em torno
daquele jardim, na sua existência exclusivamente urbana. Fora naquele
jardim que uma brincadeira idiota a deixara lesionada e tivera de deixar
o que mais amava na sua vida: dançar. Fora ali que sempre se habituara a
ver a vida passar, os velhos à sombra da faia e aquele rapaz tão calado
e sombrio com quem jamais falara. Para quê? Sabia que ele a olhava desde
sempre, mas com toda a certeza nem sabia o que iria dizer-lhe. Sentia
curiosidade, mas tinha receio. E se ele nunca a procurara era porque
também não lhe interessava a sua pessoa. Andavam os dois no
Conservatório, frequentavam os mesmos horários e, no entanto, ele jamais
a procurara. Também raramente o via acompanhado. De resto, ela própria
também gostava de andar sozinha. Irritavam-na aqueles bandos de
raparigas barulhentas e ocas de cabeça. Demasiado ruidosas para o seu
gosto.

Mas gostava de vê-lo, com as suas roupas puídas, o seu ar de menino
mergulhado num mundo só dele, observador e atento, às vezes acompanhado
da mãe, uma mulher bonita, embora apagada.

O pai apareceu por detrás dela.

- Também já reparei. Gostas dele, não gostas?

- Mas não saberia como aproximar-me. É tão calado, tão solitário. E
parece uma pessoa tão bonita. E tu sabes que eu o conheço há muito.

- Sim, sei apenas que ele apareceu por aqui, um belo dia, já nós cá
morávamos. Tinha para aí uns oito anos, a mãe estava meio-louca e o pai
tinha acabado de morrer. Uma história triste. Trágica, mesmo. Aquele
silêncio é tão fora do vulgar. Não é como o teu. É uma recusa da
linguagem, da alegria...O teu é um silêncio feliz, cheio de alegria.

- Como soubeste essa história, papá? Porque nunca me contaste?

- Não sei, um certo pudor. A história soube-a um dia, no café. Quando
ela entrou no café, com as empadas e um ar sonâmbulo de quem não dormia
há muito tempo.

- Papá, há vidas tão pobres, tão vazias. Mas ele é um pequeno génio. É
compositor. Fala-se nisso, no Conservatório. Mostrou umas peças a um
professor, que comentou com alguns alunos. Chegou-me aos ouvidos pelo
Martinho, que é amigo dele. Mas também só vive para a música. Não sai,
não se dá com ninguém...acreditas no amor platónico?

- Que raio de pergunta a tua!

- Não, a sério, não é nada dessas mentiras que ouvimos sobre o amor. Nem
sequer tem a ver com uma atracção animal, daquelas incontroláveis...é
uma sensação de pertença...

- Bolas, estás tão metafísica! O amor não existe, fixa isto que te digo.
É uma projecção das nossas fraquezas...lembra-te do Narciso!

- És tão céptico! Porque é que os velhos teimam sempre em tirar as
ilusões aos jovens? Já reparaste como as ilusões embelezam o mundo?

O pai olhou-a muito sério. Ela tinha razão. O mundo sem ilusões era uma
um nojo. Teve vontade de lhe dizer que ainda era pior do que ela
imaginava. Mas recuou e respondeu, com o sorriso mais encantador e
confiante que possuía:

- Louca, é para não te magoares. Quero poupar-te o momento em que essas
ilusões caem. E é sempre tão cedo que não nos aguentamos, podes
acreditar!

Por mais que quisesse não conseguia afastar a ideia que lhe ocupava o
cérebro. A filha seria sempre a sua pequenina Margot.

\section{\textbf{A VOZ}}

Finalmente chegou a Primavera. Era bom passear ao longo das avenidas, as
pessoas tiravam os casacos que tinham usado durante o Inverno e que lhes
davam um ar soturno, pesado.

Os dias tinham-se tornado mais longos e Florimundo aproveitava as férias
da Páscoa para se dedicar à sua nova composição. Tinha desejo de se lhe
dedicar por inteiro, sabendo-se agora mais preparado para escrever uma
peça mais complexa, já longe das suas primeiras peças.

A saudade de Margarida doía-lhe. Habituara-se à sua presença, mas o que
mais o fazia sofrer era a ausência da sua voz. Não tanto o sorriso, mas
a voz, aquele narizinho a que ultimamente se habituara, fazendo
diariamente um percurso necessário para a ouvir. Foi justamente a pensar
na sua voz que começou a compor a sua peça.

Imaginando que ela seria cantada por ela. Uma voz feminina de soprano,
como o centro.

A ideia musical impunha-se gradualmente ao espírito. Uma carícia daquela
voz, tomada como o começo para o primeiro andamento, frase que se
repetiria e desenvolveria, de forma mais complexa nos andamentos
seguintes.

As férias da Páscoa passaram num ápice, pois o jovem despendia o tempo
de que dispunha e entregava-se ao estudo e à composição do seu novo
trabalho.

Lentamente abandonava-o a urgência que sentira na composição das peças
musicais anteriores, impaciente por ver o modo como isso resultava. Essa
impaciência, reconhecia-o agora, se bem que o exaltasse mais do que o
trabalho metódico e continuado, era sua inimiga. Trabalhava de forma
regular, indiferente ao desânimo, ao cansaço, às dúvidas e aos requebros
de uma criatividade melancólica.

Sentia que a arquitectura era tão complexa que só avançava lentamente.
Pedra a pedra, ou esculpindo toda a tensão, crescendo com o tempo. Já
não o entusiasmavam tanto os arroubos da inspiração, que rapidamente
apontava, mas de que desconfiava. Era importante voltar obsessivamente
ao mesmo, limpar e deixar só o essencial.

Acima de tudo importava-lhe a qualidade, as características dramáticas
da peça, já que elas deveriam apresentar-se na voz feminina. Sentiu que
era necessário, alternadamente, a presença de dois instrumentos, o piano
e o violino, para apresentar a sua ideia e dar-lhe essa subjectividade e
a intimidade que era exigida. O piano seria o modo como tentaria
exprimir a subjectividade, o violino introduziria a dialéctica musical,
um diálogo com o piano e a própria voz humana, um intermédio que seria
capaz de estabelecer a ligação. Exprimindo a tensão entre a voz e a
música.

Quando retornou às aulas do Conservatório, passou a dedicar a sua
atenção à capacidade expressiva da voz da rapariga. Jamais a ouvira
cantar de outra forma que não fosse assim. Ouvira-lhe uma única vez a
\emph{Avé Maria} de Gounod e ficara emocionadíssimo. Mas já a tinha
ouvido com repertório diferente, incluindo Messiaen e vários outros
compositores contemporâneos, em particular Gorécky. Procurava
aperceber-lhe e reter-lhe o timbre exacto, para que pudesse introduzi-la
na sua peça, numa procura de uma harmonia com os restantes instrumentos.
Num controle e domínio perfeitos da sua voz, tal como tentava obtê-lo do
violino e do piano.

Era capaz, ao fechar os olhos, de reproduzir mentalmente todos os traços
do seu rosto, a linha fina dos lábios, a cor da pele e a suavidade dos
olhos, sempre um pouco encovados, o que lhe dava um aspecto um pouco
abatido e frágil.

Era demasiado magra, mas agradava-lhe a delicadeza do seu corpo, de
gestos nervosos e rápidos. Gostava das suas pernas finas e altas e de a
ver de saia curta. Ela vestia, muitas vezes, saias e vestidos longos e
leves, que lhe acentuavam a delicadeza e a flexibilidade do tronco.
Trazia sempre uma espécie de brisa nas mangas dos casacos e o frio
reflectido no rosto, de malares salientes. Quando fazia calor, como era
de pele clara, a boca tornava-se-lhe muito vermelha e as faces coravam.

Bastava-lhe vê-la, saber que ela existia para ele e que, provavelmente,
ela não lhe era indiferente. Claro que os colegas troçavam da situação e
eram incapazes de perceber aquela relação platónica e um tanto
obsessiva, com tantas miúdas por ali. Sobretudo, manifestavam o espanto
pelo facto de ele se mostrar indiferente às outras, algumas delas bem
curiosas e interessadas pelo seu talento. Em especial, Olga, essa
morenita tão solícita, que só parecia esperar um gesto dele para se
aproximar.

Era bem verdade que ele não saberia explicar porque ela o fazia
estremecer e o coração parecia ficar apertado quando se cruzavam e se
entreolhavam. Ninguém possuía sobre ele aquele poder, que considerava
quase terrífico, sem tréguas. O amor, tal como o ódio, pareciam-lhe
sentimentos assustadores, pois não os sabia controlar. Porém, ainda não
tinha experimentado o ódio, apenas a raiva e a humilhação, em situações
pontuais e que nunca se tinham prolongado.

A voz da rapariga confirmava-lhe a presença desse sentimento
inalterável, que crescera ao longo desses anos. Jamais haviam
conversado. Mesmo agora, ao abarcar a extensão da sua paixão por ela,
não sabia se desejava falar-lhe ou procurá-la. Como qualquer apaixonado
o faria, dizer-lhe o que sentia, envolver-lhe o corpo num abraço. Tudo
isso lhe parecia assustador.

Sabia que a amaria enquanto ouvisse a sua magnífica voz, essa presença
esmagadora de uma beleza desmedida, que lhe pousava no coração como a
mais desejada carícia. Preferia olhá-la e observá-la de longe, numa
aparente indiferença. Pensava, no entanto, que quando tivesse a sua
composição pronta haveria de lhe pedir que a interpretasse. Esse seria o
seu gesto de confissão amorosa.

Sabia que iria terminar o curso nesse mesmo ano. Era o último.
Provavelmente iria fazer uma pós-graduação na área de composição, mas
fora do Conservatório. Ela também se encontrava no final de curso.
Moravam perto um do outro, mas não sabia exactamente onde.

Na última semana de aulas reuniu todas as suas forças e esperou-a, à
saída das aulas. Sentia-se um idiota chapado. Na verdade, era um
completo nabo, do ponto de vista do relacionamento com as miúdas, tinha
sempre enterrado o desejo sexual, amarfanhado essa parte da sua vida.
Era já um homem e ali se encontrava, sem qualquer auto-confiança ou
jeito para dizer fosse o que fosse.

Margarida saía no seu passo lento. Trazia uma saia leve e uma t-shirt
azul. Vinha de rosto concentrado, mordendo o lábio inferior, de sobrolho
franzido.

Florimundo viu-a avançar na sua direcção e tremia. As pernas compridas,
os seios pequenos a adivinhar-se por baixo da camisola. Tinha o cabelo
preso num rabo-de-cavalo. A seu lado vinha um colega, muito concentrado
no que lhe dizia e que ela parecia não ouvir, mergulhada no seu mundo.

Ele aproximou-se. Era agora ou nunca, pensou. Com o coração a bater
desalmadamente, perguntou-lhe se poderia falar-lhe durante alguns
minutos. Ela fez um sinal ao colega e ele disse que tinha de ir andando.
Depois sorriu-lhe, enquanto esperava que ele tomasse uma decisão.

- Há um café aqui em frente, mas podemos ir a outro lugar. - Finalmente
ganhou coragem e olhou-a directamente nos olhos. Sentiu-a inquieta.
Sabia que os seus olhos claros, entre o verde e o cinzento, faziam mossa
nas raparigas, conforme a luz pousava neles.

- Bem, tenho tempo, podemos ir até ao jardim, se quiseres. - Ela tinha
uma voz baixa e macia, falava quase em surdina.

Ele espantou-se como a sua voz podia atingir os agudos que ele lhe
conhecia.

- Também não queria roubar-te tempo. - Ficou logo arrependido, mas já
lhe havia saído a infeliz expressão.

Na verdade, o que ele desejaria era que o tempo acabasse para guardar
aquela voz, o sorriso e a beleza de Margarida.

- Podemos começar por nos apresentarmos - propôs ela, ligeiramente
divertida com a atrapalhação dele.

- Na verdade, há muito que sei o teu nome... Margarida, não é? -
Apetecia-lhe perguntar qualquer coisa idiota a seguir como "nome de
flor?", mas calou-se a tempo. - Ouvia as tuas amigas chamarem-te pelo
nome. Cresci a ouvir o teu nome.

- É bonito o modo como o dizes...e o teu? - Ela sorria.

- Bem, no mínimo é estranho: Florimundo.

Ela riu com vontade. Os cabelos dela haviam-se soltado de lado, um
cabelo liso, entre o castanho e o dourado. Tinha a boca comprida, lábios
cheios, uns dentes perfeitos.

- De facto. Mas também já o conhecia. Também é nome de flor, como o
meu...só que uma flor cósmica. "Flor do Mundo". É um nome poético.

- Nunca tínhamos falado. Não achas estranho, durante todos estes anos?

- E, todavia, conhecíamo-nos bem, não é? Sei muitas coisas tuas... Mas
não sou muito faladora.

- Ah sim? O quê, por exemplo? - Ele sentia-se atónito. Conhecia-a muito
bem, mas não supunha que ela também o conhecesse.

- Que compões, por exemplo. Toda a gente o sabe.

- E eu sabia que tu cantavas.

- Porque é que decidiste só hoje vir falar comigo? Bem, eu também podia
ter ido falar contigo, mas és tão reservado\ldots{} - a pergunta dela
era provocatória, ele sabia-o. Ela fora tão directa, que o deixara
desarmado.

Os cabelos caíam-lhe sobre o rosto e ela passou a mão, longa e delicada,
para os apanhar. O sol bateu-lhe de chofre nos olhos castanhos, que
ganharam um tom acobreado, cor-de-mel. Ele arranjou um pretexto qualquer
para se sentar. Apetecia-lhe dizer uma loucura qualquer, mas ficou
hirto.

- Na verdade, é a música que me traz a ti. Gosto muito da tua voz e da
tua expressividade. Não queria perder-te de vista. As aulas vão acabar e
gostaria de ficar com o teu contacto, para o futuro.

- Ah...meramente profissional? Percebo...- o tom dela era de desilusão.
Uma sombra instalou-se no seu olhar.

- Não, não penses isso... - pousara-lhe a mão no braço, sem querer. O
contacto da pele dela, pela primeira vez, deixava-o perturbado.

- É verdade que sempre te quis conhecer, mas não sabia como fazê-lo.
Sempre tão recolhida no teu mundo. Eu também sou tímido. Se não se
meterem comigo... - disse-o baixinho, como se este fosse o pior pecado
do mundo.

Ela olhou-o de frente.

- Sempre esperei que ultrapassasses este silêncio. Sonhei muitas vezes
com este encontro. Sabia que um dia haveria de acontecer.

O rapaz manteve-se calado. Estava comovido e embaraçado ao mesmo tempo.

- Mesmo agora, ao ouvir-te, sinto que trazes um mundo absolutamente
desconhecido no olhar. -- Retomou Margarida.

- Desconhecido?

- Novo...parece ser a expressão mais justa. Não saberia explicá-lo.

O rapaz sentia-se compreendido. "Ela sabia", apeteceu-lhe gritar.

No seu coração qualquer coisa parecia explodir, lágrimas por dentro,
prestes a deflagrar, lágrimas de felicidade que o deixavam desamparado.
Ela sabia-o, pensou ele.

A voz dela quebrou o encanto. Perguntou-lhe baixinho, como se tivesse
medo de interromper algo:

- Vamos encontrar-nos mais vezes, agora?

- Nunca deixei de ver-te. Ou de ouvir-te. Para mim, é a mesma coisa.
Todos os dias passava à porta da tua sala, para te ouvir.

- Algumas raparigas disseram-me isso. - Ela riu-se abertamente,
consciente do efeito de desorientação que provocava nele.

- Meu Deus! E eu a imaginar-me discreto...

- Se ser discreto é estar todos os dias e à mesma hora no mesmo
sítio!... - Ela voltou a rir e depois ficou subitamente séria.

- Não me respondeste.

- Como poderia deixar de te ver? Não me lembro de quando comecei a
reparar em ti. Tu consegues lembrar-te?

Respondeu negativamente, abanando a cabeça.

- Mas agora diz-me lá o que vinhas conversar comigo. - O tom de voz dela
era profundamente irónico.

- Na verdade, posso agora contar-te tudo. Estou a escrever uma peça para
a tua voz. Acho-a perfeita.

- A minha voz... - repetiu a rapariga com um olhar sonâmbulo - não sabia
que ela exercia este efeito.

- Sim, pensada como um instrumento, um centro. Aliás, já a comecei.

- Nem acredito! Há quanto tempo andas a pensar nisso? A trabalhar nessa
ideia? É tão obsessivo pensares em escrever algo para que eu cante. Para
a minha voz. Já imaginaste a minha responsabilidade?

- Há quanto tempo trabalho nessa ideia? Na prática?

- De facto. - Respondeu ela, curiosa. Isso indicar-lhe-ia a medida do
amor.

- Desde que a ouvi. A ideia pareceu-me irrecusável, sobretudo quando te
ouvi cantar a \emph{Avé Maria}. É uma ideia muito forte, não a musical,
mas a outra. Tens o timbre que eu queria, a suavidade que eu procuro.
Que é capaz de fazer romper...

Ela aguardou a resposta dele. Ele interrompera-se, procurando as
palavras certas.

- A escuridão da música, compreendes? Na verdade, ela aparece-me como
contraste para a música. Não consigo explicar a ideia com minúcia,
preferiria mostrá-lo. Tem sobretudo a ver com as minhas ideias
complicadas sobre a música e a sua essência. Ela não é
clara...compreendes?

- Parece-me que cada um vê o que deseja ver. Isso é simples de entender.
O que eu desejo na música é essa totalidade, a simplicidade ou a
claridade. - O olhar dela perscrutou-o, para averiguar se ele tinha
compreendido o que ela lhe dizia.

- Acho que não conseguirias perceber. É um abismo. Cada vez que desço,
perco-me nele, como se fosse uma noite eterna. A tua voz é a luz que
rompe essa escuridão...

- Estou perturbada com isso e curiosa para ouvir as tuas composições. Tu
conheces a minha voz, mas eu não conheço nada teu. Estás em dívida para
comigo.

- Claro. Estou desejoso que as ouças, mas, ao mesmo tempo, estou
ansioso.

- Ansioso?! Toda a gente te gaba. Até os professores...

- Creio que nunca ouviram senão coisas insignificantes. Pequenos
exercícios. - Depois teve receio de parecer pedante e emendou -
bem...quero eu dizer que já sinto uma grande evolução, o que ando agora
a compor já não tem nada a ver com aquelas brincadeiras. Sinto que
começo a libertar-me das influências, a ganhar alguma autonomia, mas, na
verdade, o risco parece ainda maior.

- A angústia do criador, hum...E tens a certeza de que queres fazer o
que fazes? Eu não tenho a certeza de querer cantar. Também sou
intérprete, o que é diferente. Por mais que varies, tens de cingir-te ao
que está criado, quando interpretas. Ao compores, a coisa torna-se bem
diferente. Inventas...

- Não sei se se inventa. É um mito, esse! Subverte-se, renova-se...

- Agora posso avaliar a qualidade, não é? - O tom dela era trocista,
porém essa era a forma que encontrava para lhe aliviar a insegurança,
fazendo-o rir.

- Não sei se te incomoda... - ia ele começar a dizer.

- O quê? Ver, ler as peças? Claro que não...

- Não, não era isso. Queria que compreendesses o que procuro fazer.

- Claro que não. Creio que nunca ninguém falou sobre música comigo.
Verdadeiramente... O que pretendes fazer? Essa ideia de uma concepção
musical \emph{a priori...}

\emph{- }Pode-se revelar algo que não desejes, - continuou ela - depois
de o fazeres, mas é um caminho, não é? Encontro muita passividade nos
músicos que conheço. A maioria deles quer apenas ser intérprete. No pior
dos sentidos. Guardião da tradição. A criação e a originalidade são
coisas terríveis, não são? É assustador, quando se pretende criar algo e
afinal nos damos conta de que mais não fizemos senão perpetuar uma
tradição...

- Tens razão, em certa medida. Todavia, há algo de inexplicável no
criador que faz com que ele queira ir mais longe, superar-se a si
próprio, em cada composição...é difícil explicar esta sensação.

- Sim, compreendo bem o que dizes. Carregar essa densidade material, o
peso de tudo o que já foi feito, para trás. E deve ser terrivelmente
angustiante... A composição surge numa idade mais avançada, não é?

- Não sei. Acho que a ideia da composição se vai desenvolvendo, mas há
compositores muito jovens. Quem quer criar música, não devia senão
preocupar-se com o que faz. Aliás, como qualquer poeta ou artista, se
quiser ser levado a sério. Não interessa a idade porque isso não pode
ser escolhido, na minha opinião. Responsabilizar-se por inteiro,
adquirir o mais perfeito domínio e apuro, isso sim, escolhemos, mas, de
certo modo, fazer tábua rasa do que se faz à sua volta ou não fazer à
maneira de...

Pela primeira vez, encontrava alguém que se interessava por algo que
era, para ele, de uma importância extrema. Mesmo os colegas dele como o
Martinho, o mais interessado entre todos, não era capaz de levar por
diante uma reflexão sobre aquilo. Interessava-lhe mais o domínio técnico
e o virtuosismo, aquilo em que se sentia mais confiante. Era capaz de
passar horas a falar sobre técnica pianística, mas não se interessava,
como Margarida parecia interessar-se, pela questão da estética musical
ou da criação, o que para ele era apaixonante.

- Na verdade, penso bastante sobre isso. A questão da ideia como origem
a desenvolver, na obra, seja ela qual for, música ou outra qualquer. Não
podemos encarar a criação de uma forma tão esvaziada, banal. Há mais
qualquer coisa, sempre, que nos escapa\ldots{}

- Acho que tens uma concepção romântica da arte. A arte como absoluto,
como ideal, não?

Florimundo sorriu. Como se tivesse sido apanhado.

- Não acredito nela de outra forma. Embora ache importante o domínio
técnico, o virtuosismo. Como poderia o músico evoluir e criar linguagens
se não dominasse a escrita e a tradição, o que já foi feito, de todas as
maneiras e feitios? Pensa em Wagner, em Schönberg. Todos eles se
formaram no classicismo, nas formas tradicionais...

- E inauguraram novos géneros...

- E destruíram-nos. Quando qualquer um deles criou a sua música, já a
estava a destruí-los. É um verdadeiro mistério, irrepetível. Olha
Wagner, por exemplo. Achas que seria possível formar uma escola de
Wagner? Sim\ldots{}epígonos há sempre. Mas escola?

- Acredito em ti, mas sou mais «moderna». Acredito noutros factores, na
contaminação e na sujidade da criação. Essa noção de ideal dá-me
arrepios. - Respondeu Margarida - Apenas te interessa a música? - O
olhar dela tornou-se vagamente absorto. Ela desejava que ele fosse mais
apaixonado, a beijasse. Sentia-se completamente envolvida por ele, pela
sua força e pela sua paixão. Mas queria uma paixão mais física, menos
etérea.

Era a primeira vez que conversavam e se encontravam e ele não parava de
falar de música. Uma certa frustração tomava-a e desejava que ele a
acariciasse, o seu corpo esperava-o. Florimundo era inexperiente,
incapaz de tomar a iniciativa. Usava as palavras para se esconder e
proteger do medo. A ideia de que ele poderia ser mais interessado na
música do que nela própria assustava-a um pouco.

- Oh, claro que não. Sou voraz em relação a tudo.

- Que compuseste já? - Ela afundava o olhar nele, pondo-o à prova e
seduzindo-o. Agarrou numa melena de cabelo que lhe caía pelo rosto e
refez o rabo-de-cavalo. Ele pediu-lhe que deixasse o cabelo solto,
gostava de olhar para ele assim.

Margarida via-o cair nas suas mãos, render-se-lhe. Ele agarrou-lhe na
mão e levou-a aos lábios. O silêncio caiu sobre ambos e ele beijou-a
pela primeira vez. Depois, foi como se o seu corpo, as suas mãos, a sua
língua, soubessem o caminho a seguir.

Ele estava fascinado pela inteligência dela, pelo seu interesse, pela
forma como se demorava nele. Desconhecia tudo sobre o seu modo de
pensar, teria sido incapaz de lhe apreender a argúcia ou de adivinhá-la.
Ela saltava com agilidade de um objecto a outro, brincava com os
pensamentos, de uma forma leve e despretensiosa, enquanto as suas mãos
de dedos finos pareciam dançar, quando gesticulava.

Interrompeu-se, com aquela ideia estranha a impor-se-lhe no espírito.
Ela sempre lhe parecera uma bailarina.

- Alguma vez tiveste aulas de dança? - Ela olhou-o. Talvez tivesse
achado a pergunta despropositada.

- Engraçado...porque o perguntas? - Estava verdadeiramente surpreendida.
Agradavelmente, pois ele tocara-lhe numa fibra sensível. - As tuas mãos,
a tua leveza natural...A forma como te penteias, o corpo ágil e
desembaraçado. A forma como colocas os pés, o teu andar. Tudo o indica.

- E não te enganas. Quando era muito pequena tive de deixar. Uma lesão
grave. Não podia continuar. Foi o maior desgosto que tive em toda a
minha vida. Lembro-me de ter passado dias a chorar, na cama. O meu pai
não parava de me consolar. Chamava-me ``Minha pequena Margot!''.
Lembras-te dela, a Margot Fontaine? Era o meu ídolo. Cresci a vê-la
dançar, naqueles filmes antigos. Aqueles grandes olhos negros, o rosto
de pássaro...foi o meu maior desgosto.

- Ah, mas podes compensar essa perda com a tua voz.

- Não, não é a mesma coisa. Cantar só te permite voar em espírito.
Dançar é o corpo inteiro, compreendes? Quando canto, sinto sempre que há
algo que me prende à terra. Mas danço ainda, no meu quarto... aí
liberto-me de tudo, do peso...transformo-me em matéria espiritualizada,
não saberia explicar-te esta ideia... - ela sorria e o olhar sereno
pousava-lhe sobre as mãos. - Estas mãos tornam-se asas, o mundo adquire
contornos mágicos. Vou-te confessar algo de estranho. Quando tenho
vontade de dançar e esse desejo se torna imperioso, não preciso sequer
de música, parece que ela nasce em mim e me guia os passos, arrasta-me.

- Sim, o peso...na música e na composição, também podes alcançar isso
como se reinventasses a linguagem. Uma linguagem imaterial, dentro das
palavras que usas no dia-a-dia. Acho que te compreendo. Mas, repara, a
voz, apesar de grave, instaura um mundo volátil...

Havia no olhar de Florimundo uma sombra. Sabia exactamente do que ela
falava. Jamais alguém lhe havia falado desse peso ou dessa experiência
íntima, vital. Conhecia esse peso desde sempre. Estar num lado e desejar
estar noutro qualquer. Um sentimento vago de passar a vida inteira a por
detrás de uma janela embaciada. Do outro lado da janela havia o mundo.

O sol recolhia-se. Um fulgor último, ainda. Ao longe, as muralhas do
castelo davam um aspecto eterno à cidade. Tinham estado sempre juntos,
sabia-o. Mesmo que o tempo os tivesse separado. Ele olhou-a,
demoradamente. Para sempre, meu amor, apeteceu-lhe dizer, mas não ousou.
Sabia que isso a iria assustar, a intensidade do seu amor.

Tinha os dias livres, inteiramente disponíveis para trabalhar e para
desenvolver a sua criação. Margarida tinha saído de Portugal. Havia
decidido ir até Paris, onde se encontrava a fazer um curso de canto, com
um professor conhecido. No mínimo, seriam três meses sem a ver. Ela
precisava dessa experiência, de sair do meio e conhecer novas coisas.

Em criança, tinha estado várias vezes em Paris, de passagem, em férias e
com os pais. Agora, tencionava passar uma larga temporada, sem tempo
medido, em crescimento e aperfeiçoamento do trabalho. Poderia conhecer
pessoas que procuravam o mesmo e partilhar essas experiências. Ainda não
sabia o que queria, exactamente, mas também não constituía preocupação,
como filha única e um pai endinheirado.

Com alguma inveja, o rapaz deixara-a partir. Dizia-lhe que também
desejava ir, mas não podia.

- Não podes porquê? - Perguntou-lhe, mas arrependeu-se imediatamente.

Não era preciso pensar muito para perceber a razão.

- Desculpa, claro que não quero intrometer-me na tua privacidade.

Ele sabia que ela não tinha dito por mal. Mas voltava-lhe aquela
amargura antiga, que sentia sempre que via os outros com mais
possibilidades do que ele. Não era directamente contra ela, mas era um
sentimento de revolta calada e antiga que ainda lhe fazia doer.

- Tenho de ficar, procurar um emprego...a minha mãe. - Não continuou,
pois ela fez um sinal de assentimento e pousou-lhe a palma da mão sob os
lábios, em jeito de desculpa.

- Prometo que escrevo todos os dias e te conto tudo... podemos falar por
Skype, em princípio vou ficar alojada num quarto, sozinha, no Quartier
Latin.

Ele fez uma careta.

- Hummm\ldots{}todos os dias? Não fazíamos mais nada. - Sorriu-lhe, com
o rosto franco e já desanuviado. A impressão repentina tinha-se
desvanecido. - Promete-me só que te lembras de mim, de vez em quando.

- Fazes parte de mim. É uma certeza estranha. Sabes, como irmãos ou
amigos antigos...há tantos anos que te conheço.

- Eras capaz de deixar de me amar? - Perguntou-lhe a medo. Para ele,
essa certeza era inquestionável, jamais a pusera em causa.

- Ora, que raio de pergunta!...Mudemos de assunto.

Mais pragmática que ele, aqueles arroubos românticos pareciam-lhe
excessivos.

Ele foi acompanhá-la ao aeroporto. Vinha vazio quando regressou.

A princípio, parecera-lhe uma infinidade de tempo, mas o certo é que a
sua ausência lhe permitia dispor do tempo sem entraves. Escrevia-lhe
muito, todos os dias, longos emails onde se desnudava inteiramente, como
jamais o ousara. Ela permanecia mais reservada, mais silenciosa. Mesmo
através dos emails sentia uma Margarida serena e um pouco distante.

Porém contava-lhe tudo o que a entusiasmava, as novas experiências, as
idas ao Louvre, ao Pompidou, ao Palais de Tokyo, ao Museu de Orsay, as
longas manhãs que aí passava. Devido ao pai ser negociante de arte,
Margarida tinha um conhecimento de arte e de pintura que o intimidavam.
Um gosto requintado que o influenciava. Dizia-lhe que desejava que ele
estivesse perto dela, como o sentia. Então, mostrar-lhe-ia tudo o que
ele ainda não havia visto ao perto. Mandou-lhe obras e catálogos sobre
Egon Schiele. Sabia o quanto ele gostava desses desenhos e pinturas. Em
Paris, dizia-lhe, e ele sonhava com isso, havia as ruas, tudo por ver.
Tudo o que lhe vinha de Margarida parecia-lhe precioso, porque trazia o
seu perfume, a sua aura. Cheirava os livros, na ânsia de descortinar o
perfume que ela usara, no instante em que lhe enviara o livro de forma
algo obsessiva.

As aulas de canto deixavam-lhe as manhãs e, mesmo, dias inteiramente
livres, o que ela, na sua ânsia de estrangeira, aproveitava
inteiramente. Nem que fosse o correr por Saint-Germain, errando pelas
ruelas bonitas e cheias de vida, de pequenas galerias e pelas livrarias,
onde estava horas e saía, depois, carregada de livros. Depois sentava-se
num café a ler.

O que mais amava em Margarida e que a ausência lhe permitia descobrir em
toda a sua riqueza, era, além dessa capacidade poética, a sua
criatividade intensa, um fluxo que, às vezes, se tornava delirante e o
deliciava. Face ao seu espírito, de um rigor quase científico, as
audácias do seu temperamento e da sua imaginação surpreendiam-no. Como
naquela manhã, em que, deambulando pela sala de escultura grega, sentira
que as estátuas se moviam à sua passagem. Os pequenos Apolos de
Praxíteles, sorriam-lhe e moviam-se imperceptivelmente. Mas o
suficiente, garantia-lhe ela, para que lhes topasse o movimento.

Essa história encantara-o. Tal como aquela em que Margarida chorara
diante das estátuas de Rodin. Dizia-lhe que vira a verdade nessas
estátuas, como o \emph{Segredo, }ou na série das mãos, a \emph{Mão de
Deus }e\emph{ a mão do Diabo.} Não imaginava o que fosse ver a verdade,
da forma como ela o contara, mas soube que tinha sido uma experiência
impressionante para ela.

Uma semana mais tarde chegara-lhe um catálogo de Rodin, com as mãos. O
rapaz abriu o livro. Diante dele estavam as mãos, numa fotografia a
preto e branco, que tornava ainda mais belas as esculturas. Um arrepio
de horror percorreu-o ao olhar para o homem enrolado sobre si próprio,
fechado na \emph{mão do Diabo.} A figura perturbou-o de tal forma que
acabou por fechar o livro e guardá-lo longe da sua vista.

Havia uma diferença enorme entre ambos. Margarida vivia uma situação
confortável, que lhe permitia, não apenas estudar, como também viver
despreocupada, relativamente a um futuro próximo. Negociante de arte, o
pai era um mecenas conhecido no meio. E, embora o soubesse, o rapaz não
queria aproximar-se da família por pudor. Temia que o interpretassem
erradamente. Por isso, mantinha uma postura discreta e recusava
delicadamente os convites de Margarida.

Florimundo tinha concorrido ao ensino, num colégio particular.

Preocupava-o, por outro lado, conciliar esse horário com o da
universidade, onde faria uma pós-graduação em composição musical.
Chegara a sonhar com essa pós-graduação em Inglaterra ou na Alemanha, na
Áustria, mas depressa percebeu que isso seria insustentável. Mesmo que
tivesse bolsas, o dinheiro não seria suficiente para viver. Desistira
rapidamente do intento. Além disso, custava-lhe muito deixar a mãe só.

Temia, por outro lado, que os dois horários entrassem em conflito e que
tivesse de deixar as aulas, para poder prosseguir com o curso. Agora,
com as aulas terminadas, o Verão parecia-lhe um paraíso. Sabia o que
queria fazer, mas precisava de muito tempo, para se dedicar totalmente à
obra.

Avançava lentamente, menos pelo perfeccionismo do que pela
inexperiência. Às vezes, parecia-lhe que a ideia se dispersava e receava
perdê-la de uma vez. O segredo estava na relação equilibrada entre o
todo e as partes. Nada ali estava ao acaso, mas toda a linguagem devia
obedecer à simplicidade e a uma unidade que lhe conferissem a
arquitectura sólida. Por vezes, parecia perder-se num exercício
matemático e puramente estilístico. Voltava atrás, apagava o que estava
a mais e recomeçava tudo de novo.

Os dias estavam invulgarmente quentes. O rapaz deixava a janela do
quarto aberta, por onde entrava o vento fraco e os ruídos do exterior,
que o irritavam e lhe impediam a concentração. Juntamente com o calor e
a luz excessiva. Para fugir a essa dispersão, passou a trabalhar de
noite. Jantava, bebia um café forte, fechava-se no quarto e trabalhava
até ao raiar do dia. Muitas vezes apenas interrompia quando o sol
começava a nascer e pequenos farrapos de luz lhe entravam pelo quarto
adentro, anunciando a manhã.

Tinha uma ideia mais ou menos clara, ainda não conduzida ao pormenor,
sobre a linguagem que gostaria de inscrever no panorama contemporâneo,
dominado pelo minimalismo. Sentia que o animismo e a vitalidade da
música estavam intimamente relacionados com o universo interior e com os
obscuros desígnios da criação. Era preciso mergulhar nesse universo para
alcançar não sabia bem o quê, mas que pressentia.

Sabia que era impossível criar da mesma maneira que um artesão,
construindo lentamente, recortando, juntando peça a peça, somando. Havia
um universo preexistente, uma intuição que se antecipava e que era
simultaneamente aniquiladora e geradora.

Uma dimensão profundamente trágica escondia-se por detrás desse labor,
tão aparentemente artesanal, que colocava inteiramente ao serviço dessa
inspiração escura e insidiosa, a qual o assaltava como se fosse uma
doença da alma. Tão depressa estava na mais desertificada zona, ouvindo
música e lendo, procurando reencontrar o filão que o alimentava, como
era assaltado, sobretudo durante a noite, pelo furor criador. Era nessa
altura que se atirava ao trabalho por inteiro, imergindo num espaço e
num tempo sem horas nem exterioridade.

Começou a compor estritamente de noite, dormindo durante o dia e
impondo-se estranhos hábitos que não tivera antes. Não podia distrair-se
nem estar refém de pequenas distracções. Na verdade, a obra convertia-se
na referência temporal e espacial, de onde tudo partia, obedecendo
apenas ao ritmo de trabalho que lhe era ditado.

Por vezes, Clara batia à porta, antes de se ir deitar, acariciava-o,
como sempre o fizera, de modo terno.

Trazia-lhe bolachas e leite quente. Olhava para as partituras e lia-as
durante algum tempo, procurando perceber o que se lhe tornava cada vez
mais estranho e que a afastava do Florimundo que ela conhecera. Ela
havia-o acompanhado ao longo da sua evolução, por vezes sugerindo mesmo
pequenas alterações, mas aquela linguagem parecia-lhe incompreensível.
Todavia, receava desencorajá-lo, vendo-o trabalhar tão arduamente.
Começou a pensar que não seria suficientemente inteligente para
compreender os novos caminhos que ele abria diante de si.

Acreditava que o filho seria um grande compositor, mesmo que não
compreendesse a sua música. Tinha alguns conhecimentos de música, pelo
menos os suficientes para compreender que muitos compositores jamais
haviam sido entendidos, alguns nem sequer interpretados em vida. Aliás,
como acontecia em geral com os grandes artistas, escritores, poetas. Os
génios. A ideia de que tinha dificuldade em compreendê-lo, algo que a
princípio a assustara, confirmou-se cada vez mais como uma certeza do
seu valor.

Antes de sair do quarto, naquela noite, Clara deteve-se, à porta, a
olhá-lo. Havia um pedaço dele que lhe escapava. A reserva tornara-o
distante. Algo que o afastava do mundo real. Tal como havia acontecido,
pouco a pouco, com o pai.

A sua intuição captava o novo traço do seu carácter, o que, de certo
modo, a trazia ainda mais apreensiva. Uma melancolia indefinível, como
em Gabriel, que se acentuara com os anos. Uma devoção inteira ao mundo
da arte. Sabia e conhecia na pele os devastadores efeitos daquela
procura insaciável.

Aliás, nada era normal, em Florimundo. Nem a maturidade nem a
inteligência. Não raras vezes, um arrepio de assombro, percorria-a.
Havia algo que ultrapassava a sua compreensão. Ouvira falar de certas
crianças que possuem extraordinários poderes. O coração dizia-lhe que
havia algo em Florimundo, um dom, mas que ela não conhecia.

Apenas isso, pensava ela, para se convencer a si própria.

Clara olhou-se ao espelho. Deu um jeito ao vestido, já bastante usado.
Aquele vestido dera-lhe o marido, no tempo em que viviam folgadamente.
Desde aí, ela emagrecera um bom bocado. Nunca mais recuperara. Era uma
bonita mulher. Florimundo herdara-lhe os olhos claros e tranquilos.

Gabriel mergulhara neles havia 30 anos. Ela deixara que ele se perdesse
no seu olhar. Mais nenhum homem mergulhara assim no seu olhar, no seu
corpo. Trazia os ombros caídos, devido à magreza. Os braços demasiado
finos. Mas as pernas mantinham-se bem torneadas, a pele clara e bonita.
Não se pintava, não usava jóias. Desfizera-se de tudo o que tinha,
coisas de valor que Gabriel lhe havia oferecido. Sabia que jamais
voltaria a usá-las. Quando teve de pagar as lições de Florimundo, foi-se
finalmente o colar de pérolas, herdado da avó. Depois, além do emprego
numa loja de artigos de decoração, tivera de jogar mãos a tudo, fazer
bolos para fora, rissóis e tudo o que lhe era encomendado. Tinha a
facilidade de cozinhar bem, pelo que nunca lhe faltava trabalho. Chegava
a casa já cansada, preparava o jantar e punha-se, incansável, a fazer as
coisas para o dia seguinte. Muitas vezes, ficava até de madrugada, com
um olhar sonâmbulo, meio perdida no seu cansaço antigo. A vida nunca
mais fora igual.

Ficara só com o filho depois da morte de Gabriel. Fizera desse filho a
sua única razão de viver. Dera-lhe tudo. Só não conseguira dar-lhe a
alegria de viver, não soubera transmiti-la. Em seu lugar, oferecera-lhe
a solidão. Uma capacidade de trabalho e uma dedicação exemplares. Sabia
que, em parte, o isolamento do rapaz se devia à sua própria incapacidade
de estabelecer relações sociais e à sua tristeza, desde a morte do pai,
deixando-o desamparado. Mas também o sabia capaz de grandes tarefas.

De um dia para o outro ela perdera a sua alegria. Aprendera que ela não
lhe servia para nada. Não a ensinara a sobreviver nem a superar os
dramas da sua existência pessoal.

Clara lembrava-se, com alguma nostalgia, parecendo-lhe que nada daquilo
parecia ter-lhe sucedido, como gostava de sair e dançar, como apreciava
a companhia ruidosa e barulhenta das amigas, a conversa de Gabriel.
Clara havia sido, para grande divertimento seu, uma sedutora
incorrigível. Com a sua beleza pusera montes de rapazes com o coração em
sobressalto, para depois lhes responder, com um sorriso irrepreensível,
que não estava interessada.

Gabriel fora o único homem a vergá-la. Para começar, jamais se mostrara
interessado, coleccionando amigas e mulheres bonitas. Para Clara, todas
essas circunstâncias mais não faziam do que espicaçar-lhe a curiosidade.
Gabriel era alto, bem-feito, com um olhar melancólico. Quando se
apaixonaram, foi uma história fulminante. Clara tinha 20 anos, Gabriel
28. Nada havia a fazer. Por mais que os seus pais lhe dissessem que ele
era muito mais velho que ela, Clara estava-se nas tintas. Com
arrogância, saiu porta fora, para casar com aquele que dizia ser o homem
da sua vida.

Gabriel fizera um percurso brilhante no liceu, entrara na Faculdade, mas
resolveu dar o salto e, com o primeiro ano da faculdade mal acabado,
viajou até Paris. Ainda pensou em estudar na Sorbonne, mas o dinheiro
mal dava para comer. Decidiu-se a levar uma vida errante que lhe
permitia ir vivendo sem problemas, trabalhando em cafés, hotéis, lendo
tudo o que apanhava pela frente, fazendo uma formação autodidacta, que
lhe abria o espírito como um vento frio.

O rosto triste, de olhos negros e barba a condizer, o corpo alto,
esguio, os traços de uma beleza mediterrânica, faziam as delícias das
francesas. Dois dedos de conversa, uma garrafa de vinho e alguma poesia,
num francês ainda fraco, mas que foi aperfeiçoando rapidamente, eram os
ingredientes necessários para que as belas caíssem nos seus braços. Uma
boa vida. Que mais poderia desejar? A miséria cultural em que se vivia,
os limites estreitos dentro dos quais se movia a intelectualidade
portuguesa, tudo isso o afastava.

Quando a vida de Paris se tornara monótona, foi para Londres. Vivera num
quarto, na casa de um amigo da família, português, e que trabalhava na
embaixada de Portugal. Como ele se tratava bem, de certa forma Gabriel
também gozava de um estatuto privilegiado. Trabalhou durante algum
tempo, num escritório, um emprego conseguido por esse amigo, o que lhe
permitiu frequentar aulas de literatura inglesa.

Aos fins-de-semana, saía cedo, atravessava a névoa de Londres e
enfiava-se na National Gallery, a Tate, passeava pelos jardins de
Londres, quando o tempo o permitia e aos fins-de-semana. Como qualquer
autodidacta, era um eclético, deixava-se fascinar pela a informalidade
de Bram Van Velde, como as piscinas de David Hockney ou a ascese em
Rotkho; demorava-se diante da desmedida beleza de Caravaggio, do
luminoso Tiepolo, onde o infinito se lhe apresentava sob a forma de céus
azuis, repletos de anjos, das penumbras de Rembrandt. Mas o que mais
gostava era de arte antiga, as estátuas babilónicas, egípcias e os
deuses de um mundo bárbaro e estranho, meio-animais, com cabeças
bizarras.

Em Londres leu Henry James, Chesterton, Coleridge, Swedenborg. Foi
sensível ao mundo táctil e quotidiano de Larkin mas isso fazia-lhe
ressaltar mais a feerie em Walter de La Mare. Mais do que nunca, amou os
anjos, os ícones de um mundo que esmaecia, tão brilhante e tão próximo
da sua sensibilidade poética. Descobriu Conrad, Thomas de Quincey e
resolveu, ingenuamente, tornar-se escritor. Descobriu o poder imenso das
alegorias e obsessões de William Blake, de Rossetti, tal como já havia
amado Moreau em Paris. Um estranho fascínio por mulheres de rosto frio e
ausente, pálidas como figuras de cera. Dir-se-iam fantasmas, errando por
jardins de irrespirável beleza. Amava tudo que tendia a desaparecer,
esses seres de passagem.

Quando começou a escrever, as suas histórias eram delirantes e
desconexas, onde nada parecia fazer sentido e as peças precisassem de
encaixar. Uma espécie de mistura desorganizada do saber apressado e
avulso que tinha. Pouco a pouco, a arquitectura frágil começou a dar
lugar a uma estrutura mais minuciosa, onde as peças se ajustavam e os
personagens se tornavam vivos sob a sua caneta, ameaçadores e
radicalmente estranhos.

O frio londrino, as livrarias e as ruas de uma magia poderosa, a igreja
de St Martin-in-the-Fields, onde ouvia belos concertos, ao final da
tarde, com o seu órgão fabuloso e acústica impressionante,
emocionavam-no profundamente, levando-o a uma solidão ainda mais
acerada, pois não tinha com quem partilhar as suas experiência. Aqueles
que trabalhavam consigo e com quem passava os seus dias não tinham
interesse. Gostavam mais de ver o Manchester a jogar.

Por fim conheceu Michael e Paul, dois colegas de literatura. Que eram
também leitores vorazes e com quem podia conversar horas a fio no pub
mais perto da faculdade. Paul também queria ser escritor, pelo que a
proximidade entre ambos não tardou a acontecer.

Paul ensinou-o a amar aquela Londres que se pressentia mais como um
lugar da alma do que um lugar físico. Com ele, percorrera uma Londres
desconhecida, de zonas industriais que ainda traziam a marca da miséria
dickensiana. Já não tão pobre, mas a fazer lembrar a pobreza de Dickens,
uma Londres bem distante de Charing Cross ou da fina Regent Street, de
todas as zonas chiques. Conversavam durante longas horas, à beira do
Tamisa.

Com Gabriel, Paul conheceu a poesia de Pessoa e de Camilo Pessanha, de
Cesário Verde, a cartografia diversa da melancolia portuguesa. Liam-se
mutuamente. Embora a orientação de Paul fosse inteiramente diferente.

Paul queria escrever o mundo tal como ele era, num retrato sóbrio sem
qualquer maquilhagem: o lixo dos caixotes, o olhar cansado dos
transeuntes, as emoções despojadas, a ausência de metafísica. Encontrava
na poesia contemporânea americana a sua grande inspiração e não
apreciava particularmente o que Gabriel fazia, apesar de o achar
extremamente talentoso e brilhante.

Um dia, chegou um telefonema de Paul. Gabriel sentou-se e chorou. Paul
ia partir para os Estados Unidos, onde tencionava continuar a estudar e
escrever. Gabriel não podia acompanhá-lo e soube, então, que nunca mais
o veria. Habituara-se a considerá-lo como um irmão. De certa forma,
haviam crescido juntos naqueles anos. Poderiam escrever-se, mas o
afastamento daquele que era como um irmão doía-lhe profundamente.

Muitos anos mais tarde, Paul tornar-se-ia um escritor bem conhecido.
Algo que Gabriel sabia que aconteceria. De Michael, todavia, nunca mais
ouvira falar. Paul dissera-lhe que ele casara e fora viver para África.
«Porquê África?», perguntara na altura a Paul. Mas ele fora lacónico na
resposta: «Talvez por aventura. It's funny!».

Passados dois anos, a morte do pai surpreendeu-o. Recebeu um telegrama
da tia. Recebeu-o, com as mãos trémulas. Sentou-se na cama e abriu-o.
Ficou meio aparvalhado. Era assim que o destino se cumpria e lhe dizia
que estava na hora do regresso. Meteu-se no avião para Portugal.

Após meia-dúzia de anos, vividos em grande turbulência e riqueza
intelectuais, ébrio de pintura e de literatura, de cultura inglesa e
francesa, eis Gabriel regressado à pacatez do seu país, mais atónito do
que nunca.

A família de Gabriel, endinheirada, vinha da burguesia industrial do
Norte. Viviam no Porto, onde tinham uma casa numa das artérias mais
movimentadas. Desde pequeno, fora habituado à convivência social. Dado
aos prazeres da vida, o seu pai tinha estoirado o dinheiro à família,
com o jogo e as mulheres.

Com a sua morte, Gabriel descobrira que a situação económica não era
favorável, agora. Havia muitos anos que a mãe vivia uma situação de
comodismo. Não mantinha praticamente relações com o marido, fazendo a
sua vida independente. Continuava a cultivar as suas relações sociais,
indiferente ao incómodo que a sua situação lhe podia provocar. Pelos
filhos, submetera-se à situação, virando-se para a sua educação, já que
do lado paterno eles não podiam contar com auxílio ou orientação.

Por fim, aquela dissipação tivera o esperado desfecho: A ruína da
família e o desamparo da mãe. Gabriel sabia vagamente da situação, mas o
facto de se ter afastado da família fazia com que desconhecesse os
pormenores de que a mesma se revestia. Ele era o mais jovem dos filhos.
Via-se a braços com uma situação ingrata. Enquanto os outros já se
tinham formado e tinham a sua vida independente, já casados e com
filhos, com bons empregos, Gabriel não tinha profissão a que se
agarrasse, nem sequer tinha acabado o curso e era obrigado a
desenvencilhar-se. Pela primeira vez, o jovem sentia-se responsável pela
sua vida.

Enterrado o pai, pelo qual sentia apenas uma espécie de piedade que nada
tinha a ver com amor, passaram à resolução de problemas práticos. A casa
do Porto tivera de ser vendida, bem como as propriedades, para pagar as
dívidas contraídas. A mãe, agora só, mas talvez mais feliz, iria viver
para casa da irmã mais velha.

Gabriel tinha de procurar rapidamente um meio de subsistência. Pensou em
voltar a Paris, onde tinha uma vida leve e era mais fácil voltar, mas
isso parecia-lhe já um sonho distante. Chegou a Lisboa, onde já vivera,
no tempo em que andara a estudar. Um tio tinha aí uma empresa ligada a
exportações, onde poderia trabalhar.

Começara por viver na casa do tio, mas o pudor levara-o a procurar casa,
já que o emprego era muito bem pago. Encontrara uma casa fantástica na
Rua da Misericórdia, junto ao Bairro Alto, por um preço módico e
compatível com o ordenado. Não era propriamente o melhor dos sítios para
viver, de acordo com o gosto burguês vigente, mas o rapaz habituara-se à
vida boémia de Paris. O lugar parecia-lhe óptimo. Em combinação com a
senhoria, ficou estabelecido que ela faria a limpeza, umas duas vezes
por semana. Como almoçava sempre fora, perto do emprego, só lhe era
necessário cozinhar à noite.

À noite, juntava-se à fauna estranha, que errava por ali, uma mistura
boémia de prostitutas, alcoólicos, artistas, escritores e drogados. Mas
o sabor da vida parisiense e londrina parecia-lhe muito superior.
Conhecera boémios interessantes e cultos, poetas malditos, talentosos
artistas portugueses, pintores e escritores, que se encontravam
exilados, ao passo que os boémios que conhecia agora lhe pareciam menos
interessantes e mais provincianos. Liam pouco, interessavam-se pouco ou
quase nada por cultura. Era quase e apenas uma irmandade alcoólica. Não
encontrava quem lhe substituísse a intensidade da companhia de Paul e de
Michael e acabou por desiludir-se rapidamente, por tanta sensaboria,
fechando-se em casa aos fins-de-semana, onde passava os dias a ler e a
ouvir música.

O cinema que havia era absolutamente desinteressante, o teatro, à
comparação com o teatro de Paris e de Londres, não o satisfazia. Na
verdade, notava ele, apesar do seu provincianismo, a vida no
estrangeiro, durante aqueles anos, moldara-o e obrigara-o a crescer.

Numa manhã de sábado, ele estava sentado diante do seu café, a escrever.
Um bando de raparigas, bonitas e bem-vestidas, invadiu o café. Clara
estava entre elas. Uma beleza. Habituado à beleza feminina e sabendo que
não deve olhar-se com insistência, pois elas se convencem imediatamente
do seu poder irredutível sobre os homens, observou-a de soslaio. Ela
pousou imediatamente o seu olhar irresistível e doce sobre ele. Achou-o
bonito. Fora do vulgar, com um ar de poeta romântico. Apesar da
atracção, ele desviou o olhar e mergulhou-o no jornal, numa indiferença
estudada. Sempre que ela não estava a olhar ou estava de lado, ele
aproveitava para lhe estudar a pose, a curva delicada do pescoço, os
seios e as pernas elegantes. Tinha uma boca maravilhosa, rosada e
sinuosa, nem grande nem pequena, insinuante, entreaberta e que deixava
ver os dentes.

Gabriel não fez qualquer esforço para se aproximar. O tempo resolveria
essa aproximação. Havia no grupo raparigas que pareciam interessadas
nele. E mais atrevidas que Clara. Esta era mais do género de fazer
vergar um homem à sua vontade. Se mostrasse o seu interesse por ela,
sabia que rapidamente seria rejeitado. Ela estudava-o com interesse.
Sabia-o. Ele trazia propositadamente livros que sabia que ela observava
com o seu ar displicente. Esperava o dia em que ela se aproximasse dele,
mostrando-se, pelo menos, interessada nos livros. Ora, isso não
acontecia.

Pelo contrário, foi a atrevida Lina quem se aproximou. Clara bem via que
Lina não tirava os olhos de cima de Gabriel. Ficava toda excitada,
quando ele estava por perto. Punha-se a falar alto, procurando
chamar-lhe a atenção, ria-se de tal maneira que se tornava irritante.
Claro que Gabriel notara essa alteração, embora se mantivesse
impassível. Não lhe apetecia nada conhecer aquela rapariga, demasiado
infantil, faladora e risonha para o seu gosto. Gostava de mulheres
discretas e silenciosas. Mas percebeu que Lina seria o caminho até
Clara.

Alguns dias após o primeiro encontro, estava Lina a espiá-lo pelo canto
do olho, descrevendo as peripécias da soirée anterior, quando ele lhe
devolveu o olhar, de modo firme e afável. A partir desse momento, Lina
não mais o largou. Ao passar por ele, a caminho da casa-de-banho,
arranjou forma de o tocar. Ele sorriu, enquanto ela lhe pedia desculpa,
demasiado solícita. Gabriel não achava Lina bonita nem feia. Olhos
negros e grandes, baixa, bem-feita, de corpo roliço e firme. Um cabelo
bonito, longo e encaracolado, escuro.

Ao passar por ele, deixara um odor sensual. Perturbador, mesmo.
Sentiu-se inquieto. Havia bastante tempo que não tocava numa mulher.
Lina excitara-o. E a outra não lhe ligava nenhuma. Resolveu tirar o
melhor partido, mas manter uma distância que não deitasse tudo a perder.

Quando ela voltou, ele tocou-lhe e agarrou-a pela mão, perguntando-lhe
se não queriam juntar-se a ele. Ou vice-versa. Meio envergonhada, Lina
olhou, com um ar suplicante, para o resto das raparigas. Elas
fizeram-lhe sinal para que ele se lhes juntasse. Nesse instante, Clara
sentiu-se desiludida com ele. Primeiro, porque ostentara a sua
fragilidade perante as mulheres, segundo, porque se sentira traída pela
sua atenção para com Lina.

Clara achava-a vulgar e pouco inteligente, embora ela fosse a sua melhor
amiga. Confiava inteiramente na rapariga e sabia-a generosa. Mas
desagradava-lhe a forma como se atirava aos rapazes, a estratégia de
sedução que usava, demasiado óbvia. Mas sempre com resultados visíveis.
Lina era daquelas mulheres que sabia que, para agradar aos homens, não
era preciso muito. A inteligência assustava-os. Eles gostavam de dominar
as raparigas, de ser adulados por elas, o que ela cumpria na perfeição.
Gabriel também caía nas armadilhas de Lina. Isso irritava Clara. Embora
Lina lhe dissesse amiúde que, se ela não tinha namorado, era justamente
por se fazer difícil e cara.

A partir desse dia, Gabriel passou a estar presente na mesa das
raparigas. Contou a sua história, em poucas palavras, exibia o seu
charme, a sua inteligência com um despudor que irritava Clara. Mas não
podia deixar de admitir que ele a atraía de uma forma animal. Sobretudo
quando fixava nela aqueles olhos melancólicos e escuros e lhe sorria.
Falava-lhe numa voz terna.

Das raparigas, a mais interessada no que ele escrevia era Clara.
Olhava-o com aquela admiração genuína, perguntando-lhe sempre o que ele
andava a ler, interessando-se vivamente pelas histórias dele, que lhe
abriam um horizonte inteiramente desconhecido. Jamais ouvira falar dos
autores de que ele lhe falava. Levada pelo timbre sensual da sua voz,
pelo entusiasmo e dedicação que ele dedicava à literatura, pela
seriedade com que se entregava à tarefa diária e disciplinada, pela
facilidade e colorido da sua escrita, Clara podia ficar horas a ouvi-lo,
sem, no entanto, deixar transparecer os seus sentimentos. Temia que ele
se cansasse dela, por isso fazia o possível por não se deixar arrastar
para situações da qual sairia magoada.

A ele, essa indiferença estudada parecia-lhe estranha, começando a
pensar que, de facto, ela não o desejava, o que o deixava angustiado.
Procurava impressioná-la de todas as formas possíveis e ela devolvia-lhe
a sua frieza mansa, com a qual se tornava difícil de conviver, à medida
que o tempo passava.

Um dia, ele levou-a casa. Sentiu que ela se retraía, diante dele, o que
indiciava medo de si própria. Percebeu finalmente que a sua frieza não
passava de uma defesa perante os seus próprios sentimentos. Ela tinha
terror de dizer algo que a denunciasse, de olhá-lo de forma que se
revelasse suspeita. Ele achava que estava na hora de acabar com o jogo
das escondidas.

Passaram pelo jardim e as mãos tocaram-se. Ela sacudiu a mão, como se
tivesse sido picada por um insecto. Ele agarrou-lhe subitamente os
dedos, puxando-a para si, mas sem a forçar. Sentiu que ela desejava a
aproximação.

- Gostava de fazer amor contigo. - Disse-lhe sem rodeios. - Tinha a
certeza de que ela gostava dele. E sabia que aquela abordagem haveria de
a surpreender, elogiando-a ao mesmo tempo. Sentia que a clássica
fórmula, de começar por qualquer coisa como ``amo-te'' não seria bem
acolhida.

Ela sorriu, tímida. Nunca tinha sido abordada daquela forma. Mas na boca
de Gabriel, aquelas palavras tinham uma beleza especial. Era a pura
expressão do seu desejo, sem qualquer subterfúgio ou segundas intenções.
E não o comprometia, deixava-o livre, revelava-o como um animal
inquieto, que não queria qualquer espécie de compromisso. Pensou que se
as mesmas palavras tivessem sido ditas por qualquer outro homem, lhe
teriam parecido ofensivas. Na boca de Gabriel, elas desenhavam a mais
plena liberdade do amor. Beijou-o apaixonadamente, em resposta, sem
esperar que fosse ele a tomar a iniciativa.

Horas mais tarde, as costas macias de Clara permaneciam entre a luz da
madrugada e o branco dos lençóis. Gabriel olhava-a, adormecida. Tinha a
pele muito clara, os olhos fechados e o rosto quase infantil, mergulhado
num sonho que lhe fugia. Queria continuar a acariciar-lhe a pele, onde
se viam alguns pelos louros e finos, mas receava acordá-la, retirando-a
do seu encantamento. O cabelo castanho claro, cintilante nas suas
madeixas louras naturais, caía-lhe pelo rosto, deixando à vista os
rosados lábios, que pareciam sorrir.

Lembrava-se do ar tímido com que ela entrara debaixo dos lençóis. Em que
ele a tacteara com suavidade, na escuridão, pois ela pedira-lhe para
apagar a luz. Depois envolvera-o com os cabelos e as pernas firmes e
elegantes, apertando-o contra si. Murmuraram coisas irrepetíveis e
admiraram-se sob o luar, a única luz consentida e que entrava por uma
fresta da janela.

Era bom tê-la deitada a seu lado. Havia tanto tempo que esperara que
aquilo acontecesse. Durante muitos meses, o desejo deixava-o aturdido,
quase sem forças, um desejo feroz acompanhado de ternura. Não queria
reconhecer ainda, receava dizê-lo, mas amava-a que nem um perdido.
Gostava de tudo. Do seu rosto de ninfeta, da sua elegância, das suas
curvas que o deixavam excitado, dos seus lábios que lhe desciam pelo
peito abaixo e procuravam o seu sexo. Deitou-se sobre o seu corpo e
beijou-lhe as costas. Cheirou-lhe o cabelo, a pele, para poder reter o
seu cheiro, enquanto ela não estivesse ali, presente. O odor do seu
corpo na cama dele seria a sua presença. A sua salvação. Agora que o
amor o atingira como um petardo já não podia deixá-la. Nem sabia como
poderia viver longe dela. Um único gesto, um pedido, tinham bastado para
o perder.

Levantou-se, pegou num bloco de folhas de desenho e pôs-se a desenhar o
corpo nu de Clara. Destapou-a um pouco, para lhe ver a púbis, esse
jardim de delícias onde se perdia desesperadamente. E novamente o corpo
estremecia, o desejo retornava, insaciável.

O telefone tocou. Clara deslizou, no seu passo lento. Atendeu. Era uma
chamada para Florimundo. Este atendeu, com o coração sobressaltado.

Margarida não voltaria senão daqui a alguns meses. Era o final de
agosto.

Clara ficou a observá-lo. Não sabia o que pensar. Percebeu que ele
falava com um professor. Mas não sabia do que se tratava. O rapaz
acenava-lhe afirmativamente. Depois desligou. Disse que precisava de
falar-lhe.

- É o lugar de professor? - Perguntou-lhe com ansiedade.

- Bem, ainda não me disseram nada do colégio. Mas fui aprovado para o
curso de pós-graduação. Resta-me agora pedir uma bolsa. Se não me derem
o lugar, não posso arriscar. Foi justamente o meu melhor professor a
ligar-me para me incentivar a largar o lugar da mão e a pedir uma bolsa.
Ele dar-me-á o parecer e será mais fácil. Creio eu... Tenho de arranjar
um projecto...

- Tens andado a trabalhar dia e noite e não tens um projecto? Porque não
concorres com a peça que tens trabalhado? Podes fazer dela algo de mais
complexo.

A inteligência da mãe surpreendeu-o. Nunca pensara converter o trabalho
que estava a criar num projecto para concorrer. De facto, a peça estava
bastante desenvolvida e precisava apenas de justificar os pressupostos,
não apenas técnicos, como teóricos. A razão pela qual não o havia
pensado antes devia-se ao facto de ter sido escrita a pensar em alguém,
inteiramente dedicada à «sua» voz. Ainda não se confrontara com o
simples pensamento de a tornar pública nem se lembrara que podia usar o
seu trabalho de forma mais pragmática. Mas a mãe tinha absoluta razão.

A justificação teórica da sonata foi das coisas mais difíceis que fizera
na sua vida. Jamais fora obrigado a teorizar sobre música,
verdadeiramente, a apresentar uma argumentação que fundamentasse o
trabalho, nessa angústia de definir os pressupostos teóricos.

Esperavam-no duas noites de intenso trabalho, pois o projecto devia ser
apresentado daí a dois dias apenas. Essa tinha sido também a razão do
telefonema.

Estava tão embrenhado na escrita que não se apercebeu da rajada de frio
que atravessara o quarto.

- Ocupado, hein?!

A voz assustou violentamente Florimundo, que colocou a mão sobre o
coração e suspirou de alívio ao perceber-lhe a presença.

Ele estava pendurado no armário do quarto. Depois sentou-se no cadeirão.
Parecia divertido, embora não conseguisse ver-lhe o rosto.

- Apareces-me assim, de imprevisto... - Reclamou o rapaz. -- E na pior
altura, estou doido para acabar isto!

- Posso ajudar-te. - Respondeu prontamente o cómico sujeito. - Aliás,
tem sido sempre essa a minha preocupação, caso não o compreendas.

Saiu da sua garganta um riso casquinado.

- Onde pensas que vais buscar essa tua tremenda energia para o trabalho?

- Devo-te...

- Sim, mais ao tempo...

- Ah, pois! Não sabia.

- E a tua Margaridinha? Tem-te escrito, o anjinho? - O tom insinuante
irritou Florimundo, que achava que ele não tinha nada a ver com a sua
vida privada. Além disso, irritava-o o modo como se referia ao seu amor.

- Bem, vejo um olhar fulminante. Bom, para aí não te vejo tão decidido,
és um mole, rapaz... ai os ensinamentos que eu dava à miúda, aquelas
nádegas a precisar de cavalgar, aquele ventrezinho tão\ldots{}

- Não, só entendo que são aspectos da minha vida privada. Não vejo que
tenhas de falar sobre isso. Não gosto que te refiras assim à miúda -- Os
olhos chispavam, de tão furioso que ficou.

- Ah! Então nunca reparaste nos seios dela por baixo da
camisola...usa-as tão justinhas!

- Respeita os sentimentos dos outros. Só te fica bem. -- Tinha
dificuldade em manter o tom baixo da voz, com medo de acordar a mãe.

- Não fui feito para respeitar absolutamente nada. Mas apenas para pôr
em ebulição, espicaçar-te...A vida é sexo, carne, falta-te corpo e
prazer, precisas disso para te libertares. E se não te pões a pau, a tua
querida diz-te um adeus, um dia destes! - Continuou ele - Há por aí uma
série de homens saudáveis, apreciadores de umas boas curvas. Desses
cheios de músculos e que as agarram e as põem de gatas e\ldots{}

Florimundo mordeu os lábios. De raiva. Interrompeu-o, pois sabia onde ia
dar a conversa.

- Podes calar-te?? Penso noutras coisas mais importantes.

- Ora, mais importante que uma boa foda? Na tua idade? Ohhh\ldots{} --
pôs-se muito divertido com a situação, a suspirar.

Depois respirou fundo e retomou, enquanto um odor acre invadia o quarto:

- Uma boa altura para te questionares. O que queres, na verdade, fazer
da tua música? Vislumbro o novo Bach do novo século...do novo milénio?!

- Estou um bocadinho mais pessimista. Bolas, que cheiro é este?

- Matéria, matéria\ldots{}

- Algo irá aparecer no próximo milénio, o que está feito, feito está...
-- Retomou a conversa - Isto obriga-me a reflectir sobre o que tenho
feito até aqui, o significado do que gostaria de fazer. Já não digo do
que tenciono fazer. Pareceria um pretensioso. Morro de medo, só de
imaginar.

- Creio que não és um teórico... - Acrescentou o mafarrico.

- Não, não sou. Sinto que avanço à medida que vou fazendo. Prática e
teoria entrelaçam-se. Tudo isto me parece bastante enigmático...

- Claro! Como para qualquer criador. Acreditas que eles sabem
exactamente o que pretendem criar?

- Não sei, nunca fui criador.

- Sê-lo-ás...ainda que não saibas o que criarás. Acima do bem e do mal,
da vida e da morte, asseguro-te. Bom, fartam-me estas conversas
balofas...Não te estafes, dá apenas o que possas... A mim esperam-me
outras obrigações, umas gajas a quem prometi umas voltas...não me
largam, as cabras!

Riu-se e desapareceu, deixando no ar um cheiro acre. O frio também sumiu
com ele. A luz voltou.

Durante toda a noite, as frases correram-lhe a um ritmo selvagem. Nesse
projecto explicava exactamente todas as suas ideias, de forma minuciosa.
Florimundo soubera utilizar as ideias que aprendera ao longo da sua
vida, extraindo delas o absolutamente essencial, subvertendo-as e
aplicando-as, em concreto. Poder-se-ia dizer que a bibliografia
utilizada havia sido escassa, mas o admirável ensaio de Pierre
Jean-Jouve sobre o \emph{D. Giovanni} de Mozart, revelara-se como a peça
fundamental e que desencadeara toda a produção escrita.

No dia a seguir, ao fim da tarde, o rapaz tinha passado a limpo tudo o
que havia produzido. Telefonou ao professor, saiu a correr e passou pela
casa dele para lhe levar o projecto, pedindo-lhe a sua opinião.

O professor ficou boquiaberto. Com a maturidade inegável do projecto, a
sua qualidade. No dia a seguir, telefonara-lhe a dizer que o entregasse,
tal e qual como o havia escrito. Juntamente com o parecer que fora
elaborado. Agora restava esperar alguns meses a resposta do Ministério.

\section{FÉRIAS DE NATAL}

Afinal, Margarida viria a casa durante as férias de Natal. Estava
indecisa e dividida se deveria ou não ficar em Paris. As coisas
corriam-lhe bem. Tinha sido convidada para um recital, integrando o coro
do Conservatório. A oportunidade era esta. A de começar. Claro que
aceitara o convite, mas o que duvidava era se continuaria em Paris. Ela
sentia falta da família e dos amigos, em especial doía-lhe a ausência do
terno Florimundo.

Escrevera-lhe a dizer-lhe em que avião chegaria. Desejava que ele fosse
buscá-la. Excitado, o rapaz nem pregou olho nessa noite. Doía-lhe a
saudade do cheiro dela, o contacto das suas mãos. Amava tudo nela,
incluindo o seu feitio caprichoso e mimado, uma certa arrogância snob
que demonstrava, involuntariamente, em certas ocasiões.

Ela chegou, finalmente. Vinha mais magra, trazia uma saia de tweed
preta, uma camisola justa de lã vermelha, um casaco cinza e curto, uns
ténis vermelhos. O cabelo atado no eterno rabo-de-cavalo. Mais pálida,
seria do cansaço, perguntava-se. Abraçaram-se e ele afastou-lhe os
cabelos que teimavam em cair-lhe para o rosto. Beijou-a demoradamente.
Os pais esperavam-na, atrás deles, sem que ele se tivesse apercebido.
Tinham acabado de chegar. Ela mostrou-se tímida e reservada, como era o
seu hábito. Tinha tanta coisa para contar que não sabia por onde
começar.

Saíram todos no carro do pai. Ele sem saber o que havia de dizer. A mãe
era uma senhora faladora e exuberante, em tudo diferente da filha. Tinha
o cabelo de um louro falso, os lábios pintados. Era uma mulher bonita,
mas nela tudo era artificial, mesmo a forma como falava. Margarida
parecia-se mais com o pai. Homem de inteligência visível, discreto e
muito afectuoso. Abraçava a sua menina, brincava com ela, dizendo-lhe:
«Temi que ficasses por lá!».

Florimundo passou todo o dia com ela. Ela convidou-o a jantar lá em
casa. A mãe preparara uma festa em honra dela, convidara os seus amigos
mais chegados. Florimundo jamais se confrontara com tal situação. Sempre
que estivera com ela, tinha sido a sós. Constatava, agora, que não sabia
nada sobre esse lado de Margarida. A sua roda de amigos, as ligações que
ela possuía. Às vezes, via-a com colegas, mas nem sequer sabia se alguns
desses colegas, que conhecia vagamente seriam seus amigos. Ela também
falava pouco sobre isso, tinha receio de o deixar intimidado. Havia uma
rapariga com quem se dava bastante e que ele conhecia, a Mariana, muito
simpática. Num ano, haviam frequentado cadeiras em comum. Ela aprendia
violino, além do canto.

Quanto aos restantes, seria uma surpresa. Era bem verdade, agora
tornava-se quase doloroso, constatar que havia uma parte dele que não
existia. Ele não tinha contactos sociais. Mesmo os colegas mais próximos
jamais se lembrariam de o convidar para alguma coisa. Tinham-no na conta
de pouco divertido, pois passava o tempo todo com o nariz enfiado nos
livros. Embora fosse um rapaz bonito, não acompanhava modas. O sobretudo
coçado herdara-o do pai, assim como a sua magra compleição. O dinheiro
não dava para tudo. Como gastava bastante em material de trabalho, a mãe
via-se bastante aflita para lhe comprar fosse o que fosse.

Margarida vivia num bairro chique da cidade, ainda que perto da sua
casa. Um condomínio num apartamento restaurado. Nessa noite fazia
bastante frio. A casa possuía aquecimento central. Quando ele entrou,
estranhou o calor que se espalhava por toda a casa. A sua casa era fria
no Inverno. Ficou por momentos num canto, à entrada, sem saber o que
havia de fazer. Não conhecia ninguém. E havia imensas pessoas que
pareciam estar muito à-vontade. Passado algum tempo, Margarida veio
buscá-lo. Queria apresentá-lo aos amigos.

- Alguns conheces. - disse-lhe ela - nem que seja de vista! - Portanto,
porta-te normalmente, fala normalmente...- sorriu-lhe para o animar e
trouxe-o pela mão.

Aquilo parecia-lhe um suplício. Tinha vontade de sair a correr. Só o
amor o impedia de o fazer.

- Sim, tentarei não fugir\ldots{}

Um grupo de rapazes e raparigas estava sentado à volta do piano.
Florimundo aproximou-se, empurrado por Margarida.

- Este é que é o tal, Guida? - Perguntou um rapaz simpático que estava
sentado no banco.

- Trata-o com cuidado. Sabes como são frágeis os génios... -
respondeu-lhe ela, a brincar.

Nessa altura, Florimundo deu-se conta do abismo que os separava. Todos
aqueles rapazes pertenciam a um ambiente de que ele desconhecia as
regras. Tinham roupas de marca, gestos que demonstravam um perfeito
à-vontade, habituados a uma certa sociedade, que lhe era completamente
desconhecida. E, no entanto, apesar dessa diferença, menos social do que
económica, mostravam-se simpáticos e, até, deferentes para consigo.

- Quem são, Margarida? - Perguntou-lhe ele, aflito.

- Ora, são alguns do Conservatório, outros filhos de amigos dos meus
pais. Com o tempo verás, não te preocupes. Só precisas de seres tu
próprio, a autenticidade cai sempre em graça...

A rapariga procurava acalmá-lo, levando-o a ter mais confiança em si
próprio. Florimundo olhou para o piano, avaliando-lhe a qualidade. Um
piano de cauda, da melhor qualidade, incrível e que praticamente não
devia servir para nada. Estava verdadeiramente aparvalhado a olhar para
o piano.

- Senta-te e toca-nos qualquer coisa que te apeteça. Melhor, toca-nos
uma peça tua... - Pediu uma rapariga bonita e loira, que estava em pé,
ao lado dele.

O pai de Margarida aproximou-se e falou-lhe com simpatia.

- Hoje teremos espectáculo, espero. - O olhar dele era cúmplice, afável.
- Espero que nos dês uma pequena demonstração desse talento de que a
minha Guidinha anda sempre a falar. Hoje é a tua prova de fogo.

Florimundo sorriu timidamente. Aquela cumplicidade era a prova de
confiança de que ele precisava.

- Rapaz, - continuou ele, sussurrando-lhe - não te preocupes com o
aparato. A maior parte destes miúdos não vale um dedinho mindinho dos
teus...- e passou-lhe o braço por trás, e deu-lhe uma palmada afectuosa
no ombro. Estava a aprová-lo ou seria a testá-lo, pensou.

- E Mariana? - Perguntou Florimundo a Margarida. Ele esperara encontrar
a rapariga simpática e a única conhecida, para além da família.

- Ah, Mariana! Bem, ela não anda bem, não sabias? Tenho estado até
preocupada. Telefonei-lhe hoje e foi um choque. Não sabem bem o que tem.

- Tem um aspecto tão saudável! - Tinha uma expressão desolada.

- Vamos tentar não pensar nisso, sim? Se quiseres, amanhã podemos ir
vê-la ou assim. Olha, os tipos querem-te ali...repara!

A rapariga loura, soube que ela se chamava Flora, queria-o ao pé de si.
Insistia para que ele tocasse algo. A sala estava cheia de gente
desconhecida, mas não o suficiente que enchesse uma sala de
espectáculos, deixando-o a braços com um público intimista. Já tinha
tocado em recitais, mas nunca em bares ou ambientes parecidos. Todavia,
fez a vontade à rapariga. Tocou uma sonata de Mozart. Quando acabou, com
um ar triunfante, reparou que muitas cabeças se tinham voltado na sua
direcção. Agradeceu delicadamente as palmas. A rapariga deu-lhe um beijo
sonoro. Sentiu-se apatetado.

- Tocas maravilhosamente! - Disse-lhe ela, com um ar de admiração
fascinada.

- Pede-lhe que toque uma das suas composições... - disse-lhe um dos
rapazes.

Por sorte, Florimundo trouxera na sua pasta algumas das suas peças.
Tinha pensado deixá-las a Margarida, para que ela as visse.

- Fantástico! - Disse um dos rapazes. - Faze-las à mão ou num programa
específico?

- Primeiro, escrevo à mão, mas depois passo-as. Às vezes é directamente.

O mesmo rapaz, que já tinha acabado o curso de piano, olhava com atenção
para a partitura. Trauteava interiormente a sonata para piano e violino.

Bateu as palmas, com um ar pretensioso. Acrescentou:

- Não há por aqui um violino?

- Como? Tocas violino?

- Claro que não, mas aqui a Carolina sabe tocá-lo e nada mal. Já ganhou
umas medalhas. Não a conheces?

Não gostou nada do tom que ele usara. Um tanto trocista e provocatório,
quase arrogante. Resolveu devolver-lhe a estocada noutra altura.

Margarida lembrou-se:

- Sim, há um violino...papá, o teu violino? Está tudo a olhar. Não
sabiam que ele tinha tocado violino? De vez em quando ele ainda o toca.
Um bocadinho enferrujado da idade. Queres acompanhar-nos?

- Filha, não achas que a Carolina desempenha melhor essa função?

Florimundo esperou que Carolina se familiarizasse um pouco com a peça.
Leu-a duas vezes, rapidamente. Fez um sinal de assentimento.

Começaram ambos. A Florimundo preocupava-o que Carolina não conseguisse
acompanhá-lo. Mas ela tocava bem e tinha boa memória. Percebeu
rapidamente o desenvolvimento da peça. Acentuava de forma expressiva a
frase fundamental. Jamais lhe acontecera tocar assim em público uma das
suas composições. E num ambiente intimista. Ele estava de tal modo
concentrado na interpretação que não viu o que Margarida pôde observar,
encostada a um móvel, no canto da sala.

O silêncio impusera-se em toda a sala. Todos os olhares estavam
concentrados no jovem músico. Um encantamento atravessava o olhar dos
ouvintes. Durante o primeiro andamento, que começava com um som forte,
algo estridente, as cabeças voltaram-se subitamente.

A violência do som marcava uma ruptura brusca com o espaço e o tempo
quotidiano. Um som tempestuoso, onde podia divisar-se um grito
sacrificial. E, tal como irrompera, brutal, rapidamente desaparecera.
Marcando, agora, uma atmosfera de suspensão nocturna, silenciosa.

Subia, então, numa carícia lenta e profunda. As notas dissolviam-se,
antes de chegarem sequer a formar um sentimento ou uma paisagem durável.
Fluíam numa onda, líquidas, fundindo-se umas nas outras, entrelaçando-se
e tecendo arabescos, entre si. E dessa transitoriedade nascia toda a sua
força expressiva, mergulhando-os no êxtase. Por baixo dessa onda, a
pequena frase repetia-se, de forma multímoda, mas aguentando o timbre
melódico que lhe dava a força.

O rapaz de ar trocista tinha o rosto tomado pela admiração. Ficara muito
sério. Por fim cerrou os olhos. Margarida observava-o. Conhecia-o bem e
sabia-o sensível e arrebatado.

O violino irrompia em toda a sua força num momento em que o som do piano
se apagava, para lhe dar a entrada. Depois do violino se destacar por um
brevíssimo solo, o piano introduzia-se lentamente e entrelaçava-se com o
som do violino, de forma melódica. E toda ela subia e descia, percorria
o coração, as mãos, a noite. A sonata terminava com um tom pungente,
como se anunciasse uma despedida e o violino calava-se bruscamente,
deixando o piano sobressair em todo o seu virtuosismo.

Quando acabou, Florimundo ficou espantadíssimo pela salva de palmas.
Toda a gente quis ouvir mais. Estavam encantados com a sua música. Bem
tentou explicar que aquilo tinha bebido demasiado no romantismo, que
estava um tanto ultrapassado, que não era bem o que ele pretendia fazer.

- Pode dizer o que quiser - dizia-lhe uma senhora vestida de verde e com
um rosto afilado e magro - mas esta peça tem alma! Música com uma
expressividade tão intensa que faz comover as pedras. Depois do que
tenho ouvido nestes últimos tempos, credo!

Pedro era um pianista que estava em franca ascensão. Comovido, disse-lhe
que desejava tocar as suas peças.

- Mas ainda não ouviram mais nada... - ripostava Florimundo.

- Ah, mas isto, isto...este poder que exerces sobre nós! Sei de antemão
que tudo o que escreves é bom. A força, o arrebatamento que me
provocou\ldots{}não é assim tão fácil senti-lo frequentemente - Os olhos
brilhavam-lhe intensamente, num rosto belo.

Pedro estava visivelmente transtornado. Florimundo nunca tinha tido a
percepção do impacto que a sua música podia causar nos ouvintes. Aliás,
as suas composições só tinham sido ouvidas de forma incompleta, seguidas
de perto com o ar excessivamente formal dos seus professores de
composição, que lhe apontavam defeitos técnicos.

A mesma composição tinha sido censurada pelo professor de bigode
farfalhudo e ar doce, afável, o professor que ele mais admirava.

- Hum...- pigarreava, enquanto cofiava o bigode que lhe escondia a boca
e acrescentava na sua voz maviosa que contrastava com a corpulência -
demasiado emocional, demasiado expressiva, demasiado subjectiva. Não
pode confiar na inspiração, na, eis o pior erro que alguém pode cometer,
ficar para aí a olhar a beleza dos seus sentimentos. O artista não é
nenhum Narciso, não pode perder tempo a contemplar-se, tem de votar-se
inteiramente ao mundo, submergir, arrancar, ouvir, extrair o ouro das
buzinas do trânsito... nada de concessões aos sentimentos...Por amor de
Deus, liberte-se dessa sinceridade, "as boas intenções" têm de ficar
sempre à entrada, nada de deixar interferir vivências e sentimentos
pessoais na música, transforme-as, faça com que elas possam ser a
vivência de qualquer um, percebe? De qualquer um.

Florimundo compreendera inteiramente o que o que ele pretendia dizer.
Ficara-lhe essa impotência no corpo. Como? Como libertar-me dos
sentimentos?

O professor notou-lhe a interrogação. Sorriu-lhe com simpatia.

- Não se preocupe, isso vem com o tempo. Pouco a pouco, se tiver sempre
bem interiorizado esse princípio, verá...não se preocupe, para um miúdo
da sua idade é admirável, mas conseguirá fazê-lo melhor, ainda. Acredite
que chegará lá. E pôs um ar germânico, aliás aquele bigode só lhe
lembrava o do Nietzsche, de dedo em riste: «É preciso disciplina! Não só
a disciplina do trabalho, do método, mas a do espírito.».

Ficou a vê-lo afastar-se, com um sorriso estampado no rosto. Murmurava
para si «Claro que ele tem razão». Um dia todos veriam. Como ele saberia
incorporar a mais intensa emoção com a criatividade, com o apuramento
técnico. A crítica do professor dera-lhe uma vontade tremenda de
trabalhar o concerto até à minúcia, eliminando qualquer traço que lhe
parecia conceder demasiado à emoção. Cortar, cortar, acrescentar
virtuosismo técnico, cortar arrebatamentos, paixão...

A voz entusiástica de Pedro despertou-o do devaneio. Estava corado,
sentia-se confuso. Uma sensação insólita.

- Indubitavelmente bom. - Repetia o rapaz - Amanhã terás de ir à minha
casa. Tenho um piano e começaremos imediatamente a trabalhar.

Claro que Pedro não lhe dava tempo para respirar nem admitia qualquer
tipo de réplica. Apertava-lhe a mão entusiasticamente e sorria-lhe.
Tinha um ar efeminado. Pelo seu rosto passava uma admiração
caracteristicamente grácil, um atributo que só as mulheres possuem e
sabem ostentar com naturalidade.

Meio atordoado com a recepção, Florimundo foi instado pelos presentes a
tocar outra peça. Receou que aquilo se prolongasse, já que apenas tinha
trazido mais uma partitura. Então, fazendo exercício aos dedos, um pouco
entorpecidos do cansaço, sentou-se e tocou a sua sonata nº 2 para piano.
Se a primeira tinha tido um tal impacto sobre a assistência, a segunda
deixou um silêncio tal na atmosfera, que perdurou muito para além da
música.

- É incrível! - Dizia Flora - Não quero saber se é ou não original. É
maravilhoso. Onde tens andado metido?

Florimundo sorria timidamente. Sentia-se feliz, rebentava de alegria,
até estava corado de excitação. Olhou para Margarida que o amava,
através do seu olhar escuro. Duas asas de corvo misteriosas, que o
deixavam petrificado.

Chovia muito. Tinha-se constipado. Levava agarrada a si uma pasta de
cabedal, onde estavam as peças para tocar em casa de Pedro. Era assim
que o rapaz que conhecera havia dois dias, em casa de Margarida, se
chamava. Depois dessa noite fabulosa, ele levara-o a casa num Porsche
cinza, um modelo acabado de sair. Falaram ainda durante bastante tempo,
em que o rapaz lhe expôs tudo o que andava a fazer e os projectos que
tinha em vista. Pedro era falador, mas de uma inteligência brilhante, de
uma eloquência que não se tornava cansativa.

- E tu, o que pensas fazer, com todo esse talento? - Perguntou-lhe
Pedro, quando se calou acerca dos seus projectos. O modo como o fez não
era apenas formal, mas traduzia um interesse autêntico na vida de
Florimundo.

O rapaz sentiu-se verdadeiramente surpreendido pelo interesse de Pedro.
Não estava habituado. Toda a sua vida fora feita de uma espécie de
ausência, em que as coisas que fazia apenas lhe interessavam a ele e à
mãe. A alguns professores e, recentemente, a Margarida.

- Creio que tenho uma bolsa para continuar a estudar música.
Composição...a única coisa que verdadeiramente me interessa. Estou à
espera. E, bem...há uma peça em que ando a trabalhar. Tenho estado
concentradíssimo nessa sonata.

Pedro olhou-o com admiração.

- És tão jovem e já tens tanta coisa! Quando poderei vê-la?
Ouvi-la...tocá-la? - O olhar de Pedro mergulhou no seu.

- Ainda não quero mostrá-la...não é por razão nenhuma em especial, mas
estou tão envolvido nessa obra que me é difícil, ainda, mostrá-la.
Preciso de criar uma certa distância. Nem sequer sei se é boa. Será um
risco.

- Boa...temos criação! - Respondeu animadamente o jovem.

Subitamente o silêncio cresceu entre ambos. A noite estava fria, mas o
carro estava confortável. Pedro acrescentou:

- Não leves a mal o que te vou dizer...mas há algo que me escapa em ti.
Não sei o que é, mas vou descobri-lo, tens um poder incrível sobre os
outros... Quando tocas, na verdade, o mundo transfigura-se. Posso
confessar-te uma coisa? Estive sempre sob um encantamento qualquer, ao
ouvir-te. Não se trata de técnica...não sei explicá-lo...bem, logo
falamos nisso, quando nos conhecermos melhor. Não quero que penses que
estou a engatar-te.

Florimundo encolheu-se, rindo. Pedro apreendera algo que ele não queria
que os outros entendessem. O facto de ser músico e de a linguagem
musical lhe ser tão familiar possibilitava-lhe as ínfimas distinções.
Compreendera aquele animismo que havia na sua música.

Pedro vivia com um amigo. A sua casa era algo de assinalável. A mais
desarrumada que Florimundo já tinha visto em toda a sua vida. Por todo o
lado, empilhavam-se livros e CDs. Comprava tudo o que saía. O quarto
desarrumado, a cama cheia de roupas por cima. Ele lançou um olhar de
relance a toda a volta. A música, os pequenos ruídos, o pó que se
acumulava, tudo isso lhe trazia ressonâncias que rapidamente se
transformavam em sons.

Reconheceu \emph{Lohengrin}, que tocava baixinho, inundando a sala.

- Hum...o mago! - Disse Florimundo. Não sabia exactamente porquê, mas
não esperava encontrar ali aquela música.

- Adoro algum Wagner. Aquele que me devassa e me deixa com o coração na
boca...

- Sim, o \emph{Tristão, }o \emph{Lohengrin. }- Aquiesceu o rapaz. -
Porém, muita coisa de Wagner deixa-me perfeitamente indiferente.

- Creio que é preciso conhecê-lo profundamente para gostar dele.
Percebê-lo como vida pura, dionisíaca. -- Respondeu Pedro.

- Ah, mas tanto dionisíaco incomoda-me, põe-me maldisposto. Dionisos
para aqui, dionisos para ali, porque não arranjam uma paleta mais larga?
E porque é sempre o velho de Sils Maria?

- É fácil perceber porquê, quando se lê o \emph{Zaratustra. }- Retorquiu
Pedro. - Fácil perceber toda a violência da vida e da natureza, essa
paixão que atravessa a sua música.

- A paixão em excesso perturba-me. Incomoda-me, mesmo. - Respondeu
Florimundo. Um excesso de imaginação que pode tornar-se intolerável.
Todavia, não resisto ao Tchaikowsky nem ao Rachmaninoff. Vá-se lá
compreender esta contradição.

- Nossa Senhora, eu adoro o Rach. Aquela frasezinha do concerto número 2
para piano. As variações do Paganini...

- Sim e um tal Mahler da sinfonia nona...

- Isto só prova uma coisa. - Disse Pedro, num tom peremptório.

- Ah, o quê?

- Que a música é a mais subjectiva das artes. E que, por mais que
construamos juízos de gosto, andamos sempre de fora da essência da
música. Desse carácter objectivo da música.

- Uf! Deixa a metafísica para outro dia.

- Ah! Pois é uma vergonha, não é? Mas não me preocupo com isso. --
Referia-se à desarrumação - Deve estar a chegar a mulher da limpeza, que
tem faltado todos estes dias. Está tudo assim, não encontro nada, em
lado nenhum. Um horror!

- A minha mãe matava-me se deixasse o quarto assim...

- Pois, é imperdoável! Mas a minha nunca entrou no meu quarto. Deixou de
entrar quando era criança...

Florimundo sentiu um tom profundamente triste na sua voz. Pedro ainda
não tinha feito a barba. Acabara de se levantar. E não estava bem.

- Sentes-te bem?

- Nunca estou bem! - Respondeu-lhe num tom evasivo. - Toda a gente sabe
isso. Mas é outro assunto. Mudemos de conversa.

- Queres almoçar comigo? Estou esfomeado, ainda não comi nada. Ela deve
estar a chegar e voltamos quando a casa estiver mais arrumada.

Florimundo hesitou. Tinha pouco dinheiro e não podia desperdiçá-lo.

- Sou eu que convido. Anda daí. Mas espera um pouco. Vou acabar de me
vestir.

Deitou a roupa para o meio do chão, ficou em cuecas e pegou num dos
fatos que estavam em cima da cama.

- Ora, este...fica bem, para ir almoçar com o meu génio. - Riu-se.

Largou o fato, lembrou-se que ainda lhe faltava fazer a barba.

- Ai, mas não me apetece. É uma chatice. Não te importas, pois não? -
Florimundo apercebeu-se de que estava a ser seduzido. Pedro
comportava-se exactamente como faria uma mulher, para lhe agradar. Ficou
aflito. Sobretudo por apenas agora se ter apercebido da situação.
Incomodava-o menos toda a estratégia do amigo, do que mostrar-se
indelicado ou, mesmo, feri-lo. Na primeira oportunidade, teria de deixar
clara a sua posição. Mas não deixou de se sentir pouco à-vontade. A
melhor forma talvez fosse mudar o assunto e falar-lhe da namorada.

- Ia para te perguntar uma coisa. Há quanto tempo conheces a Margarida?
Não me lembro de te ver com ela no Conservatório.

Pedro olhou para ele e sorriu. Respondeu-lhe imediatamente:

- Margarida? - Pedro não acusou o toque - Ah, sim, o pai dela é o melhor
amigo do meu querido pai. Mas é preciso confessar que se revelou melhor
pai que o meu...- interrompeu-se e, com a sua perspicácia habitual,
continuou -- E\ldots{}não te preocupes, só seduzo quem quer ser
seduzido.

Florimundo desviou rapidamente o assunto. Não lhe apetecia avançar.

- Margarida é minha amiga desde infância. Digamos que somos quase
irmãos.

- É estranho ela não me falar de ti, sendo assim tão chegados.

- Tenho desiludido muita gente, na verdade. Afastei-me da Guida durante
algum tempo, aliás como me afastei de muitas pessoas. Dos meus pais,
inclusive...no fundo, isso foi apenas a concretização de um desejo.

- Bom, não quero meter-me na tua vida privada. Só nos conhecemos há dois
dias!

- No entanto, além da Guidinha e do pai dela, que são as pessoas em que
mais confio neste momento, sinto que tu és incapaz de fazer mal a
alguém, basta olhar-te, ouvir-te...Gostava de ser digno da tua amizade.

- E esse rapaz que vive aqui?

- Ah! Outro engano meu. É horrível, mas acredito no amor verdadeiro. E
tragicamente ele não existe. Está de saída... Não faz mais do que
alimentar-se e vestir-se à minha custa.

- És...- ia perguntar-lhe timidamente, mas foi interrompido.

- Sou bissexual, não te iludas! Eu próprio não sei bem o que
sou\ldots{}O amor transcende o corpo, o sexo. Apaixono-me pelas pessoas,
não me interessa minimamente a que género elas pertençam. Essas
distinções parecem-me infantis. Mas as pessoas gostam de dividir o
mundo. Maus para um lado, bons para o outro, mulheres para um lado,
mulheres para outro, normais para um lado, anormais e loucos para
outro...isso cansa-me, na verdade, cansa-me tanto!

- Se queres que te diga, não tenho muito tempo para pensar nisso. Passo
o tempo todo a trabalhar...

Pedro estava vestido. Gritou:

- Vamos comer...anda daí.

Desceram. Pedro tinha uma bela figura. Mais alto que ele, vestido de
preto, de sobretudo comprido e sapatos italianos, obrigava as pessoas a
olhar para ele. Deixava um rasto de charme onde passava. Sobretudo entre
as mulheres. Olhavam-no descaradamente. Ele sorria-lhes vagamente, sem
parecer interessado.

Entraram no restaurante e sentaram-se ao pé da janela. Pedro gostava
daquele lugar. Podia observar quem passava na rua. Entretinha-se com os
movimentos dos outros, lia no olhar das pessoas, estudava-as num
relance. Florimundo estava verdadeiramente magnetizado pela
personalidade do amigo. Se ouvira falar de decadentes, de dandys, ali
tinha um. Mandou vir uma garrafa de vinho tinto, caríssimo.

- Brindemos. Eu bebo sempre - disse-lhe ele- Sempre que posso, quando me
permitem.

- Bem, eu não estou habituado. Esperemos que...

- Que engraçado. Somos os opostos um do outro. Tu, um espartano
rigoroso, um asceta\ldots{}

- Sim. E tu um decadente. Achas que é possível entendermo-nos?

- Claro que sim. Da mesma forma que eu te fascino, também me admiro com
a tua personalidade. É inimitável. Não bebes, não sais, não conheces
ninguém, aposto que não tens sexo\ldots{}não vives, em suma...ah eu
preciso de sexo, desesperadamente, gosto do corpo, procuro o prazer,
acho que hedonista é um termo fraquinho para mim.

- Ora aí é que te enganas! Vivo num mundo que tu provavelmente não
conheces. Fora e dentro, ao mesmo tempo. Salvo-me pela música. Pela
disciplina, pelo trabalho. E tu?

- A música não me salva, com efeito. Mas isso parece-me chatérrimo. E
queres salvar-te porquê? Eu adoro perder-me. Ah\ldots{}perdia-me agora
contigo, mas tu és da Margaridinha e eu sou um devasso,
esquece\ldots{}claro que brinco, não olhes assim para mim. -- Riu
graciosamente - Eu pertenço a uma espécie de seres em extinção. Não
tarda muito...

- É estranha essa conversa. És tão promissor. Não parece a mesma pessoa
que falava comigo há dois dias, apenas!

- Sim, esse era o meu lado nocturno, iluminado...

- E este?

- Bem, este é o de sempre. Quando os dias vêm, a noite acaba com o seu
aconchego terno, a luz do sol, terrível, desoculta tudo...torna-se tudo
mais difícil, às vezes insuportável.

Pedro que já tinha bebido praticamente sozinho duas garrafas de vinho
pediu uma terceira. Florimundo achou estranho.

- Bebes sempre assim?

- Ah, isso não é nada. São os tais paraísos!

Estava tocado. Florimundo preocupou-se.

- Diz-me. É sempre assim?

- Ora, é isto que te preocupa? A mim, o que me preocupa, agora, és tu.
Não me venhas dar sermões de moral. Pareces a minha mãezinha...

O olhar dele tornou-se denso, opaco. As comissuras da boca traíam-lhe a
revolta. Teve um gesto brusco.

- Nunca conheceste nenhum filho de Saturno? Olha-o... - Falava de si
próprio como de outra pessoa. - Olha-o bem. Tem-lo aqui, diante de ti.
Doente da alma, intoxicado de acídia, sempre à beira do abismo...Tema
antigo e fascinante, estudado pelos medievais, objecto de um tratado de
Burton... É a vida sem rodeios, assumida.

- E se fôssemos trabalhar, saturnino? - O tom leve de Florimundo
acordou-o subitamente.

- E a ti, hem?! - Respondeu-lhe o outro. - Não há mistérios nessa música
estranha, vinda de não se sabe de onde, poderosa, misteriosa como só os
deuses a produzem? Vou arrancar-te o demoníaco segredo...

Ao ouvi-lo, Florimundo perdeu o sorriso, a leveza. Apertou os maxilares
e respondeu-lhe, num tom decidido:

- Não achas que estamos a perder tempo com metafísica? Não fales de
coisas que não sabes. Queres falar de filosofia ou ir ao trabalho?

- Ah, mas pela tua expressão, parece que sabes qualquer coisa sobre
demónios! Isso, antes que me embebede, vamos lá trabalhar\ldots{} -
Pediu a conta e Forimundo até teve medo de olhar.

Foi num ambiente de silêncio pesado que ambos começaram a tocar. A duas
mãos. Uma série de peças musicais em que Pedro estava a trabalhar. Com o
vício do compositor, Florimundo pôs-se a apontar os pequenos defeitos
das músicas que ouvia. Lirismo a mais, aqui, técnica pura e sem
equilíbrio, notas excessivamente dominantes...

- O mais importante, na música, mais importante ainda que as teorias que
estas novas escolas criam, é o equilíbrio melódico, a proporção...há um
segredo. A música conserva esse mistério, devido à regulação interna a
que ela obedece, as suas leis matemáticas...

- E o que é curioso é como a melodia nasce dessa abstracção matemática,
fico sempre aparvalhado com esse segredo de que falas...- Pedro
sentia-se cansado. O raciocínio saía-lhe visivelmente arrastado.

- Como dominas isto...- esfregou os olhos - Bolas, não aguento mais e tu
pareces estar a começar. És sempre assim?

A noite tinha descido. Um manto opaco.

- Que horas são? - Perguntou Florimundo, como se tivesse acabado de
acordar. - Esqueci-me da minha mãe, de Margarida. - Ah, na verdade, o
que te faz muita falta é a disciplina. Disciplina mental, física, dos
sentimentos, tudo...

- Seria uma máquina, rapaz... - Respondeu-lhe o outro, com um ar de
profundo desalento. - É isso que propões?

- Bem, acho que tenho de ir andando...O teu amigo não veio, ainda?

- Como vês, para não pensares que exagero...pensei que tinha encontrado
a pessoa ideal. Por ele, pus a minha vida toda ao contrário. Houve
momentos em que pensei que seria impossível continuar a viver. Foi
quando ele achou que eu já não lhe era útil.

- Vive do quê?

- Ah, és tão maravilhosamente ingénuo! Como é bom acreditar que existem
pessoas assim. - Tinha um ar precocemente envelhecido. Viu-lhe as rugas
ao canto dos olhos, uma certa flacidez, aos cantos da boca.

Margarida partia novamente. Regressava a Paris. Aparecera-lhe uma
oportunidade única que não podia desperdiçar. Um contrato por um ano
numa companhia de ópera. Em Portugal, não lhe apareciam propostas tão
tentadoras. Tinha estado no S. Carlos por algum tempo, mas sem segurança
alguma. Ao fim de alguns meses saíra. O pai mexera as suas influências,
mas nada feito. Em Paris, ela poderia continuar a estudar. Falou a
Florimundo em ir viver para Paris. Seria fácil para ambos, pois ela
arranjaria forma de lhe conseguir um lugar. Ele recusou, com pena. Não
conseguia abandonar Clara. Beneficiava ainda da bolsa e não terminara a
pós-graduação. Faltava-lhe um ano.

A ausência de Margarida doía-lhe fundo. Porém, sentia-se menos só, desde
que conhecia Pedro. Sentia que podia confiar nele e um laço muito forte
unia-os. O amor pela música. Pouco a pouco fora percebendo que a vida de
Pedro era um vazio tremendo, que só a música e o álcool atenuavam. Pedro
bebia muito, embora nunca o tivesse visto realmente embriagado. Era
também nessa altura que melhor tocava, curiosamente. Quando estava a
seco, como ele costumava dizer, a emoção e a capacidade de entrega
diminuíam-lhe. Não sentia o que fazia, não o emocionava. A Florimundo
assustava-o esta tendência auto-destrutiva.

O rapaz que vivia lá em casa, quando se haviam conhecido, tinha
desaparecido. Provavelmente a reacção depressiva de Pedro devia-se ao
seu desaparecimento. Passaram a sair juntos, se bem que a vida nocturna
aborrecesse mortalmente Florimundo. Todas aquelas pessoas que viviam as
suas máscaras desagradavam sobremaneira ao rapaz que, nada habituado,
achava tudo aquilo uma perda de tempo.

Por outro lado, essas saídas permitiram-lhe encontrar novos pólos de
interesse. A literatura, por exemplo. Pedro lia que nem um desalmado.
Lia tudo, de tudo. Como não se encontrava obcecado com a produção de uma
obra, o muito tempo que lhe sobrava ao trabalho passava-o a ler, a
conversar; dado que tinha insónias, quando não vadiava pelas ruas, lia
até de manhã. Normalmente, só começava a trabalhar à tarde, quando não
tinha espectáculos. Trabalhava pela tarde fora até à noite, depois saía
para jantar, encontrava pessoas e voltava tarde. Normalmente embriagado.
Invariavelmente, fazia-se acompanhar de belas mulheres, acordando com
criaturas diferentes, quase todos os dias. Muitas vezes, quando
Florimundo chegava, elas saíam a correr. Elas não faziam parte da sua
vida.

- Vais-te destruir. - Disse-lhe um dia Florimundo. - O que te falta,
céus? Tens mulheres incríveis, dinheiro a rodos, bons amigos...não
compreendo. Diz-me como posso ajudar-te. Não posso proibir-te de beber,
tens de ser tu a deixá-lo. Pela tua própria iniciativa. Os teus pais, a
tua família, não têm nada a dizer?

- Oh...fico sempre tão impressionado pela tua candura! Alguma vez viste
os meus pais dizerem fosse o que fosse? Telefonarem-me?

- Não percebo. - Retorquiu Florimundo.

- Ora, é óbvio. Sou totalmente indesejado. Maricas, um estupor na vida
dos meus célebres pais...sabes quem é a minha mãe? Se o soubesses caías
para o lado.

Disse-lhe o nome da mãe. Florimundo ficou boquiaberto. Primeira figura,
no mundo do espectáculo.

- Seu palerma. Percebes agora? Para não os incomodar, na sua vida
social, esplendorosa, dão-me o dinheiro de que preciso. Só me pedem que
não apareça.

- Eu cresci sem pai. - Confessou Florimundo.

- Mas que imagem tens do teu pai? Um herói, não é?... Pobre Florimundo.
No meu mundo não há heróis, apenas personagens de fantasia.

- Nós somos o que queremos ser, também. Podemos construir-nos. Seguir os
nossos sonhos ou as nossas miragens.

Despediram-se. Pedro levou-o à porta. Disse-lhe:

- Até já. Encontramo-nos no sítio do costume.

Florimundo saiu, com um peso no coração. O amigo deixava-o preocupado.
Não sabia mais o que havia de fazer-lhe para mudar a sua maneira de
viver.

Todavia, a música e Margarida apareceram-lhe, de novo, no espírito.
Flutuantes, ambas. Margarida de cabelo preso, a olhá-lo. Em pé, junto à
janela, o corpo delicado e nu, em contraluz. Ele esperava-a. Ela
deitava-se, cobrindo-o. Toda a ternura para ele. A maciez da pele, o
odor dos cabelos, a púbis desenhada num triângulo perfeito. A boca dela
percorrendo-lhe o corpo. Ele fechava os olhos e ela descia os seus
lábios rosados sobre as suas pálpebras. Tudo era canto, a tarde, o
silêncio, os gestos suaves, demorados. O rosto dela, quando a beijava.
Ela sabia a mar. Dentro dela naufragava.

Os trabalhos corriam-lhe bem. Tinha conseguido uma encomenda da
Gulbenkian, para integrar as novas obras. O tempo escasseava-lhe.
Apareciam-lhe convites para tocar aqui ou acolá, amiúde, ainda que
insistisse em dizer que não era um intérprete. De facto, não podia
dar-se ao luxo de desprezar tais convites. O convívio com Pedro,
introduzindo-o na alta sociedade, apresentando-lhe todas as pessoas
ligadas ao meio, facilitava-lhe bastante a vida.

Apesar de um certo mal-estar, que ele pressentia, quando se apresentava
em certos ambientes mais conservadores, talvez de pessoas mais ligadas à
sua família, as pessoas revelavam-se sempre solícitas. Florimundo cedo
percebeu a enorme influência que ele tinha no meio. O próprio Pedro era
convidado para tocar, mesmo onde a simpatia não lhe era favorável. Por
vezes exigia tocar peças do amigo. Sabia bem o que valia. E isso ninguém
poderia retirar-lhe, mesmo que não gostassem dele.

Uma noite, estava Florimundo a admirar o virtuosismo de Pedro. Este
tocava o concerto nº 1 para piano, de Chopin. A forma como Pedro se
envolvia apaixonadamente no segundo andamento, deixava o amigo
profundamente comovido. Na verdade, ninguém o emocionava até aquele
ponto. Pedro transformava-se num ser etéreo, possuidor de umas longas
mãos, que operavam um milagre sobre as teclas do piano. Sim, era uma
magnífico intérprete, dissessem o que quisessem ou pensassem dele o que
lhes apetecesse. Louco, sim, sem dúvida, mas de uma superioridade e de
um talento inegáveis. Tocar com ele, a duas mãos, acompanhar-lhe a
paixão, a suavidade melancólica, era uma experiência que ele não trocava
por mais nada. Num tempo e espaço sublime. Depois, tudo se calava. E
Pedro entregava-se de novo ao seu silêncio, mergulhando num copo de
vinho, de álcool, fosse o que fosse, onde se reflectiam os seus
fantasmas.

De repente, aproximou-se dele um homem, com a barba aparada, vestido com
um fato irrepreensível. Tudo nele era demasiado estudado, artificial.
Havia um toque de insinuante feminilidade, que desagradou profundamente
a Florimundo. Aparecera de uma forma súbita, sem aviso.

- Magnífico, não é? - Sorria-lhe, de forma insistente. - O primeiro a
ouvi-lo fui eu. Era um miúdo, com 7 ou 8 anos. Foi em casa dos pais
dele. Há quanto tempo se conhecem?

Florimundo pressentiu que havia algo de repulsivo naquele homem. Quis
marcar imediatamente uma posição, não de hostilidade, mas de firmeza. O
facto de ser tão próximo de Pedro e devido, ainda, à sua bissexualidade
assumida, dava azo a que se estabelecessem confusões.

- Há pouco tempo. Descobrimo-nos na casa da minha namorada. - Acentuou
bem a palavra "namorada" - Como sou compositor e ele um intérprete
fabuloso, aproximámo-nos naturalmente.

- Ah, é pena...- respondeu o outro - sabe que, além de magnífico
intérprete, é um amante incrível?

- Pois, nunca me dei ao trabalho de o sondar. Não me interesso por
homens...Pedro é o melhor amigo que se pode ter. Já se deu ao trabalho
de o verificar? - Devolvia-lhe assim toda a repugnância que o outro
acabara de lhe infligir.

- Claro que é um óptimo amigo. Não tenho a menor dúvida... - O homem
afastou-se com a estocada. Recebera-a com graciosidade. Como se soubesse
algo que o rapaz desconhecia. Ainda. Não o viu mais. Desaparecera tão
misteriosamente como tinha aparecido.

A sua imaginação fazia-o imaginar factos terríveis, talvez mais graves
do que o que realmente acontecera. Mas aquele episódio esclarecera em
muito a natureza do amigo. Uma raiva surda ascendera no seu corpo, numa
onda de sangue. Contemplou Pedro em silêncio.

Pedro observara a cena. Vira o homem aproximar-se do amigo. Quando ele
se afastara, finalmente, terminara a actuação com brusquidão. Tocou uma
nota estridente, várias vezes. Era como se gritasse, mas incapaz de o
fazer. Encontrara no piano o instrumento da sua vingança. O grito que
lhe saía da alma, sem procurar a voz.

Florimundo correu até junto dele. As pessoas afastaram-se, sem
perceberem o que se estava a passar. Foi um Pedro destroçado que ele
levantou do banco. Murmurou-lhe ao ouvido:

- Céus, não sou capaz de tocar mais. Olha como me tremem as mãos. Está
tudo acabado... - Disse, com um olhar ausente, profundamente alucinado.

- Não digas disparates. Estás esgotado. Vem, levo-te a casa. Fico
contigo esta noite. Há quanto tempo não bebes? Vamos ao hospital. Não
podes continuar assim.

Florimundo chamou um táxi. Pedro tremia e o olhar dele estava fora dali.
Provavelmente estaria em delírio.

Ele não parava de repetir:

- Tudo acabado, tudo acabado...

Esperaram várias horas na urgência. O rapaz tremia e balbuciava frases
que ele não entendia. Pela primeira vez, entravam ambos num local onde
se confrontavam com o sofrimento e a que nenhum deles estava habituado.

A sala estava cheia de pessoas que esperavam a sua vez, exibindo o seu
sofrimento. Havia no ar o cheiro do medo, uma decadência física que se
podia ver nos corpos das pessoas que os rodeavam. Face a eles, a figura
de Pedro, bem vestido, era quase insultuosa.

- Devíamos ter ido a um hospital privado. -- Dizia-lhe ele. Florimundo
nem pensara nisso, a pressa levara-o ao primeiro hospital e agora era
esperar. Pôs a mão no ombro do outro e perguntou-lhe:

- Não achas que devíamos telefonar a alguém da tua família? Passa-me aí
o telemóvel\ldots{}

- Nem penses! -- respondeu-lhe o outro, orgulhoso. -- Isto não é nada de
grave, amanhã já estou bom.

Um rapaz magríssimo, coberto de pústulas por todo o corpo, mantinha-se
ao fundo da sala, longe de todos os outros pacientes que aguardavam a
sua vez. Eles não sabiam se o tinham mandado para lá ou se o próprio
tinha ido só. À mente de Florimundo acorreu a imagem que vira em filmes,
de doentes leprosos, com peste, dos sifilíticos.

Mais perto de si, uma velha mostrava-se indiferente a tudo o que a
rodeava. Consciente, sem dúvida, mas mergulhada num silêncio sepulcral.
Pedro mantinha, a seu lado, o olhar alheio e vazio. Tremia menos agora,
sem que ele soubesse o que havia de fazer-lhe. O frio varreu a sala,
subitamente. Pedro voltou-se para ele, agarrou-lhe as mãos com força e
perguntou-lhe, em surdina:

- Viste-o?

- Quem? Estás a alucinar...

- O anjo negro? Viste-o passar? Sorriu-me...

- Qual, o de Rilke? - Respondeu-lhe, meio a brincar, procurando
desanuviar o ambiente.

- Falo a sério. Muito a sério...

- Olha lá, deves estar em privação, é só uma alucinação. Deves estar em
delírio e isso já passa. Tens é que te tratar\ldots{} - Olhou-o
preocupado e viu-o pálido - Tens frio? Estás gelado...- tomou-lhe as
mãos, pois o rapaz estava arrepiado de horror. Não sabia o que fazer, o
que dizer. -- Nunca mais nos atendem, que desespero. Estes devem estar
aqui há horas\ldots{}

Tirou o sobretudo. Tapou-o com ternura e acariciou-lhe a testa.

Pedro começou a chorar baixinho. Como as crianças, quando têm medo.
Abanava-se com gestos obsessivos. Parecia ter enlouquecido.

Florimundo saiu a correr pelo corredor, onde os olhares acusadores dos
outros pousavam sobre si.

- Não há médicos? - Gritou o rapaz.

A enfermeira, habituada a estes casos de desespero, saiu rapidamente do
gabinete do médico. Tinha os olhos mortiços, uma expressão impassível.

- Por favor, o senhor acalme-se. O senhor doutor está quase a atendê-lo.
Chega aqui e quer ser imediatamente atendido?! -- Disse-lhe com rispidez
- Já reparou que o seu amigo é o que parece estar em melhor estado? Olhe
à sua volta...

O rapaz serenou, reconhecendo que a conversa do amigo o assustara. Olhou
para Pedro, que parecia ter abandonado novamente o mundo, retomando a
expressão alheia, afundando-se no seu devaneio.

Retomou o lugar e sentou-se pacientemente. Aquele coro de rostos
marcados pela doença e pelo sofrimento emergia da penumbra, em vagas,
clamando, ou regressando ao seu silêncio. Na sua mente destacou-se então
uma voz, cristal finíssimo, crescendo, acompanhada pelo som do violino.
Uma voz que trazia a alegria e fazia desabar a escuridão à sua volta.
Toda a encenação operática lhe aparecia agora muito clara. Só tinha de
escrever. Desejou voltar para casa, rapidamente, sentindo-se culpado de
estar para ali a pensar em coisas tão inúteis diante do sofrimento. Mas
o que era a arte senão essa procura de redenção?

Finalmente, o médico, um rapaz novo e com um ar cansado, fez-lhes sinal
para entrarem. Pedro agarrou-lhe no braço e disse-lhe que entraria
sozinho. Estava bem, dizia, embora visse bem o esforço que ele fazia
para se aguentar de pé. Mantinha o seu porte altivo.

Esteve dentro do gabinete durante algum tempo, provavelmente uma
meia-hora. Depois saiu. Já se aguentava de pé, o médico tinha-lhe dado
um medicamento para o acalmar, talvez um sedativo. Entretanto, também
havia passado o pior da crise.

- Por amor de Deus, muda-me essa cara de enterro! - Pediu-lhe Pedro,
meio a rir, ao sair.

O médico apareceu por detrás dele. Vinha chamar o próximo doente. De
olhos encovados. Vinha tentar adiar o que não pode ser adiado. Ou as
águas do tempo, que reclamavam o seu território. Por momentos,
Florimundo deteve-se no seu rosto. Como vivia alguém, com a presença
constante do sofrimento e da morte? Habituar-se-ia? Regozijar-se-ia
pelos vivos? Fez-lhe um aceno discreto.

- Estou para ficar, rijo que nem um pêro. Só preciso de descanso, umas
noites bem dormidas, nada de orgias ou bebedeiras. A partir de agora é
só saúde e trabalho. - O tom malicioso inundou-lhe o olhar. - Bem...não
te preocupes, vamos falar a sério. É exaustão, foi o que ele me disse. E
a merda da bebida não melhora as coisas. Preciso de dormir mais, comer a
horas, fazer desporto, as balelas do costume. Amanhã... - Suspirou -
amanhã terei de procurar essa maldita associação. Acho que tenho um
problema para resolver. Mas garanto-te que vou ficar tão bem que vou
virar asceta, como tu...

Florimundo sorriu. Parecia-lhe que o susto lhe tinha feito bem. Muito
bem mesmo. Finalmente ele percebia que tinha de assentar.

- Anda, vou levar-te a casa. Vou chamar um táxi.

- O meu carro? Bolas, deixei-o lá? O meu belo carro?

- Agora não. - Opôs-se Florimundo. - Precisas de descansar. Não vais
pegar no carro a esta hora. Vamos! - Agarrou-o pelo braço. O outro
sacudiu-o docemente.

- Não, «florzinha», vai dormir, sossega a tua mãe. Telefonaste-lhe? Ela
deve estar preocupadíssima.

- Pois foi. Saí e esqueci-me completamente.

- A culpa foi minha. Vai para casa, abraça-a e diz-lhe que a culpa não
foi tua, mas do anormal do teu amigo. Fica com ela. Preocupa-te contigo,
agora. Eu vou apanhar ar fresco, vou a pé. Amanhã, irei buscar o carro.
Vou arejar as ideias. Está uma noite tão bonita, não é?

Florimundo olhou-o demoradamente. Um brilho estranho laminava-lhe a
íris. A sua expressão era confiante. E uma ternura que ele não conhecia
tomava-o, afastando o habitual sarcasmo.

- Eu sei o que te pôs naquele estado, estafermo. Podias ter-me dito há
mais tempo. Não quero interferir na tua vida, mas sabes que\ldots{}podes
contar comigo, dizer-me o que se passa. Para a próxima, não te
perdoo...quem era?

- Oh! Não sabes? Não o percebeste? O filho da puta...Ele conhece-te
bem...a ti também! Melhor do que tu te conheces a ti próprio. Claro, tu
não conheces os teus limites. Vives a adiá-los, a prolongar o teu
sofrimento, empurras a tua finitude... até onde? Ele também te irá
reclamar a alma\ldots{}

Florimundo gelou por dentro. O frio cristalizara, à sua volta. Sentiu-se
fulminado pela observação de Pedro. Calou-se, profundamente horrorizado.

- Só que, meu amigo, tudo tem um preço. Julguei que o sabias! Ele
possui-me de todas as formas possíveis, acreditas? Não há lugar que me
acolha, agora. Soube-o mais cedo que tu. Creio que isso nasce connosco.
O tal dom, essa maldição.

- Apenas procurei a perfeição, sempre. Nunca o mal... - Respondeu-lhe
desoladamente.

- Ah, a perfeição, a perfeição? - Os gestos de Pedro tornaram-se
irónicos - Num mundo de catástrofe, de imperfeição? A perfeição,
Florimundo? O que é isso? Acorda, vives num mundo de ilusões. - A voz de
Pedro tornou-se irada - Que espécie de seres somos nós, homens, sim, não
me olhes com essa cara... que procuramos a perfeição quando milhões de
crianças morrem diariamente na guerra, quando terroristas matam em nome
de um deus em que só eles acreditam? Que lugar existe para a perfeição?
Fala antes da «tua» perfeição\ldots{}

Uma irritação profunda transparecia-lhe no olhar. Riu-se, de forma
estranha, como um bufão.

- Porra, mais ideais e perfeição, amor, beleza. Puta que pariu todos
esses criadores de reflexos. Platão devia ter sido condenado e a seguir
todos os que ousaram erguer-se contra o culto do supérfluo, do
hedonismo... Amor, nos dias de hoje? Fodes duas ou três vezes e o amor
esgota-se, essa é que é a verdade. Confunde-se tudo. Oh, se eu te
contasse o que em nome do amor se pode fazer a alguém, como se pode
destruir uma pessoa - o rosto dele alterou-se inteiramente - o amor
mata. É terrível dizê-lo, mas mata. Mata quando nos entregamos
absolutamente a alguém que nos olha como se fôssemos um instrumento, um
simples meio. Como quando esperas poder dar a vida por essa pessoa e ela
se ri na tua cara. O que sabes tu disso?

- Existe, Pedro, tiveste azar. Eu amo Margarida. -- A palavra «amo»
soou-lhe artificial - E existe muito mais, como essa cumplicidade única,
como quando tocamos juntos. Não é beleza, isso?

- Um breve êxtase, é tudo...recuso-me a acreditar no duradouro. O que
temos é efémero, passagem\ldots{}deixa-te dessas mentiras platónicas e
bonitinhas.

- Esse teu niilismo dá cabo de ti.

- Deus, niilismo, muito gostas tu de categorias, devias era largar tudo,
dormir e comer que nem um abade e casar com a Guida. Tocar umas sonatas
delicodoces em saraus musicais, pagos a peso de ouro por essas
cinquentonas ricas e caprichosas. E ter miúdos e tratar bem a Guidinha,
comprar-lhe uma casa com piscina e\ldots{} - hesitou -- Ok, sei que
parece uma telenovela, mas é o mais próximo da felicidade que encontro.

Florimundo sorriu, reconhecera nele o sarcasmo que o mantinha vivo. E
deixou-o falar, divertido.

- Quero ver até onde vai o teu mau-gosto, Pedro!

- Pois, mas era o que devias fazer. De contrário, entregarás a tua vida
a algo que não existe...ah, a perfeição! O que eu soube hoje - apontou
para os olhos com os dedos - é que tudo consiste nesta coisa de lutar
contra a morte, viver o mais possível, para que possas partir sem
lamentar absolutamente nada...pois, vi-o passar à minha frente.
Sorriu-me, fez-me sinal de que a areia escasseava.

- Porra, és patético, melodramático! - Florimundo falou em voz baixa,
pronunciando arrastadamente as sílabas - estás doido! Precisas de
descansar. Acho que te prefiro bêbedo, apesar de tudo. Pareces-me mais
razoável...Não quero discutir contigo, não quero zangar-me contigo. Sei
que és um tipo suportável, quando queres. Vamos dormir, assentar as
ideias, apanhar ar na tola...se quiseres, ainda, acompanho-te, mas
peço-te que pares com essa conversa lúgubre.

- Tudo bem, meu querido. Não é preciso ires comigo. Hoje vou mesmo é
dormir! - O tom de falsete veio à tona. Depois acrescentou: - Schiu...O
silêncio...para as grandes dores. O silêncio, a música...a poesia.

Ainda hesitante, sem saber se devia ou não acompanhá-lo, Florimundo
ficou a vê-lo afastar-se.

A sua raiva desvanecera-se. Amava-o como se amava um irmão ou alguém que
faz parte de nós, que pertence à mesma massa e sabia que lhe perdoaria
sempre o que ele lhe dissesse. Lembrou-se das suas mãos a tocar o
concerto para piano, aquelas mãos que eram capazes de serenar os outros,
mas que se revelavam totalmente impotentes para se aquietar a si
próprio.

Não podia pensar a sua música sem ele, o seu olhar escuro, o outro, o
seu duplo, como o elo que o ligava à criação musical. Sempre que revia o
que mais amava neste mundo, isso reduzia-se a três elementos: o olhar de
Clara, a voz de Margarida, as mãos de Pedro.

Ainda, uma sombra negra e esguia, avançando. Caminhava com a elegância
que lhe conhecia, mergulhando na noite. Virou-se, ainda, uma vez e fez
um passo de dança, meio-brincalhão. Sorriu e acenou-lhe. Depois, a luz
da madrugada engoliu-o.

\section{PORQUE O TEMPO}

Florimundo chegou, atingido pelo desespero, pelo sentimento de culpa. O
pai de Margarida telefonara-lhe, a dar-lhe a notícia. A empregada
entrara em casa e voltara a sair, em estado de choque. Não sem antes
avisar a família. Florimundo voara até à casa de Pedro. Ele estava
sentado no sofá, junto à janela, como se não quisesse perder pitada do
mundo. Morrera de olhos bem abertos. Talvez uma última recusa tivesse
persistido nesse gesto.

O rapaz odiou as pessoas que se encontravam ali, que não lhe pertenciam
sob nenhuma forma. A teatralidade da morte, revestida das suas fórmulas
e códigos sociais, a presença obrigatória dos pais e dos amigos que o
tinham abandonado, tudo isso lhe causava uma profunda repugnância.

Pedro afrontava-os. Pálido, de rosto perfeito, tinha um sorriso
enigmático ou talvez fosse um esgar involuntário provocado pela morte. O
mesmo sorriso que Florimundo lhe tinha visto na noite anterior.
Malicioso, brincalhão. Talvez a morte lhe tivesse trazido essa leveza
por que sempre ansiara.

Pela primeira vez, via os seus pais. A mãe chorava, abraçada ao pai.
Tinha o rosto inchado de chorar, manchado pela pintura. Dizia, baixinho:

- Eu sabia que isto acabaria assim. Porque não o evitámos? Ele era tão
infeliz. Tinha tudo.

O pai mantinha-se calado. Não chorava, mas tinha o olhar vazio,
destroçado. Junto deles, estava o pai da Margarida. Florimundo,
aproximou-se. Tartamudeou algumas palavras sem jeito, deu-lhes os
sentimentos.

- Toma, esta carta é para ti. Foi a única que ele deixou. Margarida
chega amanhã. Acabei de lhe telefonar. Morrer desta forma\ldots{} tinha
tudo o que um rapaz pode desejar...o que pode ter falhado? Fui como um
pai para ele.

- Eu sei, ele gostava muito de si. Disse-mo várias vezes. Mas não era
feliz...

- Sinto não ter feito o suficiente por ele. Não ter percebido o que se
passava com ele. Não lhe ter procurado um psiquiatra, qualquer coisa.
Estas depressões curam-se, não é assim, o suicídio é uma psicose,
trata-se.

Mariana também se encontrava presente, apoiada nas suas canadianas.
Andava com dificuldade. Tinha um problema grave no fémur e já tinha
sofrido várias cirurgias. Com o seu ar de paciente de sempre,
sorriu-lhe.

- É bom ver-te, mas não nestas circunstâncias. - Disse-lhe.

- Conhecia-lo também?

- Claro. Todos o conhecíamos, mas ele afastou-se de todos. Só se dava
com gente que não prestava. - A boca adquiriu um tom desdenhoso que ele
lhe desconhecia.

O tom de Mariana irritou Florimundo. «Gente que não prestava», era
exactamente a expressão que todos usavam quando falavam de Pedro. Fora
esse desprezo que o matara. Todos tão perfeitos, na sua falta de
empatia. Dava-lhe vontade de rir. Olhava para os pais, com uma repulsa
inevitável. O que matara Pedro havia sido a incompreensão, a hipocrisia,
a intolerância. Por fora, parecia que nada lhe faltava, mas
interiormente era um deserto. Acompanhara-o nessa viagem, até ao fundo
de si, com a música como um aparente antídoto da tristeza. Porém, ele
sabia que, frequentemente, ela o deixava ainda mais exaurido, mais
entregue à solidão.

Viu-lhe os pulsos, a carne aberta numa chaga fina, traçada com uma
lâmina, que desaparecera. Provavelmente deitara-a fora, antes de se
sentar. Nada indiciava a violência do gesto. A sua própria morte tinha
sido cuidadosamente encenada. Um dia, ele dissera-lhe algo que lhe
parecera de uma beleza extraordinária: \emph{"Devemo-nos despedir da
vida como Ulisses se despediu de Nausica, - mais abençoando-a do que
apaixonado por ela."}\textsuperscript{\emph{\footnote{Nietzsche, Para
  Álem do Bem e do Mal.}}}\emph{. }Florimundo compreendia agora o que
Pedro lhe dissera nesse dia.

A superioridade de Pedro, via-o agora mais do que nunca, brilhava acima
de todas aquelas pessoas que Florimundo via. Paradoxalmente, o seu
suicídio era também uma celebração da própria vida, de uma coerência
absoluta, a contradição levada ao extremo.

O rapaz estava de tal forma perturbado que só lhe apetecia rir. Na
verdade, Pedro encontrara na sua morte a arte da suprema ironia.
Vestira-se com um fato negro e tinha uma camisa vermelha, cuja cor se
confundia com o sangue, que lhe escorria dos pulsos, já seco e
coagulado.

A seu lado, estava um cd de Mahler, \emph{A Canção da Terra, }que ele
tanto amava, e da mão direita tinha-lhe caído um copo, que se partira.
No chão via-se a mancha, provavelmente vinho, que se confundia com o
sangue. Na verdade, tudo aquilo era patético. Macabro, também, mas
essencialmente cómico. Uma arte de compor a vida, de tornar a morte
estética, alegórica. Como uma pintura barroca. Estudada ao pormenor, até
ao modo como a luz entrava na sala e iluminava o corpo.

Havia um gosto pela máscara, sabendo que a morte é ainda a última
encenação que nos resta. Depois da morte, as máscaras tombam, mas era
preciso levá-la até ao fim. Toda a vida tinha sido uma mentira, uma
ilusão, uma arte de habitar corpos e máscaras, de sobreviver entre os
seus simulacros. A única coisa que lhe dava uma consistência material, a
única verdade daquele rosto, era a música.

Florimundo percebia, agora, quão complexa tinha sido a sua
personalidade, enquanto se sentia acometido de uma profunda náusea.
Perdera o único amigo que possuía. Ele ensinara-lhe tudo sobre a
sublimidade da amizade e, agora, deixava-o só. Mesmo pensando em
Margarida, ele sabia que ela não poderia devolver-lhe aquela intimidade,
a perfeita compreensão mútua, o amor que os unia. O amor pela música,
secreto, para lá das pequenas contingências da vida, das tristezas, do
desencanto. Sim, percebia agora, Pedro amara-o. E o seu suicídio era uma
encenação grotesca, que apenas o tinha a ele como visado. Como o
impossível objecto do seu desejo.

A cabeça de Florimundo estava num caos. Parecia que ia rebentar. Saiu de
passo apressado, com a maldita carta na mão. Lá fora, desatou a correr,
as lágrimas a teimarem em correr.

Quando voltou à consciência, Florimundo percebeu que estivera sempre
ali. À sombra da sua faia. Fechado na sua dor. Imune ao mundo, à
alegria, indiferente ao riso das crianças.

Margarida sabia que iria encontrá-lo ali. Tinha a carta na mão. Não
conseguira lê-la. Aterrorizava-o o que iria encontrar. Margarida segurou
na carta. Guardou-a. Sentou-se ao seu lado. Abraçou-o. Não tinham
palavras, não precisavam. Entre eles havia um país onde só a ternura
podia penetrar, vencendo a dor, a escuridão, o medo.

Margarida assumiu por ele o gesto interdito. Aquele que ele jamais seria
capaz de realizar. Abriu a carta. Florimundo tremera, ao ouvir o som do
papel a rasgar-se. Doravante, não seria possível recuar. Tinha medo do
que Pedro sabia acerca dele. Com a sua sensibilidade, Margarida
percebeu-lhe o receio. Estendeu-lhe a folha de papel, na qual ele
reconheceu a caligrafia solta do amigo, descuidada e corrida, onde as
letras se enlaçavam umas nas outras, tornando, por vezes, as palavras
ininteligíveis.

Florimundo esfregou o nariz. Passou as costas das mãos nos olhos,
marejados de lágrimas. Começou a ler, surgindo-lhe a escrita envolta de
névoa. Não conseguia ver bem.

``Meu amado Florimundo,

Sim. Agora entrarei nessa ilha, onde tudo se purifica, a carne, a pele.
A alma (e tu sabes que não gosto desta palavra, mas não encontro outra
melhor). Levarei, não o cepticismo habitual, mas uma esperança da
libertação. Talvez agora encontre a tranquilidade, esse estar rente ao
silêncio, de que tantas vezes falámos. Talvez aqui, desculpa falar-te
assim, mas há muito que me sinto de volta a esta ilha...

Escrevo-te muito cansado, mas feliz. Feliz, sim, porque amava a vida,
mas ela não me amou do mesmo modo. Fui antes amado pela arte, pela
música, talvez pela beleza, mas não mais do que isso. Precisava de mais,
esperava mais dela, a que me entreguei sofregamente, como sabes.

Sei com tristeza que não te verei mais e, pior do que isso, que te
deixarei só e desejo-te a única coisa que quis para mim: ser feliz.
Talvez fosse amor ou uma das suas múltiplas formas...isso que houve
entre nós (rio-me, enquanto te escrevo isto, imaginando o teu olhar
aflito). Sê feliz com a «nossa» Margarida. Escrevo-te, sentindo uma
leveza que nunca senti, finalmente liberto de todo este fardo que
carrego e tu conheces. O tempo não é nosso. Levo-te assim, dentro de
mim, como uma luz íntima, sei que alguém como tu me terá amado, um homem
bom, inteiro. Guarda-me dessa forma, aquele que não conhecia a paz e que
agora está a um passo dela. Quanto a ti, meu amigo, meu amado, minha
«flor do mundo», não deixes que ele te leve.

Não deixes que esse olhar de anjo\ldots{}

Adeus,

Pedro''

Não se percebe porque não acabara a frase. Fechara a carta, onde não
havia qualquer traço de sangue, o que indiciava ter sido escrita antes
de ter tirado a vida a si próprio.

Ele sempre o soubera. Tudo se tornou claro na sua cabeça. Se sempre
evitara abordar o assunto com o amigo, no entanto, ele conhecia-o como
um irmão. Ambos eram filhos de Saturno, irmãos de uma outra linhagem,
que só a arte e a melancolia dão a ver. Os filhos de Saturno
reconhecem-se nos sinais que apresentam entre si. Talvez, aqui
Florimundo hesitou, um não seja sem o outro, uma espécie de avesso.

A diferença entre ambos estava no tempo. Em redor de Pedro, a mão do
diabo apertara-o, demasiado cedo. Nenhum antídoto teria sabido romper o
cerco da mão.

Estranhara a última frase, que quase poderia tomar-se por uma alusão a
um verso de Rilke, enigmática.

Abraçou Margarida, pedindo-lhe, no seu olhar desolado:

- Canta, meu amor.

Ela observou-o com atenção. Nunca o sentira tão sombrio. Atribuiu o
facto ao desgosto.

Na cabeça de ambos estalava um vazio. A voz de Margarida subia, suave.
Não lhe apetecia cantar, mas obedecia-lhe, sabendo que pouco mais podia
fazer.

Alguns corvos esvoaçavam, nas ramadas nuas das árvores. Não sabia se
tudo aquilo era verdadeiro. A única coisa verdadeira era a voz de
Margarida, lutando contra tanta escuridão.

Nessa manhã, enquanto bebia o café matinal, Margarida surpreendera-o,
mais pálida do que nunca, com os olhos maiores e mais encovados.
Atribuiu o seu estado à dor e ao desgosto por que haviam passado. No
entanto, os seus olhos estavam ainda mais belos do que habitualmente.
Havia neles um fogo desconhecido, uma fímbria de ternura, que iluminava
o rosto. Cada vez mais dançante, mais etérea.

A voz tornara-se-lhe mais límpida e queixou-se de um cansaço extremo.
Ele também se sentia exausto.

- Estarás anémica? Estás tão pálida...tens de ir ao médico.

- Sim, estou um pouco mais anémica do que o costume...- respondeu,
evasiva.

- Mas estás medicada, ou dizes por dizer?

- Pareces o meu pai, sempre preocupado comigo... sempre fui frágil.
Tenho uma anemia congénita. De vez em quando piora. Só isso. Tomo ferro
e passa, ao fim de algum tempo.

- Mas tens de averiguar a razão. -- Respondeu-lhe o rapaz,
obstinadamente.

- Bolas, não me estragues o dia, tu também...vamos preparar as coisas.
Como muito e recupero. Quanto tempo ficamos fora? -- Florimundo percebeu
que aquela pergunta se destinava a desanuviar a tensão e a desviar o
assunto.

- Uma semana? Na próxima, sabes que tenho o concerto. Tenho de voltar,
assistir aos ensaios...

- Esse \emph{requiem }vai dar cabo de ti, meu amor! Escrito tão em
cima...- evitou as palavras, para não magoar o rapaz.

Depois da morte de Pedro pouco tinham falado sobre o assunto.

Ele entregava-se ao trabalho para esquecer a dor. Porém, ela avivava-se
ao compor o \emph{Requiem}, evitando pegar na composição em que se
encontrava a trabalhar anteriormente.

Depois do desaparecimento do amigo, Florimundo receava voltar a pegar
tão cedo na sua sinfonia, onde tinha desejado concentrar todas as suas
energias, subordinando-a a uma ideia, uma ideia que dava unidade
orgânica à composição. Pensou em esperar até que a força regressasse,
essa pujança que havia constituído e suportado o núcleo melódico da
peça.

A morte de Pedro feria-o. Tal como na morte do pai, ele acreditava que a
música lhe traria o alívio, e evitava sobrecarregar Margarida, que
sofria igualmente pela morte do melhor amigo da infância.

Até que a rapariga se aproximou dele, perguntando-lhe:

- Porque não falas dele? Não queres saber como era em miúdo? Lembro-me
tão bem dele, tão suave. Sabes, tinha uns caracóis negros, era
lindo...era muito diferente. Inocente\ldots{}

- Margarida, pára com isso, faz-me sofrer ainda mais...

Ela afastou-se. Perante o mutismo dele, a onda de cansaço que dele
emanava, ela resolveu sair:

- Vou dar uma volta. Preciso de apanhar ar. Não queres vir?

Recebeu-a o silêncio.

- Sabes porque dizem ser a morte uma passagem? - A voz entrara na casa
da montanha, vinda não se sabe de onde, pouco depois de Margarida ter
saído.

Ele olhou em volta. O dia estava claro. A presença dele fazia sentir-se
como uma nebulosa, que não deixava distinguir nada.

- Até agora? Nem um minuto de paz? Levaste-o...vens cobrar-me algo?

- Tssstt, que exagerado. Ainda tens muito para usar e gastar. O teu
tempo é mais lento. Tu poupas-te. Ele não tinha nada a perder, vivia no
fogo da paixão...era um rapaz encantador.

Florimundo sentiu que ele não brincava e que o gesto de piedade era
autêntico.

- Porque falas assim? Encantador? Foste tu que o mataste?

- Que ideia idiota! Matar um ser tão encantador... Um animal perfeito,
de uma intuição soberba... Perfeito em todos os aspectos. Meu filho
dilecto. A intuição dele, a vivacidade, a inteligência, a malícia e a
capacidade de se exceder, não tinha medo de nada, nunca viste isso?

- Custa-me tanto entender! Também não tinha nada a perder, vivia como se
cada dia fosse o último.

- Seres onde se combina a perfeição da intuição e a inteligência pura? A
essas criaturas não lhes é concedido muito tempo num mundo odioso,
mesquinho...lembras-te do que ele te disse? Enquanto milhões de crianças
morriam de fome, tu aspiravas à perfeição musical?!

- Achava que os seres perfeitos deviam flutuar\ldots{}- respondeu
ironicamente -- ele vivia tacteando à medida que caminhava\ldots{}

- Não existem seres perfeitos, meu caro... - Respondeu o outro. - Posso
fazer-te uma crítica? Na tua música existe uma falta de energia que não
tem a ver contigo. Dir-se-ia que necessitas de fazer a última
ultrapassagem. Falta ainda superares-te a ti próprio. Abdicar dos
sentimentos, despojares-te da tua subjectividade.

- Ainda antes falavas nele como perfeito. Que contradição\ldots{}e
abdicar dos sentimentos? O que fica do homem depois disso? Mas creio que
comecei a fazê-lo, desde a sua morte.

- Tretas! A tua Guidinha, sempre tão preocupada, tão solícita? Olha, bem
a vejo, tem qualquer coisa de Eurídice, na sua palidez. Já reparaste?

Florimundo sentiu-se assaltado por uma cólera violenta. Sabia o que ele
queria dizer-lhe. Arrepiado, pediu-lhe:

- Não, não ma roubes... já me roubaste o meu pai, o melhor amigo.
Estiveste sempre lá.

- Precisas de aprender bastante sobre a relação entre vida e morte.
Claro que isso não depende de mim. O próprio tempo trata disso...

- Não...- pediu-lhe, novamente, com um lamento oculto na voz. - Sabes,
ela é a "minha voz"...

- E ele tinha «as tuas mãos»...Tudo a seu tempo, ao tempo é devido. Que
frase lapidar, não é?

- Como posso desfazer o acordo? Já não quero nada. Desisto de tudo.

- Não, não se pode negar o que nos chama, o que nos guia e sabes isso
tão bem quanto eu. Que espécie de vida seria a tua sem música?

- Estou esgotado. Sabes disso. Também Pedro... - Não chegou a concluir a
frase.

- Não contei com a sua fragilidade. Uma característica humana, claro. Tu
não me desiludirás. Sabes que a única possibilidade de celebrares a
alegria está em ti...

- A alegria? - Perguntou Florimundo, atónito.

Não obteve resposta. A nebulosa tinha-se evaporado. Ficara o cheiro a
enxofre, pairando no sítio. Inabalável.

Florimundo pegou no casaco e saiu a correr. Estava um frio de rachar.
Não havia rasto de Margarida. Ela só devia ter seguido a vereda que
faziam habitualmente. Conhecia mal o caminho. A erva estava molhada, a
chuva deixara sinais na terra. De um lado e de outro da estrada de
terra, viam-se os prados de um verde vivo.

As nuvens negras aproximavam-se, ao longe, e um vento frio anunciava
chuva. Não lhe agradava nada saber que Margarida iria demorar-se, sem
saber onde ela poderia ter ido. Arrependeu-se de não ter saído com ela.

Homens mantinham-se sossegadamente, à porta das casas. Alguns passavam
pelas brasas, num sono ligeiro, eram homens cuja única preocupação
consistia em ser nos dias e no tempo.

Florimundo invejou-lhes a placidez, a animalidade serena, mesmo quando
estavam de olhos abertos. A vida era simples. Sem música, sem arte.
Pernas fortes, os ventres proeminentes, o rosto rude e marcado pelos
frios Invernos, em que saíam de madrugada, enfrentando o vento agreste e
subindo as inóspitas escarpas, levando os rebanhos a pastar. Às vezes
voltavam-se para o observar, mas depois recaíam no seu inabalável
sossego.

Uma águia sobrevoava o rochedo mais proeminente. Ao longe, o rapaz
observava-lhe o voo, em círculos, provavelmente esperando um coelho que
se abrigara. De Margarida não havia sinal. Por fim, chegou ao lugar,
onde ambos gostavam de ficar sentados a ver a luz tombar sobre a pequena
aldeia. Ao final da tarde. Florimundo contornou a grande rocha que
encimava o monte. A rapariga chorava desesperadamente.

Teve medo de aproximar-se e assustá-la. Tossiu ligeiramente, para a
avisar da sua presença. Ela levantou o rosto. Tinha os olhos vermelhos,
o nariz inchado, estava irreconhecível de inchada. Nunca a havia visto
tão desesperada. Pensara-a sempre de uma força inabalável, à semelhança
da sua mãe. Clara jamais se lastimava ou chorava. Embora tivesse um ar
permanentemente triste, de onde a alegria se exilara para sempre, desde
a morte de Gabriel, jamais mostrava a sua fragilidade, aparentando uma
indiferença por tudo o que fosse sentimental. Percebia, agora, a
natureza dessa força que as mulheres possuem. Não passava de um disfarce
que elas usavam. Guardavam a dor para os momentos de solidão,
escondiam-na como uma maldição.

- Porque saíste sem mim? Deixaste-me preocupado...

- Só pensas em ti, na tua criação.

O olhar dela era acusador. As palavras atingiam-no inesperadamente.
Baixou o rosto, reconhecendo o egoísmo, a indiferença com que a tratava.
Não era bem uma indiferença, mas uma incapacidade de lhe dar o que ela
queria.

- Não posso viver mais contigo, sabias? Não consigo lutar contra essa
escuridão...

A economia das palavras atordoou-o mais que o seu significado. Recebeu o
golpe, sem ser capaz de responder-lhe. Sabia que ela tinha razão. Não
tinha nada com que defender-se.

- Não sentes nada por mim? -- Limitou-se a responder.

Ela demorou a responder. Esperava que ele gritasse, barafustasse,
mostrasse que a amava, em vão. As palavras dela queriam atingi-lo,
provocá-lo, obter uma reacção, mas ele não conseguia mostrar-lhe
qualquer raiva, ódio ou amor. Ele estava vazio, exausto, triste de
morte. Ficou apático, a olhar para ela, sem saber o que dizer-lhe, a não
ser jurar-lhe que a amava.

Por fim, ela levantou-se. Disse, com um ar sério, de quem havia tomado a
resolução havia bastante tempo:

- Mas não sentes paixão, eu preciso de paixão, de sentir-me desejada.

Florimundo soçobrou, as lágrimas teimavam em sair, contra a sua vontade.
Percebia que ela precisava do que ele não conseguia dar-lhe.

- Volto para Paris. Não me sigas. -- Disse-lhe ela, enfrentando-o com um
ar distante, mas não zangado.

- Casa comigo. - Pediu ele.

- Seria sempre assim. Não posso...Preciso de vida, não é isto.

\section{\textbf{TEMPUS FUGIT}}

Muito bem acolhido pela crítica, o \emph{Requiem }era tocado
frequentemente, agora que Florimundo dava concertos em Paris, Londres,
Berlim. O reconhecimento chegara e a crítica não poupava elogios à
grande revelação do ano.

Pedro, o acossado, o bêbado, o devasso, foi reabilitado e a peça de
Florimundo contribuíra para essa aura. Já não era o homossexual
excêntrico e devasso que haviam conhecido, mas o génio tocado pela
melancolia, devorado pela incompreensão e pela indiferença social. Os
pais apareciam por tudo o que era sítio, em homenagens póstumas,
reconhecendo no filho as qualidades extremas: um filho dedicado, amável
e que, infelizmente, a desgraça tinha levado. Que isso podia servir de
exemplo aos pais, quando vissem comportamentos estranhos. A depressão
transformou numa vasta e enigmática explicação para tudo o que
acontecera.

Chegavam mesmo ao cúmulo de reclamar uma pequeníssima parte da culpa,
mentindo sobre as atitudes que tinham tomado para evitar a catástrofe.
Mariana esteve presente em quase todos os concertos, mas já não
conseguiu participar na última homenagem. Tinha-lhe sido detectado um
cancro nos ossos. As metástases espalhavam-se por todo o corpo. Foi-se
num dia de Primavera, quando não era suposto, de tão jovem que era.

Margarida veio a Portugal, passados alguns meses. Anunciou-lhe a nova.
Estava grávida e ia casar. Não lhe perguntou, sequer, se ela era feliz.
Abraçou-a e deu-lhe os parabéns. Ela retomara uma luminosidade e uma
alegria que, com ele, haviam desaparecido. Não a fazia feliz. A tristeza
dele contaminava-a, arrastando-a cada vez mais ao fundo. Embora a
amasse, preferindo tê-la a seu lado, gostava de a ver assim, tão viva.
Retomara o seu aspecto dançante. Engordara. Tinha o corpo roliço, a pele
rosada. Estava linda.

Continuava a escrever-lhe. Ela também, após um breve interregno chegara
à conclusão de que nada adiantava evitá-lo. Ele seria sempre o seu
melhor amigo.

- Como é ele? - Perguntou-lhe com curiosidade autêntica.

Ela sabia que nenhuma espécie de malícia se escondia naquela pergunta.

- É fogoso. Faz pensar na maneira de ser de Pedro...

- Curioso! Parece que ambos o amámos, não é? Cada um à sua maneira...

- Sim, nunca cheguei a aperceber-me disso. É como se ele sempre tivesse
vivido entranhado em mim. Desde miúda. Puxava-me as tranças, batia-me,
chamava-me nomes...

Florimundo riu-se. Desconhecia esse lado alegre de Pedro. O traquinas, o
sacana.

- Um dia quis contar-te isso. Não me deixaste. Pedro foi sempre uma
fonte de alegria para mim. Em ti há um peso com o qual se torna difícil
viver...

- Talvez por isso o tente na música, arrancar este peso. Será rebuscado
encontrar aí uma explicação?

- Não sei. Sei que és um magnífico compositor, um amigo incrível.
Adoro-te.

- O que significa isso? Ser o melhor amigo? Que não há desejo? Atracção?
É isso que queres dizer-me?

- Nunca resultaria, sabes? Preciso de alguém que alimente a minha
natureza etérea...com paixão.

- Sei-o. Não te exigirei nada. Nunca. Só a tua voz...

- Irás a Paris ter comigo... connosco?

- Claro. Arranjo uma semana algures entre o meu trabalho.

- Tens um emprego fixo?

- É difícil a permanência, ainda. Acho que o devo sobretudo ao Mohammed,
que tem um bom emprego e eu vou-me safando.

- É giro, teres acabado por casar com um árabe. Não imaginaria.

- Não é árabe, ora bolas, é francês. Nem o Ramadão pratica, come e bebe
fartamente. Viver em Paris é bom. Bem diferente de Lisboa. Tem uma vida
cultural impressionante. Como músico irias adorar. E lá dar-te-iam mais
valor...

- Sem dúvida, mas sabes que não posso deixar Clara só. Faz-me imensa
confusão.

- Ela iria habituar-se. Mais tarde ou mais cedo, lindo, aparece-te a
mulher dos teus sonhos e ela acabará inevitavelmente por ficar só.

Ele olhou-a, sorridente. Havia uma esperança, uma escada lançada. No
entanto, ele não podia confirmar-lhe a ideia de que apenas a amava a
ela. Seria penoso imaginar-se com uma mulher que não fosse Margarida.

- Não. A mulher dos meus sonhos foi-se embora!

- Parvo, é o que tu és. Parvo, és lindo. Aparece certamente essa sujeita
e mais depressa do que imaginas. Não podes é passar o tempo obcecado
pelo passado. Os desígnios do amor são estranhos, Florimundo. Ele nasce,
conhece períodos de acalmia, retorna, com a ausência, renova-se noutros
seres...

- Hum! Conversa estranha a tua...

- Vives demasiado embrenhado na música, no teu trabalho. Pára e olha um
pouco à tua volta. Sai, conversa, repara no mundo.

- Sabes que reparo nele sempre do mesmo modo! Convertendo-o...

- Gostava de ser capaz de te amar inalteravelmente, sem oscilações, sem
qualquer necessidade de outro tipo... a paixão, já ouviste falar?

- Um dia perceberás que tudo isso existe em mim.

- Meu querido, tu deslocas toda essa paixão para o que crias, não a
transmites a ninguém. Não queria dizê-lo e comparar-te. Mas é um pouco
como o que se passava com Pedro. Essa paixão escoava-se na música, era
incapaz de se dar inteiramente a alguém, de se entregar.

- Eu entreguei-me.

- Acredito. À tua maneira. Equilibrada, com harmonia, proporção. Como se
o amor fosse matemático, regular... frio.

- Magoas-me, ao dizer-me isso.

- Desculpa-me. Tinha de dizer-to. Quando saí não fui capaz. Sentia que
havia algo que me fazia falta. Uma certa leveza. Não estou preparada
para a tua forma de amar.

- E, no entanto, crês que o meu amor é inalterável.

- Sim, como o amor que sentes pela tua mãe. Absoluto, reconfortante,
caloroso. Acho que o problema está mesmo em nós, na nossa natureza. A
terra e o ar...

- Fala-me dele.

- Quem? Mohammed? Meu Deus, isso magoa-te.

- Não, não magoa. Gosto de saber que és feliz com ele. Espero que não te
convertas\ldots{}ou ao teu filho.

- E se isso acontecesse? Seria grave? Seriam duas tradições a que ele
teria acesso. Provavelmente aprenderá árabe, espero que aprenda, é uma
língua maravilhosa.

- Sim, mas ouve-se tanto disparate. Agora, sobretudo, com os atentados,
este horror do terrorismo. Tenho curiosidade em conhecer Mohammed. Saber
tudo...como é que ele te conquistou, por exemplo. Quanto a nós, sinto
que sempre nos pertencemos, um ao outro. Talvez o erro tenha estado aí,
tenhamos confundido algo que devia ter permanecido imaterial...

Ela tinha lágrimas nos olhos. Profundamente comovida, agarrou-lhe nas
mãos:

- Eu também sinto que nos pertencemos um ao outro. Que fizeste sempre
parte de mim. Quando penso em ti, evoco-te sempre dessa forma, sabes? A
infância...tu ficavas ali a olhar-me. Eu brincava à macaca, ria com as
minhas amigas.

Ele interrompeu-a, espetando-lhe o indicador na ponta do nariz:

- Mohammed...conta-me. Bem, pormenores dispenso!

- Como o conheci? Ias espantar-te. Estava vestido de palhaço da primeira
vez. Não era bem de palhaço, mas de Pierrot. Dançava naquela peça de
Schönberg, o Pierrot Lunaire. Ele foi fantástico. Lembras-te da cena em
que Pierrot vê os morcegos a tapar o sol, a mergulhar o mundo na
escuridão? Creio que foi nesse momento. Um leve movimento de piedade, o
olhar dele, atormentado, pousando sobre mim, foi o suficiente para que
desencadeasse o imperceptível movimento da alma...

- Que romântica estás, nossa Senhora! E depois?

- E depois ele viu que eu fiquei para ali embasbacada a olhá-lo. À
saída, chovia desalmadamente e eu tive de esperar, à porta do teatro,
não levava guarda-chuva. Ele apareceu. Estava sozinha, ligeiramente
comprometida, até parecia que tinha preparado o encontro.

- Deduzo que não tenhas partido imediatamente com ele. Que resististe!
Quero acreditar que resististe...

- Sim, tentei. Ele falou-me, tem uma voz maravilhosa, é alto, magro,
lembra imenso o Pedro...não é tão moreno, mas tem os mesmos olhos,
largos, lembras-te?

Florimundo arrepiou-se. As coincidências eram notáveis.

- Estou atónito...

- Eu também fiquei perturbada. Algumas vezes, sentia que era como se ele
tivesse voltado para me ir buscar. Lunar, afastando a melancolia, os
morcegos...mudemos de assunto, que estou a atrofiar!

- Não, continua. Um certo cepticismo, uma disciplina saudável não devem
fazer-nos mal, acredita!

- Ele insistiu que eu viesse com ele. Tinha o carro a alguns metros
dali. A zona não era segura, àquela hora, chovia imenso. Ia pôr-me a
casa.

- E tu...

- Claro. Tinha decidido não voltar a amar.

- Mentirosa!

- O amor aparece do lado que menos espera. Como a morte. Nunca estás
preparado...

- Acredito. E, todavia, essa imprevisibilidade é de uma beleza
avassaladora, não é? Ser arrastado por algo que não se sabe o que é...

- De caminho, ele disse-me que tinha fome. Era cedo, ainda. Disse-lhe
que havia uma brasserie ao pé da minha casa. Ele respondeu-me, muito
naturalmente, que detestava comer sozinho.

- E tu, muito naturalmente, acedeste a fazer-lhe companhia. Vejo...
noites frias, solidão e Paris...

- Vá lá, não me gozes.

- A sério, estou a gostar. É tão bonita, a vossa história!

- Não houve nada de forçado. Apaixonámo-nos e foi tudo...ele
perguntou-me o que fazia.

- E tu disseste que cantavas.

- E ele pediu-me para ouvir.

- E tu respondeste-lhe que isso teria de ser em privado...porra, estou a
ficar cheio de ciúmes. O que cantaste?

- Ah! - Ela sorriu-lhe, com o seu ar cúmplice.

- A \emph{Avé Maria}, claro! E que mais?

- Algumas peças de Schönberg.

- Corajosa, hem?!

- Achei que, se ele era o Pierrot, eu poderia ser uma parente...

- E depois?

- E depois passei um mau bocado. Sem saber se era só amizade da parte
dele, sem me atrever a nada. À espera de que ele me dissesse algo...

- E suponho que ele também esperava.

- Exactamente. Fazíamo-nos doer um ao outro, para termos a certeza. O
desejo físico era imperioso. Um dia beijei-o e parecia que tinha ido à
lua.

- Ele disse-te que tinhas afastado os morcegos todos, claro! E depois?

- Hum! Mudemos de assunto. Vens jantar à minha casa, amanhã? Tenho
saudades tuas. Poderíamos conversar e vejo em ti uma tristeza que não me
agrada.

- Por favor, não me arrastes a esses jantares onde todos esperam, em
vão, que o génio faça salamaleques. Dói-me muito tudo...agora. Gosto de
estar só. Ouço música, leio, sabes como é.

- Mas tens de dar-te com pessoas, fora do ambiente de trabalho. Repara
como estás cada vez mais fechado sobre a música. Tudo o que te rodeia
vive dela. Creio mesmo que o suporte da tua vida...

Não continuou a frase. Percebeu, pelo olhar dele, que se encontrava
bastante longe, embora a fixasse. Não era preciso muito para compreender
que ele ainda a amava. Agarrou-lhe a mão e insistiu:

- Amanhã à noite apareces, não apareces?

- E tu cantarás? - Perguntou ele, esperançado. - Tenho saudades de
ouvir-te cantar.

- Sim, será como antes. Tu tocas e eu canto.

- Não fales dos tempos antigos. Tudo ruiu, desde a morte de Pedro.

- Bolas, liberta-te disso. Já passou tanto tempo, desde aí!

- Não conseguiria explicar-te isto. Não sobrevivi ao olhar dele. Esse
condão estranho que ele possuía sobre mim. Depois da morte dele, uma
enorme tempestade varreu tudo em que acreditava, percebes?

- Meu querido, não podes passar a vida inteira apegado à memória de
Pedro. Escreve, liberta-te dessa dor, transforma-a. Terás de viver com
ela toda a tua vida. Como eu. Como todos os que lhe eram próximos.

Os olhos de Florimundo encheram-se de lágrimas.

- Dói-me sempre tanto, tudo. De cada vez que toco aquele \emph{requiem,}
é como se revivesse tudo.

- É isso que faz de ti o que és. A tua capacidade de sentir, de te
ligares às coisas, de estares simultaneamente na profundidade e subires
à superfície... - a voz dela teve um sobressalto inesperado, devido à
emoção, que transparecia claramente.

A aragem marítima chegava até eles. O mar era todo deles, ali à sua
frente. O vento veio despentear o cabelo de Margarida. O sol baixava, no
horizonte. Fulva, uma mancha, anunciava um outro dia quente, no
horizonte. Ele olhou-a, emocionado. Uma sucessão de recordações
assaltou-lhe o cérebro. O corpo dela, macio, a penugem loira a
desfazer-se sob os seus dedos. Os lábios vermelhos. E o sol vinha poisar
agora, sobre os mesmos lábios que ele beijara com amor. Teve vontade de
acariciá-la, como fazia quando viviam juntos. Ele passava-lhe as mãos
pelo cabelo macio, ela aconchegava-se e aninhava-se no seu colo. O
silêncio vinha e a ternura arrastava-os. A música, tão suave, como havia
muito tempo que não lhe acontecia, explodiu-lhe por dentro. Apertou os
maxilares, opondo uma barreira diante dos seus sentimentos.

Por mais que se esforçasse, vinha-lhe à memória o corpo de Margarida,
como naquele dia em que tinham vindo jantar no mesmo sítio. Passearam
pela praia e ela dançara diante ele. O corpo dela obcecava-o, na noite
mansa. Tinha um vestido claro que caprichava contra o vento e a
escuridão. Dançava mentalmente, sem precisar de música, tal como ele
podia ouvir a peça que ela ouvia. Ela dançava Rachmaninoff, mas a
liberdade dos seus gestos pela praia fora, o vestido oscilando na noite
luminosa e as suas pernas altas a lembrarem os movimentos de uma
garça\ldots{}ele e ela a ouvirem interiormente essa música sem
precisarem do som, tudo isso lhe parecera tão deslumbrante nessa noite
que ainda vivia em si como a mais bela das suas memórias, agora
dilacerando-o, rasgando-o como uma ferida em carne viva. Quando não se
precisa de mais nada, nem do som, mas havia o mar, esse marulhar eterno
e as pernas dela dançavam ao som do mar, sem precisarem de nada mais.
Era isso a liberdade, o canto sem voz nem música, rente ao silêncio. Em
nós o mundo e a linguagem.

Doeu-lhe novamente o passado, o azul do mar, que brilhava ao longe, o
rosto de Margarida, tão ausente e simultaneamente tão próximo. Percebia
a impossibilidade à flor dos dedos, essa impotência de tocar-lhe, apesar
de o desejar intensamente. Quase compreendia que Margarida não o
evitaria, se ele o tentasse. Mas qualquer resposta da parte dela seria
apenas compaixão. Talvez de ternura, pelo passado que os tinha unido
estreitamente. Era amor, ela não o sabia ainda ou talvez suspeitasse,
mas era isso o amor. Uma das suas formas, irreparável.

Sentiu-se miserável. Julgara poder alcançar um ideal, um domínio mágico
do mundo através da música e percebia agora que era essa mesma
compreensão que o afastara dele. Em tudo. Que toda a vida perseguira um
sonho e se encontrava agora diante dele, a coisa mais importante da sua
vida. Aquela mulher que se afastava dele, mergulhando para sempre na sua
memória.

Começava a fazer frio. Preocupou-se com ela. Com a criança que trazia no
ventre. Quase que a sentia como dele, a mesma ânsia de protecção. Pouco
lhe importava que a criança não fosse dele. Sentia uma emoção muito
forte, uma ternura relativamente a esse pequeno ser que ainda não
possuía um rosto. Ela continuaria a ser dele. De uma outra maneira.
Sempre dele, através dessa criança que também seria um pouco dele. A
sensação consolou-o

- Estou desejoso de ver a criança. Como vai chamar-se?

Ela acordou do seu devaneio. O sol brilhou nos seus dentes perfeitos.

- Nome? Hum...Mohammed quer um nome francês, eu um nome português.

- Podes sempre pôr-lhe o nome de Pedro. Seria justo e certamente que lhe
agradaria. De certo modo uma homenagem...

- Sim, num sentido dúplice. Pierrot e Pedro...Já tinha pensado nisso.
Mas, se for rapariga? É mais complicado.

- Não sei. Adoro o teu nome. Tem tantos ecos! Mohammed não gostaria?

- Há sempre uma mãe, uma tia, uma ave agoirenta por perto.

- ...Mélisande.

- Hum, acho que ele detesta!

Anoitecia. Ela precisava de voltar. Ele tirou o casaco e obrigou-a a
vesti-lo. As mangas desciam-lhe pelos braços abaixo, demasiado
compridas.

À despedida, olhou-a demoradamente.

-... Acho que é a tua voz que me falta. Era ela que me arrancava à
tristeza. A tua voz pairava sobre a minha tristeza e desfazia-a.

- Não se pode fazer nada contra a solidão, Florimundo. Creio que é esse
o teu destino, esse que há-de levar-te mais longe...- ela queria
animá-lo, deixá-lo cheio de confiança. Percebera que a esperança, tão
habitual nele, já não existia.

- Fazes a mínima ideia do que estás a dizer? Destino? Que espécie de
cliché é esse? Desde quando acreditas nessas coisas?

- Já não acreditas em ti, na tua música?

- Já não sei em que acreditar. Essa é a verdade.

- Toda a gente corria esse risco, ao deixar-se apanhar por Pedro. - Ela
baixou o rosto. Parecia olhar para dentro. Talvez recordasse a pior
parte de Pedro e não quisesse transparecer diante dele. - Ele
encarregava-se de destruir ilusões. Vejo que foste contaminado pela
doença dele. - Agarrou-lhe na mão e beijou-a.

Como uma irmã, com a neutralidade de uma amiga. E continuou:

- Por amor de Deus, afasta-te disso. A vida é renascimento... Porque
complicas tudo? Antigamente rodeavas-te de uma solidão cheia. Bastava-te
a música, o ar que respiravas. Lembras-te? Eras de tal maneira feliz na
tua solidão que eu não sabia em que porta haveria de bater...

Ele respondeu-lhe com um silêncio pesado. Sim, recordava-se de tudo.
Daquele dia em que fora encontrá-la por detrás do penhasco, lavada em
lágrimas. E ele compunha o tempo todo, enquanto ela chorava.

- É bem verdade que só nos apercebemos das coisas depois de as
perdermos.

- Assustas-me.

- Não te preocupes. Isto já passa, é tudo junto. Ainda não encontrei
ninguém que vos substituísse, a ti e ao Pedro.

Ela abriu a porta do carro, ergueu-se com alguma dificuldade, que ele
não estava habituado a ver-lhe.

- Por este caminho, para o mês que vem, nem me atrevo a entrar em carro
nenhum. É uma vergonha. Sinto-me gorda e feia. E Mohammed não ajuda
nada. Passa o tempo todo a dizer-me que devia comer menos...

- Eu acho-te óptima. Recuperas o peso depois.

Ele ficou a vê-la afastar-se. Ela andava de uma maneira estranha e
pesada. Como uma pata pesada. Um andar de grávida que lhe dava imensa
graça. Amou-a, tão diferente do que havia conhecido.

E, involuntariamente, aquela casa onde ela se encontrava agora, ao pé da
praia, recordou-lhe a sua infância.

Regressou, não a casa, onde o esperava a desolação do olhar materno e a
solidão do quarto, mas à praia, onde a tarde mergulhava na escuridão.
Deixou-se estar dentro do carro, perto da casa arruinada da falésia. O
seu espírito recusava-se a ouvir música, sobreveio-lhe um cansaço
inesperado. O sol refulgiu, por instantes, incidindo nas ruínas. Essa
imagem fê-lo voltar atrás no tempo. O tempo da casa envolta no
crepúsculo, em que as aves a rodeavam. Apetecia-lhe demorar-se naquele
horizonte, desfazer-se na luz, dissolver-se no silêncio, em adágio
lento, lentíssimo.

Pela primeira vez, percebeu que pensava em si como uma parte da sua
própria música, não se distanciava dela. Ela vinha e integrava-o nesse
universo feito de ruínas e de memória, que o envolvia por inteiro. A sua
existência material fragmentava-se na leveza da música. Ela abraçava-o.
A vida por dentro dela. Tudo o que fazia parecia-lhe estranho,
excessivo. Sentia que já não podia fugir-lhe. O mais insignificante
gesto, uma simples deslocação de ar esgotavam-no.

Como Ahab, o capitão amaldiçoado, indo ao fundo do mar, preso para
sempre à baleia branca. Ou um Jonas perdido, que era engolido pela
baleia-música. Escura, cheia de escolhos, de objectos que não se
deixavam vislumbrar, num útero que o subtraía ao mundo. Das paredes da
barriga ressumava uma sonoridade que se entranhava no corpo todo. Um
rumor ampliado pelo imenso ouvido que era aquele gigantesco ventre. As
vozes múltiplas que lhe chegavam, vindas de todos os lados e de
nenhures.

Lá fora o mundo talvez fosse transparente, mas ele nada sabia dessa
claridade brutal, insuportável aos seus olhos. As leis da vida, que eram
os algozes dos homens, apertavam-se à sua volta. E, todavia, procurava
ainda lutar contra a escuridão. Soube-se totalmente encarcerado. Entre a
luz e o enigma escuro, a fonte. O paradoxo da criação, em toda a
violência dialética.

Reconheceu, então, que o último obstáculo havia sido ultrapassado. Já
não precisava de procurar a dionisíaca força da criação. Estava dentro
dela e ele próprio era um instrumento desse ventre medonho, sem voz nem
qualquer limite. Informe.

Assaltaram-no os versos de Eliot, em "Burnt Norton", ecoando
poderosamente na sua memória, como passos numa casa silenciosa:\emph{
"Desce mais, desce apenas/Ao mundo da solidão perpétua,/ Mundo não
mundo, mas aquilo que não é mundo". }Depois o vazio, paralisando-o.
Conhecia aquela sensação que ultimamente o acometia tantas vezes,
projectando a sombra da morte sobre tudo, mastigando-o e desfazendo-o.

Fechou os olhos. Sonhou que a morte deixara de existir. Era como se um
outro se tivesse insinuado em si. «Não há um centro», pareciam gritar
todos os lugares dispersos que havia em si.

Devia haver um lugar onde pudesse rezar pensou. Onde houvesse música e
luz, e não a selvática profusão de sons que o arrastava ao fundo de si
próprio, em cântico infernal.

Desta vez, o negrume teimava em persistir. As palavras tinham
desaparecido, tinham-no abandonado. Sentia-se um bárbaro, incapaz de
pensar, de traduzir os seus pensamentos em linguagem que fosse audível.
Sabia que ninguém o entenderia, revertera ao puro estado animal. Apenas
escorria aquele cântico por si, uma cacofonia que teimava em permanecer,
ampliando-se. Uma sinfonia incessante, martelando-lhe os ouvidos, cada
vez mais alto, mais insuportável. Teria de a escrever, para se libertar
dela.

Soube então que tinha pouco tempo para terminar o que tinha entre mãos.
Mesmo que os outros não se apercebessem disso, sentiu-se enlouquecer,
naquela algaraviada infame. De repente desatou a rir, ruidosamente.

Saiu do carro com a cabeça a latejar, procurando o frio, a aragem que
lhe devolvesse um fio de repouso. Não conseguia pensar, apenas ouvia a
música, sempre aquela música, que o conduzia à maior desolação, às
profundezas de um território interdito. Desatou a correr e desceu até à
praia. Já era noite e via a brancura das ondas, que se desfaziam na
areia. A calma desse som acalmou-o um pouco, atenuando-lhe o negrume
interior. Algumas gaivotas voavam por ali, mesmo na rebentação das
ondas.

Avançou pelo mar adentro e lavou o rosto. As lágrimas confundiam-se com
o gosto do sal, que se entranhava nos lábios. Lambeu o sal, tinha
saudades de quando era miúdo e era uma das coisas que mais adorava. O
gesto acalmava-o, levava-o à infância. E percebeu, então, o modo como a
mais sublime beleza pode, de repente, revelar o lado mais insustentável
do horror. Como desejar tanto algo, isso pode destruir-nos, fazer-nos
sofrer ao ponto de já não podermos olhar o que antes havíamos desejado.

Deixou-se estar, deitado na areia, o frio a roer-lhe o rosto, quieto,
inane. Estava morto, morto. O som continuava a martelar-lhe os ouvidos.
Daqui a pouco tudo estaria bem, poderia voltar e ser, como sempre. Só
precisava que a violência da criação o abandonasse, esse tormento que
era movido pela melancolia. Saíra desse ventre onde as coisas escuras
não são nomeáveis e lâminas de fogo lutam contra as palavras e os sons e
onde tudo é incandescente.

Quando chegou a casa, a madrugada ia alta. Entrou em silêncio, para não
incomodar a mãe. Tinha-se transformado num farrapo, um louco. Um outro
lutava em si para se libertar. Um outro irreconhecível, sem leis de
espécie alguma, capaz de tudo para saciar o seu prazer. Sabia-o e
reconheceu-o, ao olhar-se ao espelho.

Envelhecera. Por detrás do seu olhar claro, o outro, o louco espreitava.
Teve medo de si próprio, encolheu-se na cama, tapou-se e chorou como uma
criança. Adormeceu e sonhou com o pai. Num tempo de gaivotas e de luz. O
pai chegava e abria a porta. O sol entrava a rodos pelo quarto,
incendiando tudo. Ele enchia-se daquele amor, onde a alegria era uma
flor delicada. Guardada no silêncio do olhar de Gabriel.

Índice
